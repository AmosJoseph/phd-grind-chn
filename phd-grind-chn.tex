\documentclass[letter,12pt]{book}
\usepackage[T1]{fontenc}
\usepackage[utf8]{inputenc}
\usepackage{CJKutf8}
\usepackage{lmodern}
\usepackage{hyperref}
\usepackage{graphicx}

\setlength{\parindent}{2em}

\linespread{1.3}
% \setlength{\parskip}{1ex}
\setlength{\parskip}{0.5\baselineskip}

\newenvironment{dedication}
{
   \clearpage
   \thispagestyle{empty}
   \vspace*{\stretch{2}}
   \hfill\begin{minipage}[t]{.75\textwidth}
   %\raggedright
}
{
   \end{minipage}
   \vspace*{\stretch{3}}
%    \clearpage
}

\begin{document}
\begin{CJK}{UTF8}{gbsn}
\title{\Huge \textbf{ 博士磨砺 } \\ \huge 一个博士研究生的回忆录 \footnote{英文原文链接:\url{http://www.pgbovine.net/PhD-memoir/pguo-PhD-grind.pdf}}}
% Author
\author{{ \begin{tabular}{ll}
原著 & Philip Guo\\
翻译 & 齐鹏
\end{tabular} }}
\date{}

\renewcommand*\contentsname{目录}  

\frontmatter
\maketitle

\begin{dedication}
献给所有热爱创造的人。
\end{dedication}

\tableofcontents

\mainmatter

\setcounter{chapter}{0}
\newcommand{\mychapter}[1]{
    \addtocounter{chapter}{1}
    \setcounter{section}{0}
    \chapter*{#1}
    \addcontentsline{toc}{chapter}{#1}
}

\renewcommand{\emph}[1]{{\CJKfamily{gkai}#1}}


%%%% Preface

\mychapter{序}

这本书记述了我在斯坦福大学从2006年到2012年,攻读博士研究生期间六年的求学经历。这本书适合广泛的读者群,其中包括:

\begin{itemize}
\item 有志攻读博士研究生\footnote{研究生(graduate student)本是指本科之后的高等教育,通常包括硕士研究生和博士研究生两种学位。现代汉语中“研究生”一词经常被滥用,用以单指硕士研究生,这其实是错误的。——译者注}的本科生;
\item 寻求方向或灵感的在读博士生;
\item 希望更深入了解博士研究生的教授;
\item 希望聘用和管理拥有博士学位员工的雇主;
\item 在充满竞争的创新领域工作、与自我追求和自我激励密不可分的专业人士;
\item 对学术研究充满好奇的有一定教育背景的成年人(或者早熟的青少年)。
\end{itemize}


《博士磨砺》与已有的与博士经历相关的文章在写作形式、写作时机和写作基调上都有所不同:

\paragraph{形式} 《博士磨砺》是一本面向大众的回忆录,而非一本面向在读博士生的“成功指南”。尽管博士生也能在我的经历中学习到经验和教训,但我的目标并不是直接提供建议。对于博士生而言,市面上的“成功指南”和“建议专栏”已经不胜枚举,我也无意画蛇添足。这些文章充满了“持之以恒”和“不积跬步,无以至千里”等等泛泛的词汇,但相反,回忆录的形式让我能丰富、具体地叙述发生在我自己身上的故事。

\paragraph{时机} 《博士磨砺》是我在完成博士学位之后马上着手写作的,而这正是撰写这样一本回忆录的最佳时机。不同于在读博士生的是,我可以在回忆录中对整个博士求学期间的工作进行系统的整理和反思;而相比资历较深的研究者而言,我可能更容易忠于攻读期间的经历,不会引入过多由研究经历带来的有选择性的观点和感受。


\paragraph{基调} 尽管保持完全客观是不可能的,但我在写作《博士磨砺》过程中仍然试图贯穿一种相对客观的基调。与其他作者不同的是,很多撰写博士相关文章、书籍,或者绘制漫画的人通常属于以下两类之一:
\begin{itemize}
\item 成功的教授或者科学家,他们通常给出一些冠冕堂皇的建议,比如他们可能会说:“研究生生活诚然辛苦,但它同时应该是一段美好的知识之旅,你应该享受这个过程、并从中受益……因为我当年就是这样做的!”
\item 或者是苦涩的博士研究生或者辍学博士,他们常常因为自己的经历留下了心理阴影,当提及博士生活的时候会用一种过分夸张、“看破一切”、自我怨恨的腔调:“啊,那时我的世界就是个活生生的地狱,我到底拿我的青春换了什么?!?”
\end{itemize}

冠冕堂皇的建议可能能激励一些学生,而大倒苦水的呻吟可能能引起另一些处境不佳的学生的共鸣,但作为大众读者而言,他们可能并不能感受到这些极端的情绪。

最后,在我开始讲述自己的故事之前,我希望强调一下,每个博士研究生的经历与他/她所在的学校、院系、研究领域、经费状况等都有巨大的关系。在我的读博生涯中,我感到自己非常幸运,能很大程度上自由自愿地完成自己的学业;我知道很多学生相比而言受到了很多的限制。我的故事只是一个孤立的数据点,所以我所呈现的故事可能并不能泛化、推广到每一个人。然而,我会尽力避免叙述变得过分局限于我个人的情况。

祝阅读愉快!

\begin{flushright}
Philip Guo, 2012年6月
\end{flushright}

%%% Prologue

\mychapter{前言}

因为我本科的专业是电子工程和计算机科学,大部分本科同学在毕业之后,都马上投入到了工程性的工作中了。而我最终选择攻读博士学位,究其原因,一方面是受到来自父母潜移默化的影响,另一方面也和我在本科期间对工程性的实习产生的不良印象不无关系。

我父母从未要求过我去攻读博士学位,但我可以看出,终身大学教授\footnote{在美国和加拿大等国家,终身职位(tenure)是指资深学者拥有的,除因正当理由外,免于被解雇的合同权利。与此不同的是任意性职位(at-will),规定劳动双方随时可以以任何理由终止劳动合同,不需要法定的正当理由,也不需要提前通知。大多数非学术研究类(工程性)的工作属于后者。——译者注}是他们最尊重的职业——而博士学位正是成为终身教授的必要条件。为什么终身教授会是他们心目中的理想职业呢?这其实并不是因为他们对纯粹的学术追求有什么不切实际的盲目推崇。尽管我的父母也都到过良好的教育,但他们同时也是非常现实的移民——一份终身教授的职位对他们而言,更大的吸引力在于终身制所带来的工作保障。

我父母的很多朋友都是在企业的工程性职位上供职的中国移民。由于他们在英语技巧和美国文化了解上的不足,他们中的大多数人在职业生涯中的经历都并不顺利,而这一问题往往随着年龄增长而更加突出。在假日聚会上,我经常能听到一个不变的主题:人们的工作因为难以相处的经理而处处不顺、成为年龄歧视和“玻璃天花板”效应\footnote{玻璃天花板(Glass Ceiling)效应是一个政治术语,用于形容“在企业内升职过程中看不到但难以逾越的障碍,通常见于少数人种和女性身上,且与这些人群的资历和成就并无关联”的现象。——译者注}的牺牲品、甚至面临大规模裁员和长期失业的风险。尽管我父亲不是一位工程师,而是在高技术产业就职,但他也难逃这种魔咒,在一系列和管理层极其官僚做派的斗争中失败,最终早早地结束了他在公司的工作。那时他还相对很年轻,只有45岁。

我的母亲则是这种不幸潮流中唯一的幸存者。她十分热爱自己在UCLA\footnote{UCLA是加州大学洛杉矶分校(University of California, Los Angeles)的简称。——译者注}作为社会学终身教授的工作。和她的大部分中国移民朋友不同的是,她享受终身职务保障,不用向老板汇报工作,可以几乎完全自由地追求她自己的学术兴趣,在她的研究领域也小有名气。亲眼目睹我母亲成功的职业轨迹,和父亲及他们的朋友们在职业生涯上的恶性循环两者之间的巨大反差,这在我高中和大学本科的学习生涯中留下了难以磨灭的印象。

当然,仅仅因为这种非理性的、年少时的恐惧就去选择读博显然是不明智的。为了让自己对在企业的工作生活有所印象,我在本科的每个假期都参加了工程性公司的实习。因为我工作过的办公室都碰巧只有我一个实习生,我被赋予了一种罕有的特权——我的工作职责是按照全职初级工程师的标准分配的。尽管在这个过程中我学到了很多技术上的技巧,我仍然觉得这种日复一日的工作极少需要思考且非常无聊;这也可能与我实习过的公司不是一流公司有关。我本科的很多朋友都在微软和谷歌等一流公司实习过,并非常喜欢他们的实习经历,他们最终也往往在毕业后和这些公司签下了全职工作合同。

因为我对我的实习经历感到厌倦,而另一方面对本科时作教学助理(助教)和研究助理(助研)的经历比较感兴趣,我当时将未来的职业目标定在了大学教学和学术研究上。等到我在MIT\footnote{MIT是马萨诸塞州理工学院(又译麻省理工学院,Massachusetts Institute of Technology)的简称。——译者注}的第三年过半,我已经下定决心在毕业后攻读博士学位,因为这是实现我职业目标的必经之路。我决定留在MIT,完成一个五年制本硕连读的项目,以此来在申请博士项目之前积累更多的研究经验,以期能被录取到更多排名顶尖的院系中去。

我找到了一位硕士论文的导师,并且就像很多壮志踌躇的年轻人一样,开始向他阐述自己不甚成熟、遑论完善的研究思路和计划。我的导师很耐心的听我说完了我的想法,但最终仍然成功地说服我进行一些和他的研究兴趣更契合、更主流的研究,而且更重要的是,这些研究项目更契合他的基金项目要求。因为当时我的硕士项目学费一部分来自我的导师从美国政府申请的一个\emph{研究基金},我有义务在基金所规定的范围内完成研究工作。因此,我听从了他的建议,并将接下来两年半的时间用于开发一类新的原型工具,用于分析由C和C++语言编写的计算机程序的运行时行为。\footnote{运行时行为分析(run-time behavior analysis)是计算机程序编写和维护中一种重要的技术,它可以帮助软件工程师更快、更准确地发现在程序编写时难以注意到的程序缺陷和漏洞(bug),进而完善其功能并提高稳定性。——译者注}

虽然我并不是非常热衷于我硕士论文的项目,但事实证明选择一个和导师研究兴趣契合的项目是一个明智的选择:在他有力的指导下,我发表了两篇论文——一篇我被列为第一作者(主要作者),另一篇的位置稍微靠后——并且我的硕士论文获得了系里年度\emph{最佳毕业论文奖}。这些成就,加上导师在我申请文档中的帮助,为我赢得了几所顶尖计算机科学系博士项目的录取。因为斯坦福是我的首选,在我收到录取的当天晚上,我甚至激动得几乎无法入睡。

我还非常幸运地得到了NSF\footnote{NSF是美国国家自然基金会(National Science Foundation)的简称。NSF是支持美国各大高等院校进行基础自然科学研究的重要基金来源之一。该奖学金的申请只面向美国公民及永久居民开放。——译者注}和NDSEG\footnote{NDSEG是美国国防科学与工程研究生奖学金(National Defense Science \& Engineering  Graduate Fellowship)的简称。该奖学金由美国国防部出资设立,旨在推动国防相关的科学与工程研究的发展,其申请只面向持有美国国籍的人士开放。——译者注}研究生奖学金的垂青,这两者都只颁发给了大概5\%的申请者。这两个奖学金为我免除了攻读博士的六年之中,五年的全部费用,也使我不必完成各种研究基金相关的研究项目。与我不同的是,在我的研究领域中,大部分博士研究生依赖于教授提供的研究经费和院系提供的助教经费。博士研究生的经费包括全额学费,以及大约每个月1,800美元的补贴,用于贴补生活开支。(在我的研究领域几乎没有人自费攻读博士,因为那样在经济上非常不划算。)

因为我已经有一定的研究和写作论文的经验,当我在2006年9月来到斯坦福时,我觉得自己已经为未来艰苦的博士研究做好了充足的准备。然而,那时我完全没有预见到的是,我的博士第一年即将成为我生命中到此为止最为打击信心、令人灰心丧气的一段时间。

%%% Year One: Downfall

\mychapter{第一年:失落}

2006年的夏天,在我开始在斯坦福攻读博士学位的几个月前,我考虑了一些我认为自己感兴趣研究的课题。大题而言,我想要创造一些创新性的工具,用来帮助人们在进行计算机编程时提高效率,换言之,\emph{提高程序员生产力}。我之所以对这个方向感兴趣,主要源于我在暑期实习中自己的编程经历:因为日复一日,公司分配给我的工作并不能让我提起太多兴趣,工作中的很多时间,我都坐在自己的格子间里反思,在我工作的这些公司中计算机编程的流程如何低效。那时我认为,如果能投身于旨在降低这种低效性的科学研究,应该是不错的方向。更宽泛地说,我的研究兴趣集中在能让所有类型的计算机使用者更加高效的工作中——而非仅仅聚焦在专业程序员的身上。举例来说,我希望能设计出新的工具,能帮助科学家分析和绘制数据、帮助系统管理员调整服务器配置、或者帮助计算机新手学习使用新的软件。



\end{CJK}
\end{document}