% !TEX program = XeLaTeX
% !TEX encoding = UTF-8
\documentclass[12pt,UTF8,nofonts]{book}
\usepackage{xeCJK}
\setCJKmainfont[BoldFont=STXihei]{STSong}
\setCJKfamilyfont {gkai} {STKaiti}

\usepackage[T1]{fontenc}
\usepackage[utf8]{inputenc}
\usepackage{lmodern}
\usepackage{hyperref}
\usepackage{graphicx}
\usepackage{indentfirst}

%\let\cleardoublepage\clearpage



\setlength{\parindent}{2.4em}

\linespread{1.3}
% \setlength{\parskip}{1ex}
\setlength{\parskip}{0.5\baselineskip}

\newenvironment{dedication}
{
   \clearpage
   \thispagestyle{empty}
   \vspace*{\stretch{2}}
   \hfill\begin{minipage}[t]{.7\textwidth}
   %\raggedright
   \CJKfamily{gkai}
}
{ 
   \end{minipage}
   \vspace*{\stretch{3}}
%    \clearpage
}

\newcommand{\bookname}{博士研磨}

\begin{document}

\title{\Huge \textbf{ \bookname } \\ \large \textbf{一個博士研究生的回憶錄} \footnote{英文原文鏈接:\url{http://www.pgbovine.net/PhD-memoir/pguo-PhD-grind.pdf}。 本翻譯基於原文2012 年7月16日發布的版本譯成。}}
% Author
\author{{ \begin{tabular}{lll}
原著 & Philip J. Guo & (\texttt{philip@pgbovine.net})\\
翻譯 & 齊鵬 & (\texttt{pengqi@cs.stanford.edu})\\
     & 羅宇男 & (\texttt{luoyunan@gmail.com})
\end{tabular} }}
\date{}

\renewcommand*\contentsname{目錄}

\frontmatter
\maketitle

\begin{dedication}
獻給所有熱愛創造的人。
\end{dedication}

\tableofcontents
\markboth{{\CJKfamily{gkai}目錄}}{{\CJKfamily{gkai}\bookname}}

\mainmatter

\renewcommand{\emph}[1]{{\CJKfamily{gkai}#1}}
\newcommand{\breakline}[0]{\begin{center}$\sim$\end{center}}
\newcommand{\prologuesection}[1]{\begin{center}\large{\CJKfamily{gkai}#1}\end{center}}

\setcounter{chapter}{0}
\newcommand{\mychapter}[1]{
    \addtocounter{chapter}{1}
    \setcounter{section}{0}
    \chapter*{#1}
    \markboth{{\CJKfamily{gkai}#1}}{{\CJKfamily{gkai}\bookname}}
    \addcontentsline{toc}{chapter}{#1}
}

%%% Disclaimer
\markboth{}{}
\section*{Disclaimer}

This is an unauthorized Chinese translation of Philip J. Guo's memoir \emph{The Ph.D. Grind}, and the original author did not have any input on the translation.

The copyright of the content belongs to the original author, and the translation to the translator. This work may not be used for business purposes, and may only be used as non-commercial material.

Apart from the Translator's Preface, none of the content of this work represents, none should be interpreted as, the opinion of the translator; nor will the translator be responsible for the consequences of any interpretation of this translation.

\section*{聲明}

本作品是Philip J. Guo的回憶錄《The Ph.D. Grind》的中文翻譯,翻譯並沒有得到原作者的任何授權,原作者也並沒有以任何形式參與到翻譯過程中。

本作品內容的版權歸原作者所有,翻譯版本歸譯者所有。本作品不得被用於任何商業目的,只能作為非盈利性材料傳播。

除譯序外,本作品的一切內容均不代表——且不應被認為是——譯者的觀點;譯者對此翻譯的任何理解以及其導致的影響亦不負有任何責任。

%\clearpage

\mainmatter
%%%% Preface

\chapter*{序}
\markboth{}{}

這本書記述了從2006年到2012年,我在斯坦福大學攻讀博士研究生期間六年的求學經歷。這本書適合廣泛的讀者群,其中包括:

\begin{itemize}
\item 有誌攻讀博士研究生\footnote{研究生(graduate student)本是指本科之後的高等教育,通常包括碩士研究生和博士研究生兩種學位。現代漢語中“研究生”一詞經常被濫用,用以單指碩士研究生,這其實是錯誤的。——譯者註}的本科生;
\item 尋求方向或靈感的在讀博士生;
\item 希望更深入了解博士研究生的教授;
\item 希望聘用和管理擁有博士學位員工的雇主;
\item 在充滿競爭的創新領域工作、與自我追求和自我激勵密不可分的專業人士;
\item 對學術研究充滿好奇的有一定教育背景的成年人(或者早熟的青少年)。
\end{itemize}


《\bookname》與已有的與博士經歷相關的文章在寫作形式、寫作時機和寫作基調上都有所不同:

\paragraph{形式} 《\bookname》是一本面向大眾的回憶錄,而非一本面向在讀博士生的“成功指南”。盡管博士生也能在我的經歷中學習到經驗和教訓,但我的目標並不是直接提供建議。對於博士生而言,市面上的“成功指南”和“建議專欄”已經不勝枚舉,我也無意畫蛇添足。這些文章充滿了“持之以恒”和“不積跬步,無以至千裏”等等空泛的詞匯,但相反,回憶錄的形式讓我能豐富、具體地敘述發生在我自己身上的故事。

\paragraph{時機} 《\bookname》是我在完成博士學位之後馬上著手寫作的,而這正是撰寫這樣一本回憶錄的最佳時機。不同於在讀博士生的是,我可以在回憶錄中對整個博士求學期間的工作進行系統的整理和反思;而相比資歷較深的研究者而言,我可能更容易忠於攻讀期間的經歷,不會引入過多由研究經歷帶來的有選擇性的觀點和感受。


\paragraph{基調} 盡管保持完全客觀是不可能的,但我在寫作《\bookname》過程中仍然試圖貫穿一種相對客觀的基調。與其他作者不同的是,很多撰寫博士相關文章、書籍,或者繪制漫畫的人通常屬於以下兩類之一:
\begin{itemize}
\item 成功的教授或者科學家,他們通常給出一些冠冕堂皇的建議,比如他們可能會說:“研究生生活誠然辛苦,但它同時應該是一段美好的知識之旅,你應該享受這個過程、並從中受益……因為我當年就是這樣做的!”
\item 或者是苦澀的博士研究生或者輟學博士,他們常常因為自己的經歷留下了心理陰影,當提及博士生活的時候會用一種過分誇張、“看破一切”、自我怨恨的腔調:“啊,那時我的世界就是個活生生的地獄,我到底拿我的青春換了什麼?!?”
\end{itemize}

冠冕堂皇的建議可能能激勵一些學生,而大倒苦水的呻吟可能能引起另一些處境不佳的學生的共鳴,但作為大眾讀者而言,他們可能並不能感受到這些極端的情緒。

最後,在我開始講述自己的故事之前,我希望強調一下,每個博士研究生的經歷與他/她所在的學校、院系、研究領域、經費狀況等都有巨大的關系。在我的讀博生涯中,我感到自己非常幸運,能很大程度上自由自願地完成自己的學業;我知道很多學生相比而言受到了很多的限制。我的故事只是一個孤立的數據點,所以我所呈現的故事可能並不能泛化、推廣到每一個人。然而,我會盡力避免敘述變得過分局限於我個人的情況。

祝閱讀愉快!

\begin{flushright}
Philip Guo, 2012年6月
\end{flushright}

\chapter*{譯者序}
\markboth{}{}

\prologuesection{巫術與火刑}

可能最經常被家人朋友問到的問題之一,就是:“你說你在做研究、想讀博,那讀博士、做研究到底是在做什麼?”我一直想不到一個很好的答案,直到讀了Philip Guo的The Ph.D. Grind。

有時候,讀博做研究在大眾眼裏,就好像中世紀早期歐洲的醫學。彼時的人們仍然處於蒙昧狀態,現代人習以為常的生物和化學知識,在那時仍大多是未知的領域。在戰亂和饑荒之外,最多奪去人們生命的,就是疾病。當時的社會主流仍然認為身體不適是由惡魔侵襲造成的,而對此的對策也往往只是信仰治療。與此同時,一個不被主流社會所認可的群體開始形成,在使用信仰治療之外,他們開始嘗試用各種草藥對付不同的病癥。這種和當時社會常識截然不同的治療方法盡管比單純的信仰治療有效許多,但因為它和人們廣為接受的宗教信仰有所沖突,所以大多巫醫的治療都在地下進行。巫醫們似乎也不願意把自己的草藥知識普及到大眾之中,所以他們草藥知識體系的大部分都在自己的群體中形成,知識的累積和傳授也通常在巫醫們的地下集會中傳播。沒有不透風的墻,這種和教廷宣揚的教義沖突的“巫術”很快就被封禁了,很多參與者也被處以了當時的極刑——火刑——意在徹底消滅這些“惡魔”。可能迫於宗教勢力的壓迫,並沒有人挺身而出救下這些曾經救死扶傷的巫醫,而受此影響,傳統醫學的發展也大大受到了阻滯。

或許除了教會的壓制之外,另一個阻止人們挺身而出的因素,正是他們對巫醫治療原理的不了解。作為少數掌握當時較高醫學知識的群體,巫醫迫於種種現實因素選擇了將自己的知識保存在相對狹窄的群體中,並沒有試圖讓更多普通人認識到他們對疾病的深入理解和反復試驗累積的經驗。這其實某種意義上和今天的象牙塔別無二致。受到高等教育的人群往往不願意、甚至不屑於與大眾分享自己專業領域的知識或者技能,從而兩者之間通常會產生隔閡與不信任。當然今天的世界已不存在火刑這樣殘忍的刑罰,但對大眾的不了解、不認同、甚至不信任,象牙塔裏的學者們不能說毫無責任。

《\bookname》雖然並不是直接以此為目的的,但為大眾了解象牙塔內的生活也能提供第一手資料,是一本不可多得的回憶錄。也正是出於這個考慮,我才決定把整本書翻譯成中文,希望能讓更多英語閱讀有障礙的中文母語者能獲得同樣的體驗。

\prologuesection{研磨的故事}

在動手開始翻譯之前,已經在許多中文材料中聽說了The Ph.D. Grind,也見到過未完成的翻譯作品。不同的材料中,對書名中grind一詞的翻譯往往見仁見智。

\begin{quote}
Grind,原意研磨或者碾壓,指將物體細細碾成碎片或者粉末。後來這一詞匯被引申為磨礪(通過摩擦使物體變得鋒利或光滑),或者與研磨、磨礪相類似的,需要長期努力的苦工。

——翻譯自Merriam-Webster字典釋義
\end{quote}

一個相對廣為接受的翻譯似乎是《博士磨礪》,想必當時的譯者取義“寶劍鋒從磨礪出,梅花香自苦寒來”的暗喻,喻指博士生涯是一種對品格、性情、毅力、素養的磨練,當苦盡甘來時,回望來路彼時種種磨練都塑造了此時的自己。這種看法不無可取之處,它給人以希望,讓我們提前看到這本回憶錄應該有一個圓滿的結局,而之前的一切苦難和不幸,都是最後成功之花綻放的鋪墊。

而另一種廣為流傳的翻譯似乎是另一個極端,即《博士磨難》。這位譯者可能對書中描述的科研中遇到的困難和點滴都深有同感,也為我們還原了一個博士生在求學過程中經常看到的實景:真理的高峰往往在迷霧中不知所蹤,而就算有幸看到了峰頂的光芒,攀登的過程也絕非一路康莊。

剛剛開始翻譯的時候,我也曾經為grind這個詞的翻譯大傷腦筋。在翻譯書名的時候想當然地借用了“磨礪”這個更加激勵人心的翻譯,但當翻譯中一次次遇到這個詞在正文中出現時,我開始質疑自己的選擇。在整本回憶錄中,grind更像是一個中性詞,沒有“磨礪”這種事後回望,一切苦功都沒白費的慶幸,也鮮有“磨難”這種抱怨眼前困難,自悲自憐的消極。

為了盡量終於原文的閱讀體驗,並盡量保持正文和書名中的grind翻譯一致,我最終還是選擇了“研磨”這個更接近詞語原意,也不具有明顯感情傾向的翻譯。盡管這個詞在現代漢語中並沒有直接用作“辛苦工作、歷練、磨礪”的意思,我仍真切地希望讀者能理解這個選擇。

\prologuesection{獻給所有熱愛創造的人}

正如原作者在回憶錄最後強調的,讀博士帶來的磨礪、成長和創造的樂趣並不只有讀博士這一條路可以帶來,不同人在不同的人生道路上可能都有相似的體驗。尤其是那些熱愛創造的人,他們在路上往往經歷著相似的艱難險阻:因為從零到一往往是充滿挑戰的,即使眼前有確定的目標,前進的路上也決難以預料會有怎樣未知的問題在等待解決。但他們又往往有著同樣堅韌不拔的品格:盡管跋山涉水、披荊斬棘,經歷過無數次失敗和重來,無數次跌倒和重新啟程,無數次止步不前的沮喪和取得進展的喜悅,他們仍然朝著心中的那個目標努力著。

借原作者致禮的這句話,再次表達對他們的敬意:\emph{獻給所有熱愛創造的人。}

最後,也借此機會感謝背後一直支持我的家人、女友和朋友們,你們一直是我前進路上最大的動力和最堅實的依靠。還要由衷感謝羅宇男參與了本書後半段的翻譯工作,沒有他的辛勤工作,這份完整的翻譯稿也很難現在完成。其它在工作、生活和翻譯上曾經給予過我鼓勵、指導、建議和支持的朋友們,在此也一並感激。
\\[3em]
\begin{flushright}
齊鵬\\
2015年4月16日\ 於斯坦福
\end{flushright}

\mainmatter

%%% Prologue

\mychapter{前言}

因為我本科的專業是電子工程和計算機科學,大部分本科同學在畢業之後,都馬上投入到了工程性的工作中了。而我最終選擇攻讀博士學位,究其原因,一方面是受到來自父母潛移默化的影響,另一方面也和我在本科期間對工程性的實習產生的不良印象不無關系。

我父母從未要求過我去攻讀博士學位,但我可以看出,終身大學教授\footnote{在美國和加拿大等國家,終身職位(tenure)是指資深學者擁有的,除因正當理由外,免於被解雇的合同權利。與此不同的是任意性職位(at-will),規定勞動雙方隨時可以以任何理由終止勞動合同,不需要法定的正當理由,也不需要提前通知。大多數非學術研究類(工程性)的工作屬於後者。——譯者註}是他們最尊重的職業—— 而博士學位正是成為終身教授的必要條件。為什麼終身教授會是他們心目中的理想職業呢?這其實並不是因為他們對純粹的學術追求有什麼不切實際的盲目推崇。盡管我的父母也都到過良好的教育,但他們同時也是非常現實的移民——一份終身教授的職位對他們而言,更大的吸引力在於終身制所帶來的工作保障。

我父母的很多朋友都是在企業的工程性職位上供職的中國移民。由於他們在英語技巧和美國文化了解上的不足,他們中的大多數人在職業生涯中的經歷都並不順利,而這一問題往往隨著年齡增長而更加突出。在假日聚會上,我經常能聽到一個不變的主題:人們的工作因為難以相處的經理而處處不順、成為年齡歧視和“玻璃天花板”效應\footnote{玻璃天花板(Glass Ceiling)效應是一個政治術語,用於形容“在企業內升職過程中看不到但難以逾越的障礙,通常見於少數人種和女性身上,且與這些人群的資歷和成就並無關聯”的現象。—— 譯者註}的犧牲品、甚至面臨大規模裁員和長期失業的風險。盡管我父親不是一位工程師,而是在高技術產業就職,但他也難逃這種魔咒,在一系列和管理層極其官僚做派的鬥爭中失敗,最終早早地結束了他在公司的工作。那時他還相對很年輕,只有45歲。

我的母親則是這種不幸潮流中唯一的幸存者。她十分熱愛自己在UCLA\footnote{UCLA 是加州大學洛杉磯分校(University of California, Los Angeles)的簡稱。——譯者註}作為社會學終身教授的工作。和她的大部分中國移民朋友不同的是,她享受終身職務保障,不用向老板匯報工作,可以幾乎完全自由地追求她自己的學術興趣,在她的研究領域也小有名氣。親眼目睹我母親成功的職業軌跡,和父親及他們的朋友們在職業生涯上的惡性循環兩者之間的巨大反差,這在我高中和大學本科的學習生涯中留下了難以磨滅的印象。

當然,僅僅因為這種非理性的、年少時的恐懼就去選擇讀博顯然是不明智的。為了讓自己對在企業的工作生活有所印象,我在本科的每個假期都參加了工程性公司的實習。因為我工作過的辦公室都碰巧只有我一個實習生,我被賦予了一種罕有的特權—— 我的工作職責是按照全職初級工程師的標準分配的。盡管在這個過程中我學到了很多技術上的技巧,我仍然覺得這種日復一日的工作極少需要思考且非常無聊;這也可能與我實習過的公司不是一流公司有關。我本科的很多朋友都在微軟(Microsoft)和谷歌(Google)等一流公司實習過,並非常喜歡他們的實習經歷,他們最終也往往在畢業後和這些公司簽下了全職工作合同。

因為我對我的實習經歷感到厭倦,而另一方面對本科時作教學助理(助教)和研究助理(助研)的經歷比較感興趣,我當時將未來的職業目標定在了大學教學和學術研究上。等到我在MIT\footnote{MIT是麻省理工學院(Massachusetts Institute of Technology)的簡稱。——譯者註}的第三年過半,我已經下定決心在畢業後攻讀博士學位,因為這是實現我職業目標的必經之路。我決定留在MIT,完成一個五年制本碩連讀的項目,以此來在申請博士項目之前積累更多的研究經驗,以期能被錄取到更多排名頂尖的院系中去。

我找到了一位碩士論文的導師,並且就像很多壯誌躊躇的年輕人一樣,開始向他闡述自己不甚成熟、遑論完善的研究思路和計劃。我的導師很耐心的聽我說完了我的想法,但最終仍然成功地說服我進行一些和他的研究興趣更契合、更主流的研究,而且更重要的是,這些研究項目更契合他的基金項目要求。因為當時我的碩士項目學費一部分來自我的導師從美國政府申請的一個\emph{研究基金},我有義務在基金所規定的範圍內完成研究工作。因此,我聽從了他的建議,並將接下來兩年半的時間用於開發一類新的原型工具,用於分析由C 和C++ 語言編寫的計算機程序的運行時行為。\footnote{運行時行為分析(run-time behavior analysis)是計算機程序編寫和維護中一種重要的技術,它可以幫助軟件工程師更快、更準確地發現在程序編寫時難以註意到的程序缺陷和漏洞(bug),進而完善其功能並提高穩定性。Bug一詞的原意是蟲子,現代被用於形容軟件缺陷。這一說法一個可能的來源,是源於1947 年電子管計算機剛剛問世前後。當時在哈佛大學的二號機(Mark II)中,一位操作員仔細排查後發現,二號機當時一些計算謬誤的產生,是由於一只飛蛾被困在了一個繼電器上。這件事被記錄在了當時的實驗室日誌中,後來常被稱為“第一次真正找到的bug(此處為蟲子和程序缺陷的雙關)”。——譯者註}

雖然我並不是非常熱衷於我碩士論文的項目,但事實證明選擇一個和導師研究興趣契合的項目是一個明智的選擇:在他有力的指導下,我發表了兩篇論文——一篇我被列為第一作者(主要作者),另一篇的位置稍微靠後——並且我的碩士論文獲得了系裏年度\emph{最佳畢業論文獎}。這些成就,加上導師在我申請文檔中的幫助,為我贏得了幾所頂尖計算機科學系博士項目的錄取。因為斯坦福是我的首選,在我收到錄取的當天晚上,我甚至激動得幾乎無法入睡。

我還非常幸運地得到了NSF\footnote{NSF是美國國家自然基金會(National Science Foundation)的簡稱。NSF是支持美國各大高等院校進行基礎自然科學研究的重要基金來源之一。該獎學金的申請只面向美國公民及永久居民開放。——譯者註}和NDSEG\footnote{NDSEG是美國國防科學與工程研究生獎學金(National Defense Science \& Engineering  Graduate Fellowship)的簡稱。該獎學金由美國國防部出資設立,旨在推動國防相關的科學與工程研究的發展,其申請只面向擁有美國國籍的人士開放。——譯者註}研究生獎學金的垂青,這兩者都只頒發給了大概5\%的申請者。這兩個獎學金為我免除了攻讀博士的六年之中,五年的全部費用,也使我不必完成各種研究基金相關的研究項目。與我不同的是,在我的研究領域中,大部分博士研究生依賴於教授提供的研究經費和院系提供的助教經費。博士研究生的經費包括全額學費,以及大約每個月1,800美元的補貼,用於貼補生活開支。(在我的研究領域幾乎沒有人自費攻讀博士,因為那樣在經濟上非常不劃算。)

因為我已經有一定的研究和寫作論文的經驗,當我在2006年9月來到斯坦福時,我覺得自己已經為未來艱苦的博士研究做好了充足的準備。然而,那時我完全沒有預見到的是,我的博士第一年即將成為我生命中到此為止最為打擊信心、令人灰心喪氣的一段時間。

%%% Year One: Downfall

\mychapter{第一年:碎夢大道}

2006年的夏天,在我開始在斯坦福攻讀博士學位的幾個月前,我考慮了一些我認為自己感興趣研究的課題。大體而言,我想要創造一些創新性的工具,用來幫助人們在進行計算機編程時提高效率,換言之,\emph{提高程序員生產力}。我之所以對這個方向感興趣,主要源於我在暑期實習中自己的編程經歷:因為日復一日,公司分配給我的工作並不能讓我提起太多興趣,工作中的很多時間,我都坐在自己的格子間裏反思,在我工作的這些公司中計算機編程的流程如何低效。那時我認為,如果能投身於旨在降低這種低效性的科學研究,應該是不錯的方向。更寬泛地說,我的研究興趣集中在能讓所有類型的計算機使用者更加高效的工作中——而非僅僅聚焦在專業程序員的身上。舉例來說,我希望能設計出新的工具,能幫助科學家分析和繪制數據、幫助系統管理員調整服務器配置、或者幫助計算機新手學習使用新的軟件。

盡管我當時就有這些模糊、不成型的興趣,但距離我將這些興趣轉化為真正可以發表的研究項目,並最終形成一篇\emph{博士論文},還有很多年的差距。對斯坦福計算機科學系的博士研究生而言,通常他們需要發表二到四篇第一作者的學術論文,並將這些論文合並成一篇博士論文——通常是一篇長度與書籍相仿的科技文檔。當博士論文通過一個由三位教授組成的\emph{博士論文委員會}批準後,學生就可以畢業,從而獲得博士學位了。在我所在的計算機系,通常一個博士研究生需要四到八年畢業,這個年限取決於他們發表文章的效率。

在2006年9月的新生信息會\footnote{信息會(Orientation),或譯為迎新會,是在學校或大型組織中常見的,統一為新成員提供常用信息、幫助其盡快適應環境的活動,通常由資深成員主持。}上,系裏的教授鼓勵所有的博士新生盡快找到自己的\emph{導師},所以我和我的同學們一樣,把一開始的幾個月花在了找教授談話上,希望盡快找到一個研究方向匹配的導師。對於一個學生的論文委員會來說,導師扮演了最重要的角色,因為他/她對學生能否畢業擁有最終的決定權。在我的研究領域,導師通常還負責通過自己的科研經費為學生提供經費支持,並且指導他們開展課題研究、寫作論文。和幾位教授談話後,我發現Dawson\footnote{Dawson是這位教授的名字(first name),而非姓(last name),故翻譯成Dawson 教授是不準確的——原文也沒有出現Professor Dawson(Dawson 教授)這一稱呼。單獨使用名字在英語中非常普遍,通常用於熟人之間互相稱呼、陌生人之間互相介紹、以及不希望提及全名的場合。文中出現的人名都是單獨的名字。為了忠於原文的閱讀體驗,譯者沒有對這些名字進行標準化翻譯和加工。—— 譯者註}和我的研究興趣和研究風格都似乎最為接近,所以我選擇了他作為我的導師。

在我剛去斯坦福的時候,Dawson已經在斯坦福度過了八年,並且剛剛獲得終身教授職位;通常,教授在他們工作的前七年如果發表了足夠多高水平的研究論文,就可以得到\emph{終身職位}(終生的工作保障)。Dawson的主要研究興趣是創造新的工具以自動在復雜、真實的軟件中尋找\emph{bug}(軟件代碼中的缺陷)。在過去的十年間,Dawson 和他的學生編寫了許多這樣的工具,相比他們的競爭者而言,他們能找到程序中更多的bug。他們的研究成果十分有效——他們甚至成立了一個成功的創業公司,通過提供基於這類技術的缺陷檢測服務而盈利。盡管我對Dawson的研究項目感興趣,更吸引我的一點則是他的研究哲學和我自己的想法十分接近:他是充滿激情的“務實派”——相比於單純為了顯得“學術”而去研究理論上“新穎”的課題,他更關註的是得到實實在在、有說服力的結果。

我和Dawson第一次面談時,他似乎對我的大方向——讓計算機的使用和計算機編程變得更高效——只是稍感興趣。不過,有一點他說的很清楚:他非常希望招收一些新的研究生來幫助他完成一個叫\emph{Klee}的軟件缺陷檢查工具,因為這個項目是他現在的科研經費所支持的。(這個工具有好幾個名字,但為了簡便,這裏就叫它“Klee”。)和其他的教授和高年級博士談過之後我才意識到,對於新生而言,加入一個已有的、由研究基金支持的項目是一種常規現象,而並不是馬上開始進行自己原創的研究項目。我說服了自己,認為自動查找軟件缺陷也是一種間接提高程序員工作效率的研究,於是便打定主意,加入Klee項目組。

當2006年12月,我準備加入Klee項目組時,Dawson已經在指導五個學生參加這個項目了。項目組的帶頭人Cristi是第三年的博士生,而正是他和Dawson開發了最初版本的Klee。Dawson、Cristi和其他研究者不久前還發表了第一篇說明Klee系統的論文,並展示了Klee在發現一些新的缺陷上十分有效。那篇論文受到了學術界的好評,所以Dawson 希望保持這一勢頭,繼續發表幾篇跟進這一項目的文章。值得註意的是,從同一個研究項目中發表多篇論文是可能的(即“跟進論文”),只要這些新的論文有新的原創觀點,相比前作的改進和創新,或者相比前作而言在結果上有很大的改善。當時,下一個相關領域的\emph{頂級會議}的論文提交截止日期是2007年3月,所以Klee 團隊有四個月用於基於之前的論文做出創新性的改進,以期發表一篇新論文。

\breakline

在我繼續講述我的故事之前,我想簡單地介紹一下學術論文是如何評審和發表的。在計算機科學領域,發表文章最受關註的場合是\emph{學術會議}。當然,值得指出的是,再很多其他學科中,\emph{學術期刊}才是最受關註的,而對這些領域而言,“學術會議”往往和計算機科學領域大相徑庭。對計算機科學而言,學術會議的論文發表流程大概如下:

\begin{enumerate}
\item 每個會議發布一個征稿啟事,其中說明了會議所要求的課題範圍和一個論文提交的截止日期。
\item 研究者需要在指定的截止日期之前提交自己的論文。通常每個學術會議會收到100 到300份論文稿件,每篇論文大概包含30到40頁雙倍行距\footnote{雙倍行距排版時,行間距與文字高度相同,通常用於學術期刊初稿的排版。而計算機科學的學術論文通常采用單倍行距、雙欄排版。——譯者註}的文字。
\item 學術會議的程序委員會(Program Comittee,簡稱PC)通常由大約20位專家研究者組成,他們負責將論文負責分類,以便審稿。每篇論文通常由三到五個人完成評審,參與評審的人員可能包括PC 的成員,或者由PC成員邀請的、來自學術界自願參與審稿過程的審稿人。論文的評審過程通常需要大約三個月。
\item 當每位PC成員都完成審稿後,整個委員會將開會商議,通過審稿人的反饋決定接收一部分論文稿件,並拒收剩下的稿件。
\item 程序委員會會發出電子郵件通知所有的作者,告知他們論文是否被接收,並將審稿人對他們的論文提出的審稿評價附在電子郵件中。
\item 論文被接收的作者參加學術會議,並關於自己的論文做一個30分鐘左右的演講。學術會議結束後,所有的論文都將被收錄在在線的數字圖書館中。\footnote{根據研究領域不同,學術會議的舉辦方式略有不同。在一些領域中,論文會被PC分為演講展示(oral presentation)和海報展示(poster presentation)兩種形式。演講展示中,通常有5至30分鐘供作者在報告廳等場合公開真實自己的研究成果;海報展示時,所有參與的作者則將自己的研究成果展示在一張海報上,並以海報為基礎向觀眾展示自己的研究。通常演講展示會有更多聽眾和更大的影響力,在有演講展示和海報展示之分的學術會議中,PC也會把有限的演講展示機會分配給他們認為更有影響力的學術研究項目。——譯者註}
\end{enumerate}

通常,一個備受關註的\emph{頂級}學術會議的論文接收率在8\%到16\%之間,而\emph{第二級}的學術會議大概接收20\%到30\%的投稿。由於這些接收率相對較低,對於一篇學術論文來說,被拒收、修改並重新提交並不罕見——在論文最終被接收之前可能這一過程要重復多次,而這一過程可能會花費數年的時間。(在同一時間,一篇論文只能被提交到一個會議。)

\breakline

當Dawson說他想要提交一篇論文到2007年3月截止提交的頂級會議之後,他向我介紹了當時其他五個學生工作的方向,並讓我選擇自己感興趣的工作。我選擇了使用Klee 來尋找\emph{Linux驅動程序}中存在的新的程序缺陷。\emph{驅動程序}是指用於幫助操作系統完成其與外置設備(如鼠標或鍵盤)通信的軟件代碼。而\emph{Linux操作系統},和微軟Windows系統或者蘋果Mac OS系統類似,包含了成千上萬這樣的驅動程序,用以連接各種各樣的外置設備。對於傳統的調試方法而言,驅動程序中的軟件缺陷十分難以找到,甚至有可能是危險的,因為驅動程序的缺陷可能會導致操作系統死機甚至崩潰。

Dawson認為Klee可以在Linux驅動程序的成千上萬行代碼中,找到其他自動缺陷分析軟件(甚至人)從未找到過的軟件缺陷。我記得我當時考慮過,盡管在Linux驅動程序中找到缺陷寫在論文裏看起來很不錯,但我並不是很明白這樣的工作能不能算是真正的研究貢獻。按照我的理解,我要做的事情是用Klee去尋找程序缺陷—— 將一個已有的研究應用在實際問題上—— 而不是采用一種創新的方法來提高Klee的性能。此外,我並不明白,到三月份論文截稿時,我的工作和其他五個學生的工作如何能融合成一篇自洽的論文。盡管如此,我當時相信Dawson在頭腦中有一個高瞻遠矚的思路完成這篇論文。介於我剛剛加入研究項目,我並不想馬上開始對這些應該由教授決定的問題開始指手畫腳。任務已經擺在眼前,我要考慮的只是盡我所能,完成這個任務。

\breakline

我博士生涯的前四個月被我用於配置Klee以用它來分析上千行的Linux設備驅動程序,以期從中發現新的程序缺陷——這一過程並不順利。盡管看起來我的任務並不復雜,但我很快就被淹沒在了一些細節問題中,而這些細節對於讓Klee能夠分析Linux 驅動程序又是必不可少的。我經常會花幾個小時設置好Klee所需的復雜的實驗環境,以分析某一個驅動程序的程序缺陷;但這種工作又往往以Klee因為其自身的程序缺陷而崩潰告終,讓我的努力付諸東流。當我把這些Klee中存在的問題報告給Cristi時,他會竭盡全力解決,但由於Klee 本身極其復雜、環節眾多,在其中找到並解決程序缺陷絕非易事。我並不是想要專門指責Klee:任何以科研為目的開發的原型軟件都會或多或少有一些難以預見的缺陷。我的任務是用Klee去Linux驅動程序代碼中尋找缺陷,但諷刺的是,我整個第一個月的工作都變成了在Klee裏找缺陷。(Klee不能在自己的代碼裏自動找到缺陷,這實在是太遺憾了!)隨著時間的流逝,我對於手頭的工作越來越感到沮喪,我覺得自己被分配的任務就是單純的苦力,毫無知識含量——我的時間都用在讓Klee能正常運行上了。

這是我生命中第一次感受到被手頭的工作所淹沒的無望。以前,我的暑期實習都相對不難,而且就算一些學校的作業對我來說有些挑戰,作業中也總有一個需要找到的正確答案等著我。如果我課上有沒聽懂的內容,助教或者高年級的學生總可以幫我答疑解惑。就算在本科進行學術研究的時候,我也總能請輔導我的博士學長幫忙,因為我當時處理的問題相對比較簡單,而他通常知道問題的解決方法。對於本科的研究助理而言,對學術研究的投入以及期望也相對較低:科研只是我日常生活很小的一部分。如果我在某個科研問題上毫無頭緒,我可以選擇集中精力在課程作業上,或者幹脆和朋友們出去玩。本科畢業也和學術研究毫無關聯。然而,作為博士研究生,學術研究是我唯一的工作,除非我在學術研究上能有所成果,否則我將得不到博士學位。我難以把自己的情緒和每天的研究進展分離開來,而在那幾個月中,我的研究進展出奇得緩慢。

我現在步入了一個完全陌生的領域,所以我很難再向本科時候那樣向別人尋求幫助,因為問題的答案往往是不確定的。因為我是唯一試圖將Klee應用於驅動程序代碼的人,我的同事們並不能為我提供任何指導。Dawson偶爾會給我一些較高層面的策略性的建議,但就像所有已經獲得終身職位的教授一樣,他的角色並不是和他的學生們一起“在戰壕裏(第一線)參與戰鬥”。在做出研究成果的過程中,弄清楚所有復雜的細節是我們學生的任務—— 對我來說,這些細節就是如何找到別人從未發現過的、Linux 驅動程序中的軟件缺陷。教授們都喜歡重復這句老生常談:“如果有人曾經做過這樣的工作的話,那就不能叫做學術研究啦!”這是我第一次親身體會到了這句話的含義。

盡管我每天都覺得工作很無望,但我仍然不斷安慰自己:\emph{我只是剛剛開始在這裏工作,我應該保持耐心。}我不想在我的導師或者同事面前顯得軟弱無能,尤其因為我當時是Dawson組裏最年輕的學生。所以,我在超過100天的時間裏,每天修復Klee不斷產生的新問題,不斷遇到新的、更加棘手的問題,在我的“尋找Linux驅動程序缺陷之旅”中,拖著沈重的腳步前進。

當時,在我醒著的每時每刻,我不是在工作,就是在思考研究中的問題,抑或是在因為自己在研究中被艱深的技術問題所困擾而感到沮喪。與一般的朝九晚五的工作(比如我的暑期實習)所不同的是,以前每天晚上我可以把工作留在辦公室,坐在電視前放松自己;而現在的學術研究在情緒上和心理上都是無休無止的。我晚上幾乎沒法讓大腦停止思考問題,盡情放松休息——後來我發現幾乎所有的博士研究生都有類似的困擾。有時,因為我的研究任務繁重得難以想象,我甚至會因為壓力而失眠。想要在研究中休息一段時間也幾乎不可能,因為在3月份的論文提交截止日期之前,還有很多工作需要完成。

在做這些苦力時,我曾經試圖想出一些半自動的方法,可以讓每天的研磨不那麼辛苦。我和Dawson交流過一些初步的想法,但我們最終的結論是,如果我們想讓Klee 能在Linux驅動程序裏找到缺陷,這種耗時間的研磨是無法避免的。接下來,在論文提交前的幾個月裏,我只能繼續辛苦工作。

理性而言,我十分理解在科學和工程領域中,通過實驗式的研究方法獲得研究成果,通常並不光鮮、甚至包含很多辛苦的工作。而博士研究生,尤其是第一、第二年的博士生,通常不得不在這一過程中參與最辛苦、最勞力的工作——這是他們的資金來源所決定的。通常,在一個研究組中,教授和高年級博士生制訂高瞻遠矚的研究計劃,並將任務布置給低年級的學生,由他們去研磨所有的細節,並讓項目實際運轉起來。第一、第二年的學生通常很難影響研究組項目的大方向。盡管我完全接受自己在這種“等級制度”下較低的地位,但感性上,我仍然深受打擊,因為手頭的工作實在是太他媽難了\footnote{原文此處為“so damn hard”,此處為保持原文語氣,進行了直譯。},而且毫無成就感可言。

\breakline

經過兩個月的研磨之後,我收獲了一些小小的成功。我成功地讓Klee能在一些較小的驅動程序上良好工作,並開始找到了幾個程序缺陷。為了確定這些缺陷是否真實存在(而非因為Klee的局限而導致的誤報),我向開發這些驅動程序Linux的程序開發者發了幾封電子郵件,說明了這些可能存在缺陷的地方。幾位開發者回復了我郵件,確認了我的確是在他們的代碼中找到了真正的缺陷。當我收到這些確認的郵件時十分激動,因為這是我第一次受到外界的認可,盡管它們只能算很小的鼓勵。盡管我沒有做出什麼革新性的研究,我依然收獲了一些成就感,因為這些缺陷在沒有Klee的幫助下可能非常難以找到。

在這些小的驅動程序裏的缺陷得到確認之後,我的士氣有所恢復,所以我開始著手分析更大、更復雜的驅動程序。然而,接下來的幾周中不斷浮現的技術問題逐漸變得難以承受,並幾乎把我推到了崩潰的邊緣。這裏簡單總結一下我當時遇到的問題:Klee 實際上只能在不超過大概3000行,由C語言寫成的代碼中找到缺陷。最小的Linux驅動程序代碼大概有100行,所以Klee分析它們遊刃有余。而稍微大一些的驅動程序就有大約1000 行代碼,而這些代碼還與大約一萬行到兩萬行的Linux操作系統代碼密不可分。這就導致問題一下遠遠超過Klee 可以分析解決的範圍,因為Klee 並不能把這
1000行代碼和其他部分“斬斷聯系”而單獨分析。我試過很多方法減少這些外部的連接(術語稱為\emph{依賴}),但是這樣做意味著對每一個驅動程序,我將需要花費幾天的時間去完成十分復雜的工作,而最終這些工作也只對這一個驅動程序有效。

我和Dawson面談了一次,向他說明了我面臨的問題的復雜程度,也表達了自己對此的消極情緒。對我來說,在每一個新的驅動程序上都要花費幾天的時間才能讓Klee 可以運行,這簡直是荒唐至極的工作。這種工作不僅讓我疲憊不堪,它本身甚至不能算是學術研究!在我們的論文裏,我應該怎麼描述我的工作——我花了大概1000小時的苦工,就為了讓Klee 能分析一堆驅動程序,除此之外我並沒有其他可說的了?這根本不是學術貢獻,反而聽起來太愚蠢了。另一方面,我開始著急,因為距離論文提交截止只有五周了,而Dawson似乎還沒有提過我們組的研究論文應該從什麼思路著手寫作。通常一篇優秀的學術論文需要至少四周時間才能完成,而且我們的情況可能更加復雜,因為我們有六個學生各自負責項目的一部分,我們需要把這些部分融合在一起。

我和Dawson面談過後幾天,他提出了一個改進的思路,讓Klee能夠克服我所面臨的依賴問題。他提出的這個觀點叫做\emph{約束下運行}(UnderCons-\\trained execution,簡稱UC),而這種方法也許能讓Klee 將Linux驅動程序和外部的一兩萬行代碼剝離開來,並單獨在驅動程序代碼中進行分析。他馬上就開始和一個高年級的學生開始將UC技術融合在Klee 中;他們把改進後的版本稱為\emph{Klee-UC}。盡管我當時筋疲力竭、接近崩潰,但我仍然非常欣慰看到自己的痛苦掙紮至少激發了Dawson的靈感,想到了這個有朝一日可能成為學術貢獻的新思路。

接下來,Dawson和那名學生在Klee-UC上花了幾周的時間。與此同時,他們讓我繼續用原來的方法,手工在Linux驅動程序中尋找缺陷。他們打算通過用Klee-UC試圖找到我用普通Klee 找到的缺陷,來證明其有效性。他們想在論文裏闡述的觀點是,與其讓一個博士生(就是我!)為了找到每個缺陷而在繁雜的工作上花費幾天的時間,不如用Klee-UC來在幾分鐘之內找到這些缺陷,而且不需要準備任何特定的實驗環境。

在近乎瘋狂地又研磨了幾周之後,我終於讓原版的Klee成功分析了937個Linux驅動程序,並找到了55個新的軟件缺陷(其中32個收到了開發者的確認)。接著,我還得把剛剛成型的Klee-UC配置好,並準備分析這937 個驅動程序。這比之前的配置還要麻煩,因為在我準備用Klee-UC分析驅動程序的同時,Dawson和那個學生仍然在(編程)實現它。謝天謝地,最終Klee-UC的確找到了我找到的缺陷中的大部分,這樣我們做了一些學術貢獻,也有成果可以在論文裏展示了。

但是這裏有個巨大的問題。等到我們得到那些不錯的結果時,到論文截止日期已經只剩下\emph{三天}了,而我們都完全沒有著手開始寫論文。在這麼短的時間裏,撰寫、編輯、完善一篇文章,使得它能被頂級計算機科學學術會議錄用是完全不可能的。但我們還是努力這麼做了。在提交截止前的72小時裏,Dawson和我們五個學生(其中一位現在已經退學了)兩個通宵住在辦公室,以期完成實驗和論文。我們所有學生心裏都知道這篇論文不可能被接收,但我們還是在Dawson的帶領下,順從地向目標前進。

最終,我們提交了一篇令人尷尬的文章,裏面充滿了錯誤拼寫、不通順的句子片段、沒有解釋的圖表,甚至連結論段都沒有。整個是一團亂麻。在那一刻,我難以想象,如果我的研究工作一直是這樣毫無章法,我到底要如何從博士項目中畢業。三個月後,不出所料,我們收到的審稿意見極其消極,甚至包括這樣的嚴厲譴責:“程序委員會(PC)認為這篇論文簡直太過草率,根本不可能被錄用;請不要提交這種甚至無法審閱的論文。”

\breakline

這次慘痛經歷之後,我申請了去谷歌實習,因為我急需換一換環境。那份實習和我的研究興趣毫不相關,但我並不在意。當時我只想逃離斯坦福幾個月。

這時是2007年4月份了,距離6月份我的實習開始,還有十周的時間。我不知道我能做些什麼,但我想逃離我之前四個月折騰Klee和Linux驅動程序的經歷,逃得越遠越好。我根本不關心我們是不是會修改並且重新提交那篇論文(最後沒有);我只想逃離那段夢魘。但因為我已經積累了近千小時的使用Klee的經驗,而這也是Dawson 當時唯一關心的項目,我想這可能是一個開展新課題不錯的切入點。於是,我找Dawson 談了談,和他交流了一些“不尋常”的使用Klee的方法,而不僅僅是局限於尋找軟件缺陷。

然而我很快就意識到,我完全沒必要把自己局限在Klee上,因為我的經費來自NDSEG 獎學金,而非Dawson的研究經費。而Dawson其他學生的境況則於我不同,他們除了繼續在Klee 項目中工作之外別無選擇,因為他們都是靠Klee的研究經費支持的。所以我決定仍然讓Dawson 作為我的導師,但離開Klee項目組去從頭開始開展我自己的研究項目。

為什麼我不早點“單飛”呢?因為盡管我的獎學金理論上給予了我研究方向的自由,但我知道我仍然需要一位導師的支持才能最終畢業。當時,Dawson顯然想讓他所有的新生都把精力投入到Klee上,所以我就花了四個月的時間在Klee上研磨,塑造了一個“好士兵”的形象,而不是一開始就傲慢地要求做自己的項目。而且,就算我選擇了別的導師,我一開始還是會需要花時間在他們的項目上,來證明自己的能力。這種“交學費”的工作根本無法避免。

接下來的十周裏,我在一個完全真空的環境思考自己的研究思路,完全沒有與人交流。因為我一開始對研究組留下了如此消極的印象,我現在只想自己一個人清靜地思考。Dawson 對我的消失並沒什麼意見,因為他並沒有在用自己的研究經費支持我。

我把自己完全與世隔絕,心理上基本處於崩潰狀態,但我仍然試著每天能有所進步。我嘗試著每天閱讀幾篇計算機科學的學術論文,並做好筆記,以幫助我激發自己的靈感。但因為沒有良好的引導和研究背景,我最後浪費了很多時間,有時讀完了論文卻一無所獲。我還曾經騎著自行車在校園裏漫無目的地遊蕩,試圖在過程中思考自己的研究思路但常常無功而返。最終,我把時間都浪費在了拖延上,可能甚於我這輩子至今為止的任何一段時光:我看了很多電視節目,睡了很多覺,花了無數個鐘頭在網上蹉跎。與我的朝九晚五的朋友們不同的是,我沒有老板每天監督我的工作,所以我可以無拘無束,讓我的大腦自由思考。

盡管我對研究方向的頭腦風暴大部分都毫無目標,但我的想法逐漸地開始向一個問題聚攏:\emph{我們如何才能經驗性地度量軟件的質量呢?}這是在我開始讀博之前,受到工程性實習中遇到的低質量的軟件而啟發,所想到過的一個感興趣的研究方向。然而,在真空中空想研究思路的問題所在,是我當時沒有足夠所需的經驗來把這些思路轉化為現實的項目。因為我還沒有準備好,顯然這種完全的思想自由在當時更像是一種詛咒。

盡管我對創造新方法來度量軟件質量非常感興趣,我當時也明白這只能是一個模糊的夢,沒有足夠的形式化的研究方法支持,是不可能被學術界所接受的。如果我當時自己閉門造車研究這個項目,我的下場只能是成為一個胡說八道卻又假裝內行的人。我永遠不可能有機會把這些想法發表在頂級、甚至次一級的學術會議中,而發表不了論文就意味著不能畢業。那時,我已經不再抱有虛無縹緲的夢想,想要成為一名終身教授了:我只想能找到個辦法畢業。

在那整整十周的與世隔絕中,我幾乎沒跟任何人說話——甚至包括朋友和家人。找人抱怨也毫無意義,因為沒人能明白我當時經歷的一切。我的不讀博的朋友們以為我就是“在學校上上課”。而少數幾個我在系裏新認識的朋友也和我一樣,正在因為他們博士第一年裏的掙紮感到抑郁——這些掙紮的主要部分,多半就是不得不直面充滿挑戰、不知從何入手的研究問題帶來的震驚,以及無法影響實驗室研究方向、只能隨波逐流的無奈。我們這一群年輕的計算機科學家來到這裏,自願地投身於復雜而看起來毫無意義的工作中,所收獲的卻只是我們在公司工作的朋友們四分之一左右的工資。這種現實簡直可悲到可笑的地步。但我覺得大家湊在一起互倒苦水也毫無裨益,所以我選擇了閉嘴。我選擇了不來計算機系的大樓裏工作,因為我害怕碰到同事、同學們。我害怕他們會不可避免地問我我現在研究的課題是什麼,而我並沒有什麼可說的東西。藏身於圖書館或者咖啡店讓我感覺更為安逸。

反觀當初,這麼早脫離實驗室單幹是一個非常糟糕的選擇。和想象中一位孤獨的學者坐在路邊,一邊喝著拿鐵咖啡一邊在一張筆記紙上塗塗畫畫這種浪漫化的場景不同的是,真正的學術研究從來就不是在“真空”中完成的。為了創新,一個人需要有堅實的知識、歷史、甚至物質基礎(比如實驗室設備)。在那幾周中,我應該追尋的更明智的路線是找更多機會和Dawson談話,並積極尋求和其他教授以及高年級學生的合作。但當時的我太過崩潰和沮喪,不滿於“等級分明”的基於研究組的學術風格——把新來的博士研究生放在最繁雜枯燥的工作中——所以產生了逆反而選擇了單飛。

\breakline

在我十周的與世隔絕接近尾聲,即將去谷歌暑期實習之前,我給Dawson發了個電子郵件,簡單說明了一下我最近讀到的一篇讓我思考了很多的科技博客。那篇博客讓我想到了一個想法,那就是通過開發者在程序代碼存在的整段時間裏編輯代碼的過程,來衡量一個軟件項目的質量。令我驚喜的是,Dawson很快給我發了一個簡短的回復,說他對這種衡量軟件質量的技術也很感興趣,尤其是如何用這種技術來幫助Klee這樣的自動缺陷檢測工具。

當知道Dawson對我感興趣的領域也感興趣之後,我再次燃起了希望,因為他也許能幫助我把這個想法變得更接近現實。我很快的為這個新想到的\emph{經驗化軟件質量度量}的項目記下了一些筆記,並打算從暑期實習回來後開始研究這個課題。這樣,在四個月與Klee 的痛苦研磨,以及十周漫無目的的找尋方向之後,我博士生涯的第一年畫上了一個還算樂觀的休止符。

%%% Year Two: Inception
\mychapter{第二年:整裝啟程}

在谷歌的暑期實習對我而言,是一段脫離學術研究的輕松時光。我的工作壓力並不大,所以我經常有機會和其他實習生一起閑聊、交朋友。臨近夏天結束時,我已經從第一年的痛苦和折磨中痊愈,做好了重新開始博士生活的準備。

在夏季最後幾天,我給Dawson寫了一封電子郵件,重申了我對追求自己研究興趣的渴望,但同時強調了從研究中發表論文對我而言的重要性:“我從這個暑期以及之前的暑期實習經驗,發現自己很難集中精力完成一個博士科研項目,除非我對這個項目有強烈的歸屬感和科研的熱情;所以我非常希望能在令我感到激動的研究方向,和學術界中的教授們廣泛認為具有‘研究價值’的問題中,找到一個交集,並為之奮鬥。”

我準備繼續和Dawson一起在經驗化軟件質量度量的課題上做一番研究,因為我們已經在我第一年結束的時候討論過這一問題。然而,我當時預感這個項目可能會風險較大,因為這不是Dawson主要的研究興趣,也不是他主要的專業知識的所在;對他而言,Klee仍然是最高優先級的工作。於是,我想到自己可以去尋找另一個課題,通過同時在兩個課題上做研究(並寄希望於至少一個能成功),來沖淡風險。我將在這章稍後介紹我和Dawson合作的主要研究項目,但我想先說說我的另一個項目。

\breakline

就在2007年9月,我準備開始我博士生涯的第二年之前,我去波士頓度假了一周,看望我大學時的朋友。因為我當時離得不遠,所以我給幾位我本科時就認識的MIT教授發了郵件,希望可以向他們尋求一些指導。他們和我見面的時候,大概都談到了同樣的事情:\emph{要主動去找教授談話,盡量去找一個共同的興趣點作為研究課題;不論如何,把自己關起來閉門造車是萬萬不可取的。}這句簡單的建議在我博士接下來的五年裏反復得到印證,也最終指引我順利完成了博士學習。

還在波士頓時,我就馬上將這個建議印在心裏並付諸行動。我給MIT計算機系一位叫Rob 的教授發了一封\emph{冷郵件}(向從未通信過的人冒昧發送的郵件),並禮貌地請求和他面談一次。在這封郵件中,我簡單地說明了自己是MIT畢業不久的學生,現在在斯坦福攻讀博士,並希望能開發出一些讓計算機程序開發者能提高工作效率的工具。因為我知道Rob也對這一研究領域有興趣,我希望他能回復我的郵件,而不是把它標記為垃圾郵件。Rob非常慷慨地在他的辦公室和我面談了一個小時,期間我向他介紹了幾個研究項目的想法,希望能從他那裏得到反饋。他似乎對這些想法很感興趣,這讓我信心倍增,因為我的想法或多或少得到了一位研究領域內的教授的認可。可惜,因為我不再是MIT的學生,我沒有機會和Rob一起開展研究。在我們面談的最後,Rob推薦我和斯坦福計算機系一位名叫Scott的教授談一談,看看我有沒有機會成功向他推銷這些觀點。

回到斯坦福後,我給Scott發了一封冷郵件,請他給我一個機會面談。去面談時,我針對想和他溝通的三個具體的研究思路準備了一些筆記,大致格式如下:

\begin{enumerate}
\item 問題的定義是什麼?
\item 我準備用什麼方法解決?
\item 用什麼樣的實驗,才能有說服力地證明我的方法具有良好的效果?
\end{enumerate}

Rob的一位博士學生,也是我的朋友,Greg,教會了我其中第三點——用實驗指導思考——在提出研究項目思路中的重要性。教授們很樂意讓自己的名字出現在發表的論文上,而計算機科學領域的學術論文通常需要強有力的實驗支持才能有機會發表。因此,在提出項目的想法時就考慮到實驗設計是十分重要的。

盡管我的介紹都沒能贏得Scott的全力支持,但他仍然希望可以和我一起開展一個和我大致研究興趣相關的研究課題。當時,他是一位剛剛來到斯坦福三年的助理教授(尚未獲得終身職位),所以他非常希望能多多發表論文,早日拿到終身教職。因為我自己有獎學金,Scott並不需要用自己的經費支持我,所以對他來說也沒有什麼顧慮。

\breakline

Scott所精通的,是一個稱為人機交互(Human-Computer Interaction,簡稱HCI)的、較為貼近實用的\emph{子領域}。和其它子領域不同的是,HCI的研究方法以人的需要為核心。一般來說,HCI的研究項目都是這樣完成的:

\begin{enumerate}
\item 觀察人,並弄清楚他們面臨的真正問題是什麼。
\item 設計一些創新性的工具,以幫助緩解這些問題帶來的影響。
\item 用實驗來評估這些工具,以檢驗它們是否真的可以幫助人們解決問題。
\end{enumerate}

因為我希望創造能提高程序開發者工作效率的工具,Scott就先建議我去開發者們真正的工作場所,觀察他們的工作,並借此發現他們真正面臨的問題。具體來說,Scott非常好奇為什麼現代的軟件開發者使用多種編程語言,並且嚴重依賴於在網上搜索、並剪貼代碼片段來完成開發。之前幾十年的高效開發工具的研究中,都假設了開發者只用一種語言進行開發,並且認為他們所處的環境中開發者的水平大致相同,而這些假設現在顯然都早就過時了。通過觀察現代程序開發者的日常活動,也許我就能設計出更貼合他們需求的新工具。

接到Scott給我的定下的大致目標之後,我就馬上著手開始尋找可以讓我觀察的、專業的軟件開發者。因為我剛剛在谷歌實習,我試過在那裏找到目標,所以我發郵件給了我實習時的經理,他也同意將我的郵件轉發給其他同事。我很快就收到了幾封婉拒的郵件,因為沒人想因為一個非雇員看著他們工作,而惹上知識產權的麻煩。接著我又給一些在斯坦福附近創業的朋友們發了郵件,但很快就發現他們和大公司的程序員一樣,也收到種種限制。遺憾的是,他們更難以答應我的請求,因為為了避免競爭者抄襲,他們需要對自己的創業項目保密。而且,他們遠比大公司的開發者們更加繁忙,所以也不太希望浪費自己的時間,去幫隨便一個研究生完成這種對他們沒什麼幫助的科研項目。

我當時最後的希望是去謀智(Mozilla)觀察軟件開發者工作,因為謀智是一個非營利性的軟件開發組織,他們的產品包括著名的火狐(Firefox)瀏覽器。因為謀智的軟件項目都是開放源代碼\footnote{開放源代碼(Open Source),簡稱“開源”,是一種在自由軟件開發者中非常流行的形式。開發者自願將自己軟件的源代碼向其他開發者公開,而在一定的通用協議下(通常被稱為“證書”),其他開發者可以基於這些軟件代碼進行改動和再次開發。開放源代碼有助於鼓勵開發者分享自己的成果,避免重復勞動。當然,盡管軟件源代碼是開放、可供人任意下載的,有些開源軟件也是可以盈利的。——譯者註}的,所以我以為他們應該不怕會有一個局外人來觀察他們的程序員工作。我沒有試著發冷郵件(其實我都不知道應該要給誰發!),我決定直接開車去謀智的總部,直接登門造訪。我冒然和我看到的第一個人打了招呼,並向他介紹了自己。他很熱心地告訴了我兩位謀智高層領導的電子郵箱,並說明這兩位可能會對這樣的研究合作感興趣。我當天就給兩個人發了冷郵件,令我感到驚喜的是,其中一位回復稱對我的研究很感興趣。不幸的是,那也是我收到的來自他的最後一封郵件;當我請求繼續開展研究時,他就再也沒有回復我了。

現在反觀當時,對於自己試圖在工作場所觀察別人工作的嘗試以失敗告終,我並不感到奇怪。畢竟,我並不能為那些被我觀察的專業程序員帶來任何好處;相反,我只會幹擾他們的日常工作。幸運的是,幾年後,我有機會觀察了一組(非專業)程序員—— 為科學研究而編寫程序的研究生們——的工作,他們並不排斥我可能帶來的小小打擾,反而非常樂於和我交流他們工作環境中的問題。這些訪談的經歷,最終激發了我的靈感,並直接為我的畢業論文做出了很大貢獻。

\breakline

在尋找專業程序員的路線上碰了南墻,我就決定回頭,開始在斯坦福內部尋找這樣的機會了。當我看到一張海報,宣傳系裏即將舉辦一年一度的編程競賽時,我馬上給主辦者發了一封冷郵件。我向他說明了自己想要觀察學生們比賽過程的想法,他欣然接受了。

盡管學生的編程競賽對說明現實中的軟件開發環境而言毫無代表性,但總是聊勝於無。Scott的一個比我高一年的博士學生,Joel,也希望來參加這種觀察。Joel和我把整個周六上午四個小時都花費在了觀察幾個學生參加比賽這件事情上。我們看到的結果很無聊,最終也沒能從我們的筆記中整理出太多有用的東西。然而,這個經歷促使我們產生了一個想法,那就是進行條件更加可控的\emph{實驗室研究}。

這時,Scott已經決定讓Joel和我合作進行研究,一起開展一個實驗室研究,而不是分散精力在不同的項目上。因為Joel和我的研究興趣相似,Scott也可以通過讓我們合作,減輕自己管理學生的壓力。Joel主要負責了實驗室研究的設計,我也同時樂得清閑,因為我同時還在和Dawson做另一個課題的研究。

Joel、Scott和我設計了一個時長兩個半小時的實驗室研究,研究中,一名斯坦福的學生將會被要求從頭開始,編程完成一個簡單的網絡聊天應用。他們可以使用任何他們想用的資源,最主要的就包括在網上搜索現成的代碼,或者已有的教程。我們找到了20名學生參加實驗,其中大部分來自Scott當時在講授的\emph{人機交互導論}課程的學生中。

在接下來的幾周裏,Joel和我在系裏地下室的機房坐了50個小時(20個學生,每人兩個半小時),觀察參與研究的學生,並同時將學生使用的電腦屏幕錄制下來。一開始,觀察學生如何工作非常引人入勝,但好景不長,我們很快就發現這個工作變得冗長繁雜,而且總能看到學生們一遍又一遍地做同樣的操作、犯同樣的錯誤。之後,我們又大概花了50個小時重放我們錄下的視頻,並在視頻中標記了一些關鍵事件發生的時間。最終,我們分析自己的筆記,並回放標記好的關鍵事件,得到了一些關於“當這些學生面對一個簡單、但實際工作中可能遇到的任務時,會遇到哪些問題”的深入了解。

我們把實驗室研究的成果寫成了論文,並將論文提交到了一個頂級的HCI學術會議。三個月後,我們發現論文被錄用了,審稿意見非常令人振奮,而論文甚至還被提名了\emph{最佳論文獎}。在這篇論文裏,Joel 是\emph{第一作者},而我是第二作者。幾乎所有計算機科學領域的論文都是由多位作者合作完成的,而作者的順序也是至關重要的。名字出現在首位的作者通常是研究項目的負責人(比如Joel),他通常在項目中完成的部分遠多於其他位置靠後的作者,所以也最應當受到承認。其他的作者通常都是項目中的助理—— 通常是年輕的學生(比如我)或者遠程的合作者——這些人也做出了足夠多的貢獻,可以被列為作者之一。博士生們通常把自己的導師(比如Scott)列在最後,因為導師通常會在研究想法形成、項目規劃和論文寫作方面提供幫助。

因為我不是論文的第一作者,這篇論文也就對我的畢業論文沒有什麼幫助;然而,這個經歷教會了我許多關於如何開展研究、以及如何寫作學術論文的知識。更重要的是,看到這個研究項目最終發表了一篇頂級會議的論文,我感到很有成就感——這和我博士第一年被拒收的Klee的論文形成了鮮明的對比。

Joel後來繼續沿著這條研究路線走了下去,把我們的論文轉化成了他博士論文的第一部分。在接下來的幾年中,針對在一開始的實驗室研究中發現的程序開發者們面臨的困難,他又創造了幾個能幫助克服這些困難的工具,並發表了論文描述這些工具的原理。同時,Scott也沒有積極招我入麾下,我也就沒有考慮過更換導師的事。當時看來,似乎我和Joel的興趣太過接近了,而且Joel已經在他的子領域小有成就了。所以,我繼續將研究重點放在和Dawson 一起研究的課題上,畢竟他仍然是我的導師。然而在這一邊,研究進展就沒這麼順利了。

\breakline

讀者可以回憶一下,在我博士第一年接近尾聲時,我開始在一個被稱為\emph{經驗化軟件質量衡量}的子領域尋找研究靈感——具體來說,就是通過軟件開發的歷史記錄,來衡量軟件的質量。正好Dawson也對這一課題感興趣,所以在我博士第二年裏,我們就繼續在這一問題上開展研究。他的主要興趣還是在於發明像Klee這樣的自動缺陷檢測工具,但他也在軟件質量方面有一些興趣。對他而言,讓他感興趣的研究課題包括:

\begin{itemize}
\item 如果給定一個大型軟件項目,其中包含上千萬行代碼,如何區分哪些部分對項目而言更加重要,而哪些部分不那麼重要?
\item 有哪些因素影響了一些部分更容易出現軟件缺陷?比如,是最近剛剛被新手修改過的程序代碼更容易包含更多缺陷呢,還是短時間內被很多人修改過的程序更容易出問題?
\item 如果一個自動查找缺陷的工具找到了1000個可能的問題,哪些會是其中更重要的呢?程序開發者可能並沒有時間和經歷處理全部1000個缺陷,所以他們必須根據一個重要度的評估,來優先完成重要的部分。
\end{itemize}

我通過分析和Linux系統核心這個軟件項目相關的數據集,開始調查這些問題。我選擇了Linux作為研究對象,因為它是當時最大且影響最廣泛的開源軟件項目,有數以萬計的程序開發者在二十年間貢獻了幾千萬行的程序代碼。Linux完整的\emph{版本控制歷史}能在網上免費得到,所以這成為了我主要的數據來源。項目的版本控制歷史記錄了在項目的整個生存周期中,所有代碼文件中出現過的所有改動,包括改動出現的時間,更重要的是,改動是誰做出的。為了將這些項目歷史和軟件缺陷對應起來,我從Dawson的軟件缺陷檢測公司拿到了一份數據集,其中包含了公司的一個產品在Linux中發現的2000個缺陷。當然,這不會是Linux裏所有的缺陷,但這是我能接觸到的數據集中,唯一能為我提供了足夠的信息,來研究我們的學術問題的了。

那時,我每天的工作流程就是寫程序來從Linux的版本控制歷史以及那2000個缺陷報告中提取、和清洗數據\footnote{數據清洗(Data cleansing)在科研項目中通常被廣泛應用,尤其是當數據來自實際的項目,或數據產生、采集的過程沒有嚴格監控時。數據清洗通常包括丟棄數據中對研究沒有幫助的部分,或者丟棄因格式、內容、完整性等問題不能被用於研究的“臟數據”。此處作者應該指前者,即將和研究相關的數據分離出來的過程。——譯者註},並將它們轉化成便於分析的格式,並進行分析。為了能對這些數據有更深層次的認識,我自學了量化數據分析、統計學和一些數據可視化的技巧。在研究過程中,我細致地將自己的實驗進展記錄在了一本\emph{研究筆記}中,用來記錄哪些嘗試並沒有給我想要的結果。大概每周,我都會和Dawson面談一次,匯報自己的研究進展。我們的會面通常是我想他展示我從分析中得到的圖表或者表格,而他則會提供一些較高層次的評價,比如:“嘿,這張圖的這個部分看起來不太對勁。這是為什麼呢?試試這樣分割你的數據,並做一些更深入的分析。”很多年後,我意識到這其實是一種對於很多學術領域中的\emph{計算研究者們}而言,再平常不過的工作流程;而在我的博士論文中,我也創造了一些工具,來消除這一過程中的一些效率不高的環節。不過那時我還沒有這麼長遠的計劃;我所想的只是做出一些有趣的發現,並把他們寫成論文發表。

\breakline

Dawson和我在將我們的成果發表的道路上並非一帆風順。在那一年裏,我們提交了兩篇論文,最後都以被拒收場。這個研究項目後來又經過了一年,才轉化成為一篇長度較短的、第二級學術會議中的論文得以發表,論文影響力微乎其微,也沒有“作數”成為我博士論文的一部分。不過到那時,我已經不太計較這些了,因為我已經在開展其他研究項目了。

我們的論文在發表上遇到了這麼多障礙,究其原因是因為我們不是經驗化軟件質量衡量(有時也被稱為\emph{經驗化軟件工程})這一子領域的“圈內人”,而我們的論文恰恰屬於這一子領域。在我們步入這個方向之前,已經有幾十所大學的研究小組在進行類似的研究了。Dawson和我在這種競爭中完全處於劣勢,因為我們的競爭對手包括很多專於此道的教授和科學家,而他們手下通常有很多博士生為他們產生研究成果。這些人對發表論文如饑似渴,因為他們中很多人仍然是年輕的教授,渴望借此得到終身職位。他們也更擅長於一些能讓論文更容易被接收發表的技巧,包括統計方法、相關文獻的引用,以及“營銷技巧”等。最重要的是,他們中很多人在相關的學術會議中擔當了外部審稿人或者程序委員會成員等職務,所以他們很清楚如果一篇論文想要被這個領域的學術會議所接收發表,應該需要怎麼撰寫。

讀者可以回憶一下,我介紹過每篇論文都會由三到五位自願參與的領域內專家——通常都是領域內的教授或者科研工作者——進行\emph{同行評審}\footnote{同行評審(Peer-review)是一種廣泛用於計算機科學學術會議的評審模式,其中參與審稿的審稿人可能是領域內的教授、專家、研究者或在讀博士生(本科生和碩士生較為罕見),這些人可能也是學術會議中的投稿人(審稿人不能審閱和自己有利益相關性的——比如自己撰寫的—— 稿件)。“同行(Peer)”一詞即用於強調這種審稿人和投稿人之間的平等性。與此不同的是,很多學術期刊采用專業審稿人審稿的制度,審稿人往往是領域內名至實歸的教授或者科研工作者,一般認為他們在領域內的經驗、學識等比一般投稿作者都略高一籌。——譯者註}來決定是否可以發表。如果審稿人認為文章值得發表,文章就會被接收並發表;否則,作者就需要修改論文,之後再重新投稿。同行評審的目的在於保證所有被發表的論文都有一定的質量保障,這一保障的水平則由學術圈來決定。這種檢閱的機制是必要的,因為總要有一些公認的標準,來過濾掉一些不靠譜的言論。然而,同行評審這種制度本身存在其缺陷,盡管它努力使得評審結果更加客觀,審稿人也仍然是人,還是不可避免地有自己的品味和偏見。

因為學術會議通常只接受不超過百分之二十的論文,如果一篇論文給審稿人以不好的第一印象,可能就會被拒收。Dawson和我都不是經驗化軟件度量領域的專家,所以我們並不能以一種審稿人所期待的方式,去“推銷”自己的論文。所以,我們經常因為一些負面的評審意見受到打擊,比如:“結果,我發現我就是對作者展示的結果不太信任。我的疑惑來自於兩個方面:一方面我不相信他們對度量方法的認知是否合理;另一方面他們也沒有使用一些有效的統計方法驗證自己的成果。”在發表學術論文這個殘酷的世界中,僅僅對一個課題感興趣遠遠不足以讓自己的論文得以成功發表;想要自己的論文可以發表,就得非常清楚作為論文審稿人的、自己領域中年長的同事們的偏好。簡而言之,我們的數據集沒有別人的好,技巧沒有別人的細致,我們的成果和展示方式也和這一領域中有經驗的研究者所期待的形式似乎格格不入。

相比而言,我和Scott、Joel發表的論文就更加成功,因為Scott是HCI領域的專家,並且在這一領域(尤其是我們發表文章的學術會議中)已經發表和評審了很多論文。當然,作為內行人並不能讓審稿人在評審我們的論文時,對標準有任何的放松,因為那就太不公平了。但是,Scott仍然可以利用自己的經驗,讓我們的項目從動機到成果,都以一種審稿人所期待的形式展現出來,從而論文被接受的概率也大大上升。

\breakline

在我博士第二年接近尾聲時(2008年6月),我因為仍然缺乏有說服力的成果,又看到在經驗化軟件度量領域的論文如雨後春筍般地冒出來,開始變得越來越沮喪。我仍然沒能將自己的發現發表成論文,也意識到自己和圈內的老手根本沒法競爭:因為Dawson和我都不知道審稿人期待看到的是什麼,我開始明白繼續在這個領域試圖發表論文,只能是一路逆水行舟。又因為發表論文是畢業的先決條件,我必須盡快找到一個新的項目,否則我就不能拿到自己的博士學位了。在我準備在斯坦福開始自己博士生涯的第三年時,我非常絕望,以致不願放過任何一棵能為我帶來論文發表的救命稻草。也就是那時,我選擇了回到博士第一年曾經讓我痛苦不堪的項目——Klee——當中去。

%%% Year Three: Relapse
\mychapter{第三年:噩夢未止}

在2008年中,我的博士第三年拉開帷幕時,我作為項目組最年輕的學生,重新回到了Klee 項目中。當時,項目裏只剩下兩個人了——兩位Klee的創造者,Dawson和Cristi (Cristi 這是已經是他最高年級的博士生了)。Klee項目組的其他成員都已經離開了這個項目。

對於回到項目中來,我的心情很復雜。一方面,我第一年不堪回首的經歷讓我對項目本身和項目組的工作狀態都感到抵觸。另一方面,Cristi和Dawson都對這個項目充滿激情,他們也都希望發表更多的跟進論文。因為他們在軟件缺陷檢測這一領域已經身經百戰,我覺得和他們合作似乎更有機會能夠發表論文。我的另一個選擇,則是繼續在我第二年開展的經驗化軟件度量項目上進行研究。雖然我仍然對那個項目很感興趣,我清楚地知道對於我和Dawson兩個外行人來說,在那一領域發表論文並非易事。因為我的目標是發表論文,取得博士學位,我決定收起自己的自尊和自大,並重新投身於Klee項目中。我給Dawson發了一封郵件,說到:“我大體的計劃就是與你和Cristi合作,盡量用Klee做出一些腳踏實地的研究,爭取內能投稿幾篇論文。我認為對我來說,為學術做出貢獻以及每天做出一點令人滿意的成果的最佳選擇,就是盡可能利用Klee,並和你們的研究興趣以及目標保持一致。”

\breakline

Cristi和Dawson希望我能實驗一種被稱為\emph{交叉檢測}的、運行Klee的新方法,這種方法能讓Klee找到同一個軟件的不同版本之間的差異。在接下來的四個月中(2008年7月到10 月),我每天的研磨和我第一年用Klee尋找Linux驅動程序缺陷的任務別無二致,只不過我現在學會了更好地勞逸結合,讓自己不至於過度勞累。但就像我第一年的情況一樣,我在用Klee做很多繁重的苦工,而實質上並沒有用什麼方法改進Klee,以讓它能找到新的軟件缺陷。我每天的工作就是配置好一系列Klee的運行環境,用10個小時運行Klee讓它完成對軟件的一系列檢測,第二天早上再來收集、分析、可視化、並弄明白結果的含義,對Klee 的設置進行適當的調整,然後再開始新的一輪10小時的實驗。

就像很多復雜的軟件工具一樣,Klee有很多可以設置的選項。而因為它是一個由很多學生合作完成的用於學術研究的原型系統,它的選項中有很多都沒有清晰的文檔說明其功能。這就導致我因為誤解了一些選項間復雜的相互作用,進而浪費了很多時間在調整這些選項上面。當時我的研究筆記上隨處可見這樣的臟話:“媽的,我覺得我的問題是沒發現Klee 需要某些調用參數(比如-emit-all-errors),而目標程序也需要一些參數(比如-sym-args),如果把這些搞混了,你也不知道會發生什麼,因為Klee在運行目標程序時采用的參數列表和你想象的不一樣!”

在那幾個月間,Cristi和Dawson時不時地提起他們想提交一篇關於Klee交叉檢測的論文,所以我對於這個看似具體的目標充滿動力。在我工作的同時,我完成了一篇論文的初稿,並在裏面填上了我的實驗結果和一些筆記。然而,令我驚訝的是,Christi 和Dawson似乎都不急於完善這篇論文並讓它得以發表。我當時還做不到在沒有他們幫助的情況下,在這一領域發表一篇像樣的論文,因為我還只是一個幫忙完成苦工的研究助理:真正的研究意義以及有說服力的“營銷”仍然離不開他們。後來,我們完全沒有提交論文,所以我四個月的工作再一次付諸東流,就像我第一年在Klee上的研磨一樣。

因為我自己對領域內的技術不夠精通,加之缺乏年長的同事的指導,這個項目很快就流產了。盡管Cristi 在指導我交叉檢測的思路、以及調試Klee的各種奇怪問題時十分耐心,但可以看出他的精力並沒有完全投入在這個項目上。因為當時他已經即將完成博士學習了,他的主要精力都用在了申請助理教授職位上。教學職位的申請通常需要花費幾個月的時間,投入十足的精力準備,而盡管這樣也仍然有很多申請者無法獲得職位。每年,每所大學的院系至多會招收一到兩名新的終身制導向\footnote{終身制導向教授(Tenure-track professor)職位是區別於研究導向教授(Research professor)職位的一種稱呼,後者和學校的工作合同通常是任意性(at-will)的,相應也不需完成教學的任務。只有終身制導向的助理教授才有機會獲得終身職位。—— 譯者註}的新教授,而上百位能力超群的高年級博士生、博士後(博士畢業後暫時留在實驗室進行研究的學者)、以及從事研究的學者會競爭這些令人垂涎的空缺。申請學術職位是一種充滿壓力、耗盡一個人所有時間和精力的過程。所以,Cristi並沒有動力花幾百個小時的時間再去投稿一篇論文,因為就算論文被接收了,對於他的職位申請而言也已經太晚了。

事後回想,我能明白這個項目因為參與者動機不同註定失敗,而當時的我卻缺乏預知這種失敗的智慧。我當時只是單純地想要和年長的博士生以及富有經驗的教授一起在他們熟知的領域發表論文,才回到了Klee 項目中和Cristi、Dawson一起合作。我之所以這麼打算,是因為同樣的策略在前一年收效顯著,我幫助Joel(年長的博士生)和Scott(富有經驗的教授)完成了他們的HCI項目,而那個項目最終讓我們發表了一篇在一流學術會議中被提名最佳論文獎的論文。

這邊的狀況又有什麼不同呢?簡而言之,Cristi和Dawson都沒有發表論文的渴望。此前,他們已經一起發表了很多篇Klee相關的論文,而和我一起完成一篇交叉檢測的論文,對他們而言是“錦上添花”,但絕非雪中送炭。Cristi已經是博士最後一年了,所以並不需要再發表任何新論文就可以畢業了;Dawson已經獲得了全職職位,所以他也不急於發表論文。相比而言,Joel當時正處於博士生涯的中段,急需發表用於博士論文的\emph{第一篇}論文;而Scott還是助理教授,他需要發表足夠多的論文才能獲得終身職位。這兩次結果完全不同的經歷讓我明白了,在和可能的合作者合作研究之前,深入了解他們對項目的動力和動機的重要性。

\breakline

因為Cristi在忙於申請教職,交叉檢測的項目完全沒有進展,我決定自己起頭,試圖完成一個和Klee相關的項目,而不再繼續作為交叉檢測的實驗助手。和Dawson討論之後,他建議我試著改進一下Klee的核心組件—— 它的\emph{搜索算法}。Klee通過在軟件的可執行代碼中像搜索“迷宮”一樣查找軟件缺陷,所以改進搜索算法可能能讓Klee 找到更多的缺陷。

這是第一次,我開始在Klee上進行創新——改進其搜索算法——而不是為了用Klee找到軟件裏的缺陷而做苦工。對於衡量我的工作的有效性而言,一種辦法是測試Klee在一系列測試軟件程序中的\emph{覆蓋率}(即測試Klee的搜索算法能覆蓋代碼迷宮中多大的部分)。Dawson的目標很簡單:取得一個比最近一篇Klee論文中報告的結果更高的覆蓋率。當時,在89個測試程序上,Klee已經能達到91\%的覆蓋率了(完美覆蓋是100\%),而我的任務就是盡可能提升這一比率。每天,我都會嘗試修改一下搜索算法,然後在89 個測試程序上運行一遍(大概需要十小時的時間),然後第二天早上再查看覆蓋率的數據,然後再改動程序、運行測試,如此往復。

這時已經是我博士第三年的一半了,我和很多同期的同學一樣,都陷入了一種“煉獄”般的狀態,我們都不太會每天到辦公室工作了。我們還因為自己所研磨的問題非常抽象和過分專業而飽受孤獨的折磨,因為我們研究的問題可能除了我們之外很少有人理解或者感興趣。我們的導師和高年級同事會給出一些較高層面的建議,但他們幾乎從不坐下來和我們一起仔細研究項目中每一個惱人的細節。

和我們在公司上班的朝九晚五的同學們不同的是,我們沒有馬上做出任何實物的壓力—— 我們沒有項目截止日期的催促,也沒有中層管理人員需要取悅。對於系裏大部分學生而言,他們偶爾休息一天其實並沒有人會在意,所以以此類推,為什麼不幹脆休息兩天、休息一周,甚至休息一個月呢?所以,博士生在第三年左右退出博士項目並不是什麼應該令人驚訝的事情。

為了防止拖延癥,我不知疲倦地工作,逼自己保持自律,並保持工作日規律地工作。我試著對自己進行“微觀管理”,每天給自己制定較小、容易完成的目標來激勵自己,期望著最終能收獲一個積極的結果。但結果是,因為很難看到自己每天的進展,我很難保持自己在工作中充滿動力。

只有自律是不夠的——在三個月的對Klee搜索算法的調試之後,我仍然沒有得到任何令人振奮的結果。因為Klee在我們的測試程序上已經達到了91\%的覆蓋率,把這個數字僅僅提高幾個百分點,比如到94\%,都極其困難。雪上加霜的是,這些微乎其微的進展寫在論文裏也不會引人註目。我們的論文主旨用一句話來說將是:“我們以某種方式改進了Klee的搜索算法,使得它的平均覆蓋率從91\%提升到了94\%。”這種成果在審稿人眼中很難談得上令人激動,甚至可能連有趣都談不上;這只是一種常見的、無趣的在已有的項目上進行的增量式的工作。不出意外,Dawson也並沒有興趣投稿這樣的一篇無趣的論文。

如果我當時能用一種更有趣且有效的方法改進Klee的搜索算法,可能Dawson會更感興趣投稿這樣一篇論文。然而在2009年1月,我經過三個月的研磨徒勞無果後,我開始覺得自己日復一日的緩慢努力永遠不可能達到一個符合Dawson期待的突破性進展。我不喜歡被稱作一個經常放棄的人,但我意識到Klee搜索算法這個項目只有死路一條,所以我還是放棄了。

現在回想當初,我在一個悲劇性的事實中找到了對當時選擇的寬慰:在我離開Klee 搜索算法的項目之後,Dawson的博士生中有兩個人先後對這個問題展開了研究,結果兩個人都沒能在過去的\emph{三年}中就這一問題發表任何論文。我並不覺得自己能比那兩個學生做的更好,所以如果我繼續在那個課題上努力,可能我也會像這樣卡在一個長達三年的煉獄之中。

\breakline

盡管在Klee項目上屢戰屢敗,我還是想要在它上面做出一點成果,畢竟這是Dawson 當時唯一關心的項目。我開始討厭Klee了,但因為我已經在這個項目上消耗了上千小時的時間,我希望能最終收獲一些實質的成果來證明自己的努力沒有付諸東流。這時已經是我博士第三年的一半,我非常急於發表一篇第一作者的論文作為我博士論文的一部分;我開始感到自己有些落後於身邊的同學們,因為他們中有些人已經發表了自己第一篇可以貢獻於博士論文的學術論文。我天真地以為Klee是我通往博士學位道路上“阻力最小的方向”,因為它和我導師的研究興趣比較貼合。

這時,一個名叫Peter的第一年的博士生加入了Dawson的實驗室,並且在尋找研究項目。我和Dawson商量了一下和Peter合作的事情,考慮是兩個人一起工作可能能比我一個人更容易做出較好的成果。Dawson對此表示贊同,並建議Peter和我一起在Klee 中重新實現\emph{約束下運行}的功能(簡稱“Klee-UC”)。讀者可能還記得,在我博士第一年時,Dawson和他的另一位博士生一起實現了第一個版本的Klee-UC。他們做出了一份粗略的初稿,投稿了一篇在三天之內粗制濫造出來的論文(我仍然清晰地記得當時的慘痛教訓),而在那個學生從博士項目中退學之後不久,項目也就停滯了。所以,兩年之後的現在,輪到Peter和我來重新實現這個想法,並盡量讓它能夠做出足夠好的成果發表一篇論文了。

我帶著我能鼓起的所有勇氣和樂觀精神接受了這個新任務,並努力忘掉我和Klee並不愉快的過去。我對自己說,如果我能發表Klee相關的論文,那一定就是這個Klee-UC 的項目沒錯了。我全心全意地相信Dawson 的Klee-UC創意在學術研究的角度看來是新穎有趣的,所以如果Peter和我能夠足夠努力實現它,並用它找到一些更加重要的軟件缺陷,我們就能發表一篇很好的論文。而且,我還可以重復利用我之前在Linux驅動程序的實驗中大部分的程序和設置。然後,我還幻想著一篇關於Klee-UC的很好的論文可以一掃之前我在Klee 上幾千小時苦工的陰霾,讓我揚眉吐氣。畢竟,還是我博士第一年用Klee查找Linux驅動程序缺陷時的經歷,讓Dawson想到了Klee-UC這個點子。這樣的話,如果我能以第一作者身份完成這篇論文,把想法變成現實,也算是一個合情合理的交代(在我所在的領域,盡管有時候研究想法是教授產生的,他們也會讓自己的學生作為論文的第一作者)。我甚至期待這個項目會成為我博士論文的第一部分,並從此為我最終博士畢業掃清障礙。

在接下來的兩個月之間(2009年2月到3月),Peter和我全心全意地投入到了變成實現Klee-UC之中。我們每天一起在辦公室編程,過程十分有趣;這對於很多博士生所過的孤獨的生活而言是一個很好的調劑。然而不久之後,Dawson似乎就開始對我們緩慢的進展感到不滿。Peter和我都認為我們的進展還算不錯,但是Dawson 似乎對我們的工作很不滿意,所以他很快就打消了寫一篇論文趕在下一個截止日期之前提交的打算。

當時,我完全不理解Dawson為什麼對我們如此缺乏耐心,但現在我能完全理解他當初的感受了。他當時心中有一個十分清晰的Klee-UC的想法,並且希望幾個有天分、並且勤奮努力的研究生能把他的想法變成現實。如果Dawson仍然是一個博士生,他肯定用不了幾個月就能讓Klee-UC正常工作,然後自己單打獨鬥就能寫出一篇頂級論文。當他還是個學生的時候,他的發表文章的記錄就已經非常驚人了,也正是因此,他才能得到像斯坦福這樣的頂尖大學的教授職位。然而,現在他需要完成一些教授需要完成的任務,比如教學、論文/程序委員會工作、論文審稿等等,所以他不能再像之前一樣投入上千小時,專註於把一個想法變成一篇可以發表的論文了。就像所有勞力密集型的研究領域的教授一樣,Dawson需要一些學生的幫助才能把他的想法變成現實。

我認為Dawson期望Peter和我能比實際進展更快得到能用於論文發表的成果,所以對於他來說,我們似乎對於給我們的任務不是能力不足,就是不夠用心。作為一個頂尖大學的教授而言,Dawson所有的學生恐怕都不如博士時代的他自己有能力,這著實是一個不幸的事實。而這一事實的原因很簡單:在頂尖大學的博士畢業生中,大概75個學生中只有一個能有幸在斯坦福這樣的大學任教(而對於普通大學的博士畢業生中這個數字可能是200比1)。所以其實並不出乎意料的是,Peter和我都不是這樣出類拔萃的人才。如果Dawson 能和他自己的年輕版的克隆人一起研究,估計研究進展會快很多!

盡管Peter和我在那兩個月中在這個項目中投入了很多精力,但我們能清楚地意識到Dawson 對我們並不滿意。Peter對此感到很失望,於是他之後更換了導師,並最終從博士項目中退學了。我的合作者離開之後,我對Klee的最後一絲幻想也隨之破滅,所以我也很快離開了Klee項目,而且再也沒回來過。

\breakline

Peter和我離開Klee項目的兩年之後,Dawson終於找到了一個博士生,可以幫助他把他Klee-UC的想法最終變成現實。2011年,Dawson和這位新學生發表了一篇非常優秀的論文,其中包含了Klee-UC和交叉檢測的思想。最終,通過四個博士生長達五年的努力,Dawson的Klee-UC的想法終於變成了一個可以發表論文的研究項目。在這四個學生中,只有一個“幸存”了下來——我離開了Klee項目組,另外兩個博士生則直接退學了。從每個學生的角度看來,成功的概率實在是過分渺茫了。

然而,從一個教授的角度來說,Klee-UC是一個令人激動的成功!因為Dawson已經獲得了終身職位,所以他的工作並沒有什麼風險。事實上,終身教授制的主要意義之一就在於,獲得終身教職之後,教授們就可以有機會嘗試更加大膽的項目思路。不過,這個制度的陰暗面就是,往往教授們會讓學生去風險較大的項目上研磨,而這些項目的成功率往往並不高。學生也並不能拒絕教授的要求,因為他們都是由導師的科研經費支持的。幸運的是,我有外源獎學金的贊助,所以對我來說離開Klee項目要容易很多。

我並無意專門把Dawson和Klee拿出來,作為批評的典型。在\emph{大多數}勞力密集型的科學與工程學的研究領域中,這種在終身教授和博士生之間學術動機的鴻溝都是普遍存在的。通常的情況都是,首先,教授獲得一些研究經費,並產生一系列高瞻遠矚的研究思路(比如Klee-UC或者交叉檢測)。然後,教授用研究經費將一些學生招入麾下,指導他們實現自己的研究思路,通常(但並不一定)作為這些學生的畢業論文的主要工作。離開了這些上千小時的學生的工作時間,這些研究思路永遠不可能轉化為現實的成果,更遑論發表論文了。

有時,教授們可能需要經歷好幾屆的學生在項目上的失敗甚至退學,直到某一屆學生最終將項目引向成功。這一過程有時需要兩年,有時需要五年,有時甚至需要十年的時間。很多項目常常持續的時間都超過了博士生的“生命周期”。但只要最終這些項目的想法得到實現並形成論文得以發表,項目就算是成功了。教授對此感到很滿意,院系也感到很滿意,出資資助研究項目的機構也表示滿意,最終將項目引向成功的學生也會感到滿意。但在這之前,一路上“傷亡”的學生呢?終身教授通常可以接受幾年的損失在臨時的失敗上,但對一個博士生來說,這一系列的失望可能導致的是學術事業夭折在萌芽中,甚至心理健康都可能岌岌可危。

\breakline

2009年5月,在我博士第三年接近尾聲的時候,我參加了Cristi的\emph{論文答辯}。論文答辯是博士生獲得博士學位之前最終的“神聖儀式”:博士生就自己博士期間和博士論文相關的研究做出一個一小時的口頭匯報,並需要回答一個教授小組提出的尖銳的問題。Dawson通常都很安靜和內斂,而這次當他向觀眾介紹Cristi、並激動地說到和他一起合作創造出Klee是一段令人愉快的時光時,他臉上的驕傲和自豪清晰可見。他對Cristi的誇獎並非空穴來風:Cristi在博士階段的工作無可指摘,而且他在Klee 中發展出的思想,創立了軟件缺陷檢測領域中的一個新的子研究領域(稱作\emph{混合式實際/符號化程序運行})。

看過Cristi的論文答辯展示之後,我更清晰地意識到自己幾乎不可能從Klee中繼續做出一整個博士論文的工作了,這也進一步增強了我離開Klee項目的決心。Cristi 閃耀的成功,讓Dawson年輕的學生們發表論文和最終畢業變得更加困難了。從無到有的Klee的創造的工作已經完成;所剩下的大多是在此基礎上一些添加和改進,或是將Klee應用在一些新的軟件上,比如Linux驅動程序軟件。盡管這些工作也可以最終成為可以發表的論文,甚至成為博士論文的一部分,Dawson已經不像其它剛剛進入我們研究領域的競爭者一樣急於發表論文了。

因為Klee(以及一些2005年到2008年的相關項目)創建了一個全新的子領域,很多助理教授和年輕的學者很快就加入到了這個大潮中,並迅速發表了很多篇論文描述針對這些項目的改進,以期獲得終身教職或者職位晉升。這就像一個學術圈的淘金潮,而Cristi、Dawson以及其它一些少數的先驅者引領了這個潮流。因為Dawson已經獲得了終身教職,並創造了包括Klee在內的很多著名的項目,他並不需要隨波逐流,也沒有強烈的欲望發表很多論文來填滿自己的簡歷了。

這一現象的結果是,和那些年輕的研究者一起開展研究的博士生更容易發表論文並取得博士學位,而與此同時Dawson的學生似乎都沒那麼順利。在我離開Klee項目後的三年中,來自世界各地的很多研究小組發表了上百篇和Klee思路類似的論文。令人欣慰的是,其中有15篇論文描述了對Klee本身的改進——作為我們開放Klee 源代碼,以促進這方面研究進展的結果。同時,Dawson的博士生中有五位在Klee上下過苦工;而到目前為止,\emph{只有一位}成功地發表了一篇論文(即Klee-UC)。

讓然覺得諷刺和不幸的是,現在作為學術會議程序委員會成員和論文審稿人的都是Dawson 的直接競爭者,盡管他參與創立了這一研究領域,他自己的論文也更難以得到發表了。因為Dawson已經有幾年沒有積極發表論文了,他已經對於發表論文所需要的修辭技巧、新誕生的術語、以及符合審稿人口味的一些“營銷技巧”有些生疏,所以在頂級會議發表論文已經變得困難了。此外,他的競爭者們發表越多的文章,審稿人對論文的要求也就越加苛刻,他和他的學生在發表論文時也就越需要說明自己研究思路的與這些論文的區別與聯系,最終導致他們的論文更容易被拒收,收獲更多令人失望的結果。這很諷刺,因為歸根結底,如果不是因為十年前Dawson的遠見,這些挑剔的審稿人根本不可能在這一領域開展研究,更別說拒收他的論文了!

我考慮了一下,留在Klee項目中唯一的優勢就只剩下Dawson發自內心熱愛這個項目了。盡管可能我最後不能在這個項目上發表論文,也許他也能最終給我一個“憐憫畢業”。但因為我之前在Klee上的經歷並不順利,我沒有勇氣在Klee上再研磨不知道多少年的時間,僅僅指望一個“憐憫畢業”的結果。

這時,我已經度過了我博士生涯三年的時光,但仍然對我最終要怎麼完成一篇博士論文毫無概念。我對未來沒有什麼計劃,但我清楚地知道,自己肯定不可能再回到Klee 項目中來了。

%%% Intermission
\mychapter{插曲}

我博士第三年結束後,我馬上就動身去了華盛頓州的西雅圖,去那裏的微軟研究院總部實習。那個夏天成了我生命中最有趣和最有成效的夏天之一:我實習時參加的研究項目讓我發表了三篇頂級論文,而且更重要的是,這個項目為我的博士論文奠定了基石。

現在,微軟研究院已經是一個進行高水平學術研究的企業研究所了。在更多的公司中,研究所往往將精力投入在研發工程中,以期改善公司即將推出的產品。然而,微軟研究院(簡稱MSR)的主要任務則是進行科學與工程方面的基礎科研,並以在相關領域發表頂尖學術論文作為目標。

如果做比喻的話,MSR可能最像是一個沒有學生的、巨大的研究型大學。全職的研究員就像這裏的教授,而區別是他們並沒有教學或者指導學生的義務,所以他們可以將自己的幾乎全部精力投入到研究當中。可能他們最享受的職業優勢,則是不需要去申請科研經費,而對教授們來說這是一種周而復始、極其消耗時間精力的工作。因為微軟已經是一個利潤可觀的公司,所以它可以每年將成百上千萬美元的盈利用於支持學術研究(發表論文)。微軟的打算是,由其麾下的研究員研發的一些知識產權可能作為新產品靈感的來源,而且它也希望能擁有計算機科學界最頂尖的研究者能作為自己的顧問。正因為如此,微軟公司給予了研究者們所有他們所需的資源,讓他們完成最好的學術研究。

在MSR獲得全職研究員的難度不亞於在一個知名大學獲得教職的難度。盡管技術上來說MSR 的研究員們並沒有終身職位,他們的工作保障很優越,只要他們能保證不斷發表學術論文。因為在計算機科學領域,很多研究領域都是勞力密集型的,所以研究員們經常招募博士生作為暑期實習生,幫他們完成自己的研究計劃。這對於雙方來說都是個不錯的交易:研究員們可以請博士生們來幫助處理一些研究中的苦工,而博士生們同時也能有機會和學校以外的著名研究者一起,發表頂級的學術論文,甚至之後得到工作的推薦信。在過去的十年間,計算機科學領域頂級學術會議的論文中,有相當一部分都是由MSR 的研究員和他們的實習生完成的。

當初夏我抵達MSR總部時,整個地方都充滿了活力,有上百位博士生在和他們的經理會面,並準備開展工作。因為我們只能在那裏實習三個月,我們的經歷都為我們精心策劃了研究項目,以期在實習結束時能發表論文。我們中的大部分人都能用自己暑期實習的工作投稿一篇論文,而其中一部分最終能夠得到發表。當然,學術研究本身是有風險的,所以也有一些實習生被分配到的項目並不能產生可以發表的論文。盡管如此,幾乎每個人都很享受在這裏的時間——我們的工資比平時在學校的津貼幾乎翻了兩番,微軟還會組織有趣的出遊活動增進大家的交流,我們還有機會聆聽由一流的學者做出的激動人心的學術講座。

可能MSR實習帶來的最深遠的影響,就是我們在實習中形成的友誼。在那個夏天,我有幸結識了一些我同輩中最聰明、最充滿靈感的年輕的計算機科學家。比如,和我同一個辦公室的三位實習生之一的一位女生即將開始她在MIT的博士項目,而她在本科學習期間發表的頂級論文可能已經超過了大多數博士生在博士期間所能發表的數量。另外一位是來自加州大學伯克利分校的博士生,在工作日白天的實習工作之外,他還將每天晚上和周末都用於有全國各地的合作者參與的另一個研究項目。他們都很可能將來成為獲得諸多榮譽的教授、研究領袖、或是高科技行業的創業新秀,所以我感到非常榮幸能和這些人共事了一個夏天。

\breakline

關於我如何獲得進入MSR實習的機會,這個過程則是一個對充分結合自己的研究成果、並積極發展專業內的人際關系,其重要性的明證。很多博士生都會通過某種人際關系獲得實習機會(以及之後的全職工作),而我也不例外。

在我博士第二年,和Scott、Joel在他們的HCI項目上開展研究時,就申請過一次MSR 的實習。我通過常規渠道申請,在網上遞交了自己的簡歷,但我的申請很快就被拒絕了,因為實習崗位被發表論文更多、且通常有更多內部人際關系的博士生們占去了。

一年之後,在我博士第三年,一位MSR的研究員註意到了我和Scott、Joel合作的HCI 項目(論文發表在一個該領域的學術會議上),所以他給我發了一封電子郵件,問我是否有興趣在一個大致相關的項目上,跟著他做實習。他專門找到了我,是因為在我MIT本科的時候,我的研究導師曾經將我介紹給他過,所以他對上了號。

我對他的邀請深感榮幸,但我回復了他我已經不在HCI方向做研究了;那時,我已經回到了Klee項目中,繼續進行著軟件缺陷檢測的研究。不過,我表達了自己對去MSR 開展經驗化軟件質量度量課題的強烈興趣,因為我博士第二年都在和Dawson做這類的工作。他馬上就把我的郵件轉發給了他的同事Tom,而Tom是經驗化軟件度量領域正在冉冉升起的新星。通過郵件簡單地自我介紹之後,我將自己和Dawson撰寫的關於Linux 缺陷報告度量的論文發給了Tom。Tom很喜歡我的論文,於是便決定招聘我作為他的實習生。在我博士第二年中我曾經讀過好幾篇Tom的學術論文,所以我對於他招募我為他工作,感到十分激動。

如果我只是像上百位其他申請者一樣,盲目地在網上遞交自己的申請,可能我永遠也沒機會吸引Tom的註意。很多其他的實習生也是通過某種人際關系得到實習職位的,不過對他們而言,可能是導師直接將他們推薦給了在MSR研究方向相關的學者。有趣的是,不是Dawson,而是我本科一位研究導師(我們的項目大概是在六年前)為我提供了這種我當時亟需的人際關系。這位導師後來還為我的第一份全職工作,介紹了一位關鍵的人\footnote{原文是``provide a crucial introduction'',也可能是指介紹了一個機構/組織等,但從上下文無從推斷。這裏譯者采用了更符合中文語言習慣的翻譯。——譯者註}。從這個經歷,我明白了被有影響力的人推薦的重要性;在一個充滿挑戰的領域,僅僅做得好是遠遠不夠的。

\breakline

Tom在較高層次上為我的實習研究制定了計劃,並鼓勵我完成一個可以實現但雄心勃勃的目標——在夏天結束之前投稿一篇頂級學術會議的論文。我的項目是使用量化的方法研究在軟件缺陷報告生成之後,缺陷被修復,被分配給其它開發者,或者被認為已經修復、然而後來發現還有問題等,這些現象中背後的人為因素。為了深入研究這個問題,我需要寫程序分析軟件缺陷數據庫,並對比微軟內部的員工個人信息。我當時非常擅長做這類的數據挖掘和分析的工作,因為我整個博士第二年都在和Dawson一起,就Linux的缺陷報告和版本控制歷史數據做著類似的分析。

每天下午五點,Tom在下班之前都會來看看我的進展。盡管每天的進度檢查可能讓人壓力很大,但實際上我發現這些進度檢查非常有幫助,因為Tom態度和藹,對我的工作也並不抱有批判的態度。每天都能得到及時的反饋,讓我能保證專心在工作上,也更有工作的動力。這種確定、短期的目標和持續、有效的反饋的組合,使得我實習期間的工作日比之前三年博士生涯中任何一段時間都更加高效。最讓我開心的是,我每天只在正常的工作時間工作(早九點到晚六點)。我並不能將工作帶回住處,因為我需要的數據都只能從微軟內部才能訪問,所以我每天晚上都有機會出去放松,而不需要擔心自己是不是要做更多的工作;而在學校時,我常常在擔心自己是不是工作得夠多,因為幾乎醒著的每一分每一秒我都能工作。

由於Tom發表過也評審過很多軟件度量方向的論文,他是絕對的“局內人”,他知道什麼樣的結果和寫作方式在這一子領域更受歡迎。在夏天結束、我們提交論文的時候,Tom非常熟練地在相關研究的基礎之上說明了我們研究工作的貢獻,闡明了我們論文的結果新穎、有影響力的原因,並盡量完善了論文的敘述。三個月後,我就聽說了我們研究軟件缺陷修復的論文被一個頂級會議收錄的好消息,而那年這個會議的收錄率只有14\%。

但Tom沒有就此收手!因為他剛剛入職MSR不久,他非常渴望能夠發表更多的跟進論文,以建立自己在公司的名望。在接下來的幾年中,我們用我2009年暑期實習時得到的實驗結果又發表了兩篇頂級會議的論文,一篇是關於缺陷報告負責人變更的,另一篇是關於缺陷報告重新開放的(這篇論文獲得了\emph{最佳論文獎})。

\breakline

我在MSR和Tom開展經驗化軟件度量研究所收獲的成功(發表了三篇頂級論文),對我來說是一個令人滿意的拯救——將我救出了博士第二年在這一領域失敗的低谷(投稿了兩篇論文被拒,最後只發表了一篇次級會議的短文)。因為我在這兩次截然不同的經驗之間並不是變得聰明了很多,我將我實習項目的成功歸功於兩個因素:微軟和Tom。

首先,作為一個MSR的實習生,我可以訪問到包含微軟軟件缺陷和人員資料在內的大量內部數據。如果是作為外部人員,我是永遠不可能獲得這些數據的訪問權限的。MSR所擁有的龐大的數據資源使得他們的研究員們(比如Tom)可以更快地做出開辟性的研究,而他們的競爭者因為沒有相應的數據,可能永遠都做不到這一點。相比而言,當我和Dawson 一起做研究的時候,我能獲得的Linux數據集規模小了很多,質量也差強人意,因為開源軟件項目往往不如一個世界一流的軟件公司在維護數據記錄上更為用心。

此外,Tom在其中的貢獻也不可或缺:因為他在經驗化軟件度量領域是老手,所以他知道應該如何作為一個技術導師來指導我,他還清楚論文需要怎樣的精心調整才能盡可能最大化被接收的概率。與此相反的是,Dawson在這一領域只是一位充滿研究熱情的局外人,所以對他而言,他並沒有在這一領域開展研究的動力,也不能良好地指導我完成研究項目(盡管他在另一領域——軟件缺陷檢測——是世界知名的學者)。

在我博士第二年時,因為Dawson和我在經驗化軟件度量這一領域發表論文困難重重而倍感煎熬,因為我們的競爭對手都是這一領域的專家和老手。現在當我終於和這些專家之一一起發表論文,站在了成功的這一邊時,這種感覺令我由衷地欣喜。

\breakline

盡管我在暑期經歷了一段美妙的“插曲”,當秋天回到斯坦福時,我仍然對博士論文的項目毫無頭緒。我只知道我不想再Klee項目上繼續消耗下去了,但我對於什麼樣的研究工作使我感興趣——而且更重要的是,可以發表論文——仍然一無所知。

我考慮過將我在實習做的研究工作擴展成我的畢業論文。然而,我最終得出的結論是,我回到斯坦福之後就很難在這個問題上發表更多論文了,因為那樣我就無法訪問微軟的數據了。最好的打算似乎是,我能在微軟完成自己博士論文的所有相關工作。然而這顯然可能性不大,因為我沒聽說過之前有學生這樣做過。

作為最後的努力,我嘗試聯系了我在谷歌實習時的經理,以尋求重新回到谷歌實習,利用他們關於軟件缺陷的內部數據集繼續經驗化軟件度量的研究。他似乎對此沒什麼意見,但我後來沒有繼續跟進這個計劃:因為他自己也不是學術研究者,而除了他之外,我似乎也想不到誰會願意幫我完成這個研究計劃。所以,我最終決定放棄經驗化軟件度量這個方向,所以在MSR發表的三篇論文最終也沒有為我博士畢業做出貢獻。然而,這些經歷對於我提升研究水平和論文寫作水平,無疑產生了積極的影響。

因為求畢業心切,我在暑期實習期間,將很多天晚上和周末用於閱讀學術論文,或者是在咖啡店展開頭腦風暴。在某一階段,我甚至產生了一個想法,那就是創造一個和Klee 類似的博士論文項目,但不使用到Klee這個軟件本身。這個計劃可以一方面讓我遠離Klee的深淵,而另一方面也保持Dawson對我研究項目的興趣。不幸的是,我沒能從已經發表的論文中找到我自己原創的、有足夠創新性的思路。

接著,2009年7月24日,就在我的暑期實習剛剛過半的時候,靈感突然降臨了。在MSR 的辦公室編寫程序分析數據的時候,我開始想到了一個想法,而這個想法最終轉化為了組成我博士論文的第一個項目。我發狂似地寫下了一頁接一頁的筆記,並讓我的朋友Robert幫我確認我的想法不是異想天開。當時,我完全沒有預料到這個想法會不會被更廣泛的學術圈認可,但對我而言,這至少是我回到斯坦福、開始我博士第四年時,一個明確的前進方向。

%%% Year Four: Reboot
\mychapter{第四年:卷土重來}

在我整個博士第二年、和2009年在MSR暑期實習期間進行經驗化軟件度量研究的過程中,我一直在使用一種叫Python的非常流行的編程語言\footnote{Python是一種解釋型的編程語言,語法靈活且簡潔明了,非常適合用於編寫需要很快完成的數據處理、分析程序,也因此被很多研究者、數據分析師等采用。},進行數據處理、分析、和可視化。具體來說,為了我的程序可以運行,我需要在我的電腦上存儲很多程序所需的數據;為了我的程序在數據分析過程中,經過每一次很小的改動之後可以運行得更快,我不得不把我的程序改得過分復雜。當這種想法在我的頭腦中慢慢地醞釀一個夏天之後,在七月的一個安靜的下午,我突然萌生了這樣一個想法:如果我能以某種創新的方式修改Python 的運行環境(術語稱為\emph{解釋器}),也許我就能消除我每天面對的很多低效的工作,從而提高使用Python進行計算的研究者的工作效率。我將我對Python解釋器的改進想法稱為IncPy,這來自於增量式Python (\textbf{Inc}remental \textbf{Py}thon)的縮寫。

\breakline

就像所有務實的研究者一樣,我想到IncPy想法之後的第一件事(其實是從一開始的激動平靜下來之後的第一件事),就是仔細地在網上搜索,試圖確定是不是有人曾經做過相關研究。謝天謝地,沒人曾經做出過我計劃完成的工作,但的確有一些之前的研究項目和我的想法有點接近。不過這不是大問題;沒有什麼研究創意完全是原創的,所以學術界永遠存在著很多相似、相關的研究項目。不過,如果想最終發表論文,我需要令人信服地說明IncPy和這些已經存在的項目有何不同。在幾天之內,我就簡單地規劃出了一個初步的項目計劃,包括一些說明IncPy為什麼獨特、新穎、值得研究的論述。

因為我還有一個多月才離開MSR,我充分利用了我周圍的環境——有很多聰明的同事都在像我一樣,為他們的研究編寫著類似的數據處理程序。我和幾位同事一起喝咖啡談過話,我簡單地問了問他們在工作中進行數據分析時的習慣,以及他們工作中遇到的低效問題。我還向他們簡單闡述了自己的IncPy想法,並和他們討論了如果要讓這個想法更加實用、以及更加適合作為學術研究問題,項目的哪些方面還需要微調。這些早期的談話增進了我的信心,讓我知道我正前進在正確的方向上,因為其他人和我一樣對數據分析中的低效感到失望,並且渴望看到一個IncPy這樣的解決方案。

那個夏天接下來的時間裏,我將晚上和周末都用於在咖啡店裏不斷完善我逐漸成型的IncPy 想法,強化它的“營銷宣傳”,並且從我的MSR同事們中獲得反饋。我還將我的想法的初稿發給了Dawson,但那時我並不太關心他對這個想法感不感興趣,因為這將會成為我自己的研究項目。我不是在征求他的同意;我只是想告訴他這是我秋天回到斯坦福之後想要開展的研究工作。

\breakline

我的博士生涯在我在斯坦福的第四年重新啟程,我斬斷與過去三年的聯系,並翻開新的一頁。不再在已有的研究項目上開展工作,不再去嘗試制定計劃使得自已與教授和高年級學生的研究興趣相一致,也不再去擔心什麼樣的研究才是學術圈樂於看到的。我現在拼命地只想把我IncPy的新想法實現出來,形成一篇可發表的文章,並使之成為我畢業論文的第一部分。

雖然充滿了新的熱情與激情,我還是有些憂懼的,因為我是脫離了原有的工作來做一件新的工作,並且還未得到教授的支持,不知道是否會成功。我們系大多數博士生都會做他們的導師或者教授感興趣的項目,畢竟有了教授的幫助和專業知識會更容易發表論文。然而,即使沒有教授撐腰,我還是預感IncPy會成為一個能發表出來的想法;在過去三年中的研究失敗中我積累了不少經驗教訓,從而對哪些想法可能會成功有了一個更好的直覺。對這種直覺的堅信不疑成為了我博士生涯的轉折點。

還有一點比較務實的事是,我依然保持Dawson作為我的導師,因為他同意我做自己的項目,而且還能時不時給我的工作一點高層面的回饋。由於我仍然有外源獎學金來作自我基金支持,我也就沒有義務像Dawson的其他學生那樣必須繼續在Klee項目中工作。在我博士的接下來幾年裏,Dawson開始不太插手我的工作了。我們碰面討論項目可能總共才十余次;我大多時候還是自己工作或者尋找其他合作者。但我卻不想正式地更換導師,因為那樣的話,我到了一個新的研究組裏還是得再一次“交學費”,一切從頭做起。況且,我也不知道系裏還有哪個教授對我IncPy的想法感興趣,不然的話我可能還是會認真考慮一下換導師的事情的。

為了最終能夠畢業,我需要找到三位同意閱讀並且會批準通過我的畢業論文的教授,來組成一個博士論文委員會。大多數學生會直接請導師幫他們選擇這個委員會的成員,畢竟他們一直在做導師指定的“官方”項目。但由於我單飛了,沒有在Dawson的Klee項目上開展工作,事情就沒有這麼簡單了。我努力地尋找我們系或者其他相關院系(如統計系,生物信息學系)的教授來做我的委員會成員。我給一些教授發了冷郵件,並試圖通過他們現在指導的學生聯系到他們。我甚至還做了一個幻燈片準備向這些潛在的委員會成員展示我畢業論文的初步想法,但不幸的是,沒有教授有興趣和我面談。然而,我沒有選擇放棄並回到有院系教授支持的傳統項目上去,而是冒險在IncPy的征途上繼續前進,並且覺得委員會成員的事一定會車到山前必有路。

\breakline

受第二年時與Scott和Joel在人機交互工作上的啟發,我2009年秋天一回到斯坦福,就開始采訪一些寫Python 程序來在研究中分析數據的同學。我的目標是發現他們在編程中被什麼影響了工作效率,以及如何能用IncPy消除這些影響效率的問題。我的一些朋友還幫我在他們的實驗室組會中安排了報告的機會,讓我能介紹關於IncPy的想法——盡管它當時還只是半成品。在項目啟動階段的這些采訪和匯報,對我新想法的提出和“營銷技巧”的提升都有很大幫助。我也特別感謝那些在我除了幾頁粗制濫造的幻燈片而外別無其他東西展示時,還能幫助我讓我的項目起步的朋友們。

我越來越感到振奮,因為我發現很多基於計算的領域,例如機器學習、制藥學、生物工程、生物信息學、神經科學,還有海洋工程,都在使用相似的方法進行數據分析,並且能從類似IncPy這樣的工具中受益。經過了幾個星期的調研采訪和後續的計劃細化以後,我自信我有了足夠的賣點來令人信服地“推銷”出這個項目,形成一篇論文投稿出去。我想提出的論點是:各種領域的眾多計算型研究者每天都在與編程工作流中常見的低效問題抗爭,而IncPy正好提供了針對這些低效問題的一個完全自動的解決方案,這是之前沒有人做到的。這個初期形成的賣點最終也演化成了我整個博士論文的主題。

有了一個總體的策略之後,我準備好了開始一個長達幾千小時的辛苦工作——研磨—— 這也是把IncPy從想法變成現實的原型工具所必經的工作。在那個夏天快要結束的時候,我扮演了一個“教授的角色”,草擬出了高層面的設計提綱,作了報告演講,並細化了概念性的想法。現在我準備好了轉成“學生的角色”,在接下來的一年經過辛苦的研磨從而把IncPy變成現實。

\breakline

這時我也清楚有了一個不錯的想法只是做出可發表的研究成果的必要條件,而絕非充分條件。年輕的研究者——通常是博士生——必須在這裏那裏花費成百上千小時,在細節處揮灑汗水,最終才能讓想法結出果實。在計算機科學的研究中,苦力勞動最主要的形式就是編寫計算機程序來搭建、測試和評估基於軟件的新的原型工具和技術。我在過去十年中花了近萬小時在各種形式的編程工作上,這些工作有的在課堂上、有的在業余生活中、有的在實驗室裏、還有的在工業界的實習中,這些讓我對實現一個類似IncPy這樣的研究創意所必須的、高強度的、復雜的編程要求有了充分準備。

實現IncPy包含了一些我從未完成過的編程中的臟活累活。如果我在過去十年沒有經歷過這麼多小時的如火煉真金般的訓練,那我也絕不會去嘗試這麼一個勞動密集型的項目。不用懷疑,肯定也有其他人已經發現了我所註意到的計算型研究工作流中的這些低效問題,他們本可能成為我的競爭者,但將我與他們區分開來的要素是,他們沒有創造出像IncPy這樣\emph{完全自動的}解決方案所需要的深厚編程功底。他們可能最多也就能做出一個半自動的解決方案而已,而實質上這樣的工作還不足以在優秀的會議上發表論文。

雖然我不太喜歡我研究生初期和本科期間的研究項目,但我從這些經歷中獲得的技術上的技巧以及判斷力,使得我現在能將自己真正關心的想法實現出來變得可能。在接下來的三年中(2009年到2011年),我不知疲倦地做出了五款幫助計算型研究者的新原型工具(IncPy是其中的第一個),發表了多篇\footnote{原文是``published one or more first-author papers describing each tool",這裏沒有直譯,而是根據上下文翻譯成了“多篇”。—— 譯者註}介紹這些工具的第一作者論文,接著將這些工作組合在一起形成了一篇我深感自豪的博士論文。那三年——我博士生涯的後半程——是我迄今為止最具創造力和最高產的一段時期,與我博士生涯的前半程形成了鮮明的對比。

強烈的\emph{想要做出成果的狂怒}讓我瘋狂地主動工作。我變得剛毅、果決、高度專註。每當我回想起我前三年博士生活所忍受的低效、失敗、沮喪所帶來的煎熬時,我心中就會產生一團怒火,然後逼迫自己工作得更努力;我被這種強迫性的鞭策所驅動,來彌補假想的浪費掉的時間。誠然,早些年的光陰實際上並沒有浪費;沒有那些掙紮,我也不會獲得創造出組成我博士論文的這五個項目的靈感和能力。

\breakline

然而,2009年9月,在我開始我第四年博士生涯的時候,我還不知道有什麼會在未來等著我,也絕沒有像五個項目、230頁的畢業論文這樣的宏偉計劃。我甚至都不知道有沒有教授會認真地看待我未經檢驗的想法,並同意擔任我的論文評委。我想做的就只是將IncPy 實現出來,盡可能讓其發表,然後到時候再來考慮接下來一步做什麼。

當我正準備開始(編寫程序)實現IncPy的時候,出人意料的好運從天而降。當時,我還不知道這麼一件小事竟能帶來一系列的好運,為我鋪開了一條通往畢業的大道。有一天午飯的時候,我的一個朋友Robert給我說他正計劃把他一個新研究項目的論文投稿到一個研討會\footnote{研討會:相比起學術會議(conference),研討會(workshop)的規模較小,更傾向於集中討論某一個特定的話題。—— 譯者註},這個研討會的論文截止日期是兩個半月之後。

按常理來說,我對這件事應該不會有太多想法,有兩個原因:首先,Robert的研究主題(一個叫做\emph{數據溯源}的子領域)和IncPy並不相關,所以他準備投稿其實和我沒有一點關系。第二,在我們系,一篇發表的\emph{研討會論文}對於博士論文來說並不“作數”。研討會論文的目的是宣傳初期的研究想法,它並不像會議論文那樣要經過嚴格的審稿。會議論文的接收率一般在8\%到30\%,而研討會論文的接收率卻有60\%到80\%。 我們系的大多數教授都不主張學生往研討會投稿,因為論文一旦被接收(這通常也是很容易的一件事),教授就得從他們的經費中拿出大約1500美元,來承擔學生參加研討會並作報告產生的車旅費、住宿費和註冊費。送一個學生去參加研討會的費用已經幾乎和參加一個會議的費用相當了,但發表會議論文對學生和教授來說卻又更有聲望。因此,頂級的計算機科學教授會強烈建議學生發表更多的優質會議的論文,同時也要避免投稿到研討會。

我問了問Robert為什麼他的導師會建議他把初期的想法投稿到一個研討會,而不是再深入挖掘一下,日後再投稿到一個學術會議。他說部分原因也是出於方便,因為這個研討會就在聖何塞\footnote{San Jos\'e,加利福尼亞州城市,距離斯坦福校園不足一小時車程。——譯者註}舉行。他們整個組都準備去參加這個研討會,聖何塞離斯坦福也就20英裏的路程,所以他也想在那做一個論文報告。

出於好奇,我瀏覽了一下那個研討會的主頁,看一看組織者們對哪些話題感興趣。雖然這是一個關於數據溯源(Robert的研究領域)的研討會,它的話題列表裏卻包含了這麼一點:“高效/增量式再計算”。我心裏暗想:\emph{嗯,IncPy為Python程序員提供了高效的增量式再計算,所以如果我以合適的方式推廣我的論文,也許還正好契合這個研討會!}由於我不需要長距離奔波就能參加這個會,所以如果我的論文被接收,我也就不用忐忑地去找Dawson 讓他幫我報銷註冊費了。我發郵件給Dawson告訴他我準備往這個研討會投稿的計劃,他不冷不熱地回復了我。不出所料,他對研討會提不起多大的興趣,但是覺得我要想投稿過去也可以,因為他很敬重這個研討會的程序委員會的聯合主席,一名名叫Margo的哈佛教授(她後來也扮演了一個至關重要的角色,幫助我得以畢業)。

時間也正合適:我還有兩個半月來加班加點地實現出第一個能夠工作的IncPy原型,然後寫一篇研討會論文並投稿。由於我知道研討會的接收門檻比會議論文的低很多,所以我的壓力也沒那麼大。我只需要讓一個基本的IncPy原型能夠合理地運行,並且生動地討論一下這個經驗就行了。我發現這種找到並為自己設定短期目標的策略出奇地有效,也讓我能夠在我後來的博士生涯中一直保持專註。如果沒有一系列自我設定的截止日期,很容易就會陷入倦怠期並且患上拖延癥。

\href{http://www.pgbovine.net/projects/pubs/guo_tapp10_camera_ready.pdf}{我的論文}最終被接收了,還獲得了不錯的審稿意見,我在2010年2月參加了那個研討會並做了個半小時的報告。值得一提的是,我的論文還很榮幸地得到了程序委員會聯合主席Margo的贊揚,她還指出了我的論文是如何與她的學生Elaine正起步的一個基於Python 的新項目產生關聯的。Elaine沒有來參加這次研討會,不過Margo給了我Elaine的郵件地址並建議我們聯系一下。

起初,我擔心Elaine會成為我的直接競爭對手,還可能在我之前發表會議論文,這會使得我發表一篇關於IncPy的會議論文變得更為困難:因為在學術界成為第一人是很被珍視的,一旦其他研究者在你之前發表了一篇類似你的想法的論文,你再想把你的想法發表出來就會變得更難了。但是經過幾次試探性的郵件接觸和視頻聊天之後,我發現她並沒有把這個工作作為會議論文投稿的打算,這讓我放松了一些。而且,她的工具也沒有全自動的能力,而這正是把IncPy和其他類似工作區分開來的特點。當我們意識到彼此不是競爭對手後,我們立馬成為了朋友。我很高興在接下來的幾年裏,我還和Elaine保持了聯系。後來我和Margo重新取得聯系,讓我能夠奔向博士征途的終點,部分也要歸功與Elaine。

\breakline

即使做這樣一個有關IncPy的研討會論文報告有利於獲得反饋,尤其還讓我有機會能和Margo會面,但那篇論文對我的博士論文而言並不算做是“真正”發表的論文。我知道我仍然還需要將這個工作發表在我們系教授所認可的會議上。研討會論文和會議論文最大的區別就是一篇會議論文必須有一個有說服力的\emph{實驗評估}來體現論文中所描述的工具或者技術的高效性。一篇論文的評估章節可以是多種形式,包括從對運行時間的測量,到用戶行為的控制性實驗研究。由於很多研究者都提出了相似的想法,所以審稿人會仔細地審查這些想法是如何被實現出來的,以及試驗評估是如何設計的,從而決定哪些論文可以接收,哪些論文應該拒絕。

雖然在IncPy項目開始時,我就知道展示一個令人信服的實驗評估是很困難的,因為我想說明的論點——IncPy 可以提高計算型研究者的生產力——是一個主觀且模糊的概念。在讀了一些同樣也提出生產力的提升這一論點的論文之後,我設計了一個可分為兩部分的評估策略:
\begin{enumerate}
  \item \textbf{案例研究:}從各種計算型研究者處收集一些用Python語言編寫的程序,並進行模擬,看看如果研究者們采用IncPy而不是常規的Python 後,生產力可以提升到什麼程度。
  \item \textbf{部署:}找一些研究者來讓他們在日常工作中使用IncPy而不是常規的Python,之後再讓他們報告是否IncPy提升了他們的生產力。
\end{enumerate}
下一個相關會議的論文提交截止日期在七個月之後,我計劃在那之前準備好論文。我花了很多時間來尋找用於案例研究的程序樣本和用於部署的潛在用戶。沒有這些,就沒有評估,也就沒有會議論文,也沒有博士論文,更別說畢業了。(當然,我還是把我大部分醒著的時間花在了惱人的寫代碼、調試,以及測試上,從而來實現IncPy。)

我學著扮演一個一半推銷員一半乞求者的角色,堅持不懈地詢問同學是否有Python 程序可以拿來給我用作案例研究,或者更好的是,是否願意安裝並使用IncPy,以此作為日常工作的基礎,並向我提供他們的使用感受。如我所料,我得到的回答多半是否定的,但我還是禮貌地詢問了他們能否給推薦其他一些我可以聯系聯系的人。我也到許多研究組的組會上做了不少報告來招徠大家對IncPy的興趣。歷經了幾個月的“乞求”之後,我獲得了來自不同領域的近十個研究者的Python程序,足夠我進行案例研究了。我感激當時幫助過我的每一個人,畢竟他們除了來自一個素不相識博士生的善意請求外,沒有得到任何東西。

\breakline

雖然案例研究對我論文中評估的那一個章節來說已經足夠了,但我真心希望的還是\emph{部署}我的原型工具—— 讓真正的研究者使用IncPy。這不僅是因為我覺得通過部署我能夠做一個更有力的評估,而且更重要的是,我發自內心地相信IncPy可以提升計算型研究者們每日工作的生產力。大多數研究型原型工具的搭建都是為了證明一個新穎的想法,但之後就被拋棄了,但我想讓IncPy 成為一個實用且能被沿用下去的工具。我並不是為了發表論文才憑空想出IncPy 這個點子的,創造它的靈感來自於我研究時碰到的真真正正的實際的編程問題,所以我希望人們也能夠在實際中使用它。

盡管我追求理想主義,但我也理解為什麼幾乎沒有研究型的原型工具成功地得到部署使用。這背後的原因是因為人們往往不願意去嘗試一個全新的工具,除非他們看到這個工具有很多立竿見影的好處,而且沒有一點缺點;研究者們卻通常沒有足夠的時間和資源來讓他們的原型工具運行得足夠好,從而滿足這些嚴苛的條件。尤其對我而言,我的目標用戶對使用常規的Python很滿意——即使它存在一些低效問題,他們也不願意冒險轉移陣地到IncPy上來。為了說服某個人試一試使用IncPy,我就必須保證IncPy在任何情況下都比常規的Python表現得要好。盡管這在理論上是可以保證的,但實際上IncPy 只是一個僅由一個學生維護的研究型原型工具,所以它一定會存在很多漏洞。相反,官方版的Python是一個有二十余年歷史的項目,而且同時還有幾百個經驗豐富的程序員在維護它,所以它更加穩定和可靠。只要有一個人發現了IncPy的一個漏洞,大家立馬就會覺得IncPy 是靠不住的,轉而回去使用常規的Python。 我知道勝算不站在我這邊,但我不在意。

在斯坦福尋找人來安裝和使用IncPy未果後,我轉而在附近的大學來尋找願意第一個吃螃蟹的人。2010年3 月,我給UC Davis\footnote{UC Davis是加州大學戴維斯分校(University of California, Davis)的簡稱。——譯者註}的幾個博士生發了封冷郵件,他們都是和我同在MSR實習的一位同學的朋友。感謝他們無私奉獻的精神,在我這個研究生亟需援助時拉了我一把,還邀請我去宣傳我的IncPy。 幾個和藹的教授甚至還同意和我會面,包括一個我在數據溯源那個研討會上碰到的教授。盡管我得到了不少有幫助的反饋,我還是沒能找到人作我案例研究或者部署的實驗對象。

那天晚上我待在了UC Davis,打算第二天早上回斯坦福。突然,我有了個沖動的想法—— 給Fernando發封冷郵件,他是UC Berkeley\footnote{UC Berkeley是加州大學伯克利分校(University of California, Berkeley)的簡稱,出於習慣,下文將簡稱其為“伯克利”。—— 譯者註}的一個研究科學家,他非常熱心於將Python投入到計算型研究中。幾個月前,一個來聽我作報告的研究生同學給我發了封郵件,裏面有一個Fernando的一篇博客的鏈接,我當時就記下了一個備忘錄,準備在將來某個時候聯系一下Fernando。現在似乎正好是一個合適的時機:伯克利正好處於UC Davis和斯坦福之間,我回學校的路上可以順便去他辦公室拜訪他一下。我給Fernando發了冷郵件,詢問他明天早上是否有時間和我聊一聊。這是一個不太會成功的嘗試,不過他最終還是答應了和我見一面。我和Fernando歡快地聊了一個小時;有這麼一位資深的高級科學家能支持我IncPy的想法,我感到非常高興。

第一次與Fernando會面最重要的一個收獲就是他邀請我接下來這個月回伯克利做一個有關IncPy的報告。他說我的聽眾中會有使用Python進行計算型研究實驗的神經科學家。在我那個一小時報告的過程中,我被三個研究員稍稍嚇了一跳,他們老是打斷我,反復糾纏一些IncPy在實際中表現如何的細節問題。一開始,我覺得這幾個人實在是太過追根究底了,不過報告後他們過來向我表達了試用IncPy的強烈興趣。他們抱怨的所遭受的低效問題正好是我創造IncPy的初衷!看來他們並不是在我報告時添亂;他們是真真正正想了解其中的細節,以此來評估將IncPy部署到他們實驗室的計算機上去是否真的可行。

擁有第一批用戶的這種可能性讓我很激動。我和他們郵件交流了幾次,還專門驅車去伯克利協助他們安裝和設置IncPy。我的第一封郵件采用了一種謹慎樂觀的語氣:
\begin{quote}
  非常感謝您有興趣成為我的IncPy記憶式Python解釋器的第一個用戶!我非常期待能將IncPy做成一個您能在日常工作流中使用的工具。我覺得最大的問題應該是安裝、設置、配置中惱人的工作,我會很樂意解決這些問題,從而給您最流暢的用戶體驗。
\end{quote}

在我嘗試給伯克利這些神經科學家的電腦上安裝IncPy之前,我已經搭建和測試了幾個月,所以我很自信也能給他們安裝成功。然而,安裝之後沒一會兒,我們就發現IncPy與這些科學家天天都在使用的很多第三方Python插件(稱為\emph{拓展模塊})並不相容。我自己測試的時候,只測試了一些基本用例,而沒有考慮到這些擴展模塊。這給了我一個關於現實部署的深刻教訓:失敗可能以你沒有料到的形式出現,而一旦給用戶留下一個不好的第一印象,就全完了!就像絕大多數研究者只在象牙塔裏閉門造車做原型工具一樣,我絕不會想到這個不可預見且不起眼的拓展模塊問題,會讓我在第一次部署的嘗試中出局。

但我並沒有放棄。我花了接下來幾個星期的時間重新設計和實現了IncPy中一個關鍵部分的代碼,使它能夠與任何Python拓展模塊一起完美工作。我又給伯克利那幾位神經科學家發了封郵件,希望他們再給我一次機會,但我沒有得到任何回復。唉,我曾經有一個絕佳的機會,可我卻揮霍了它。

\breakline

這個慘痛的經歷驅使我不斷地從實際的角度出發來不斷完善IncPy:一邊修補了幾個不明顯的漏洞,一邊我還制作了一個\href{http://www.pgbovine.net/incpy.html}{IncPy的項目網站},上面包含一個短短的演示視頻,文檔以及新手教程。做這些東西的幾百個小時對我原本的研究沒有絲毫貢獻,但這對我獲得實際用戶的事例,從而在將來投稿時完成論文的評估章節來說卻是必須的。

幾個月以後,來自世界不同地方的三個陌生人通過這個網站發現了IncPy,他們下載了IncPy,並將其用來提升他們研究中一些編程工作的速度。雖然三個人只是一個可憐的用戶總數,但也總比零好,而大多數研究型原型工具用戶數就是零。IncPy已經達到了一個打磨得還不錯的狀態,我很滿意——這多虧了我從伯克利回來之後的做的改進——這幾個陌生人現在可以不需要任何人工指導就能使用IncPy了。這三個人發郵件告訴了我IncPy是如何在他們的一些工作中起到幫助的。這些事跡雖然並不是IncPy高效性的強有力證據,但也聊勝於無。

2010年9月,在我的第四年博士生涯接近尾聲的時候,我把我IncPy的論文投稿到了一個頂級會議,論文的評估章節包含了案例研究和三個簡要的部署實例。這個會議之前一年的接收率只有14\%。所以我知道我的論文很有可能被拒收,既因為它極低的接收率,也因為我的IncPy 並不能很好的契合任何一個傳統的子領域。盡管這樣,先以頂級會議為目標,之後有必要再重新投稿到二級會議仍是一個明智的選擇,畢竟頂級會議的論文發表在將來畢業的時候會占到更大的權重。

我仍然還沒擁有一條通往畢業的康莊大道,但我至少能開始自我管理自己的研究內容,而不是跟著別人一起做他們的研究項目了。我在過去這一年還是挺開心的,我把腦海中靈光一現的IncPy想法變成了一個半實用的工具,還讓三個下載它的陌生人從中受益。雖然這只是一個很小的成就,而且對我的畢業沒有絲毫幫助——教授們更看重理論上的新穎性,而非實際中的部署情況——但在我博士的第四年末尾時,這多少還是有一點讓人滿意的。

%%% Year Five: Production
\mychapter{第五年:崢嶸歲月}
我的博士第五年(2010年9月)剛開始的時候,我還沒有任何東西可以放到我的(還未存在的)博士論文裏去。這時,我的大多數同學都以第一作者身份發表了至少一篇博士論文水準的會議論文。而我卻一篇都還沒有(IncPy的那篇論文當時還處於評審階段),因此我很擔心我將需要七八年才能順利畢業。

然而在接下來的十二個月中,我發表了四篇會議論文和一篇研討會論文(均為第一作者),這也鋪平了我通往畢業的道路。毫無疑問,我的第五年是博士生涯中最高產的一年,我不懈地保持著專註。

\breakline

2010年的夏天進行到一半的時候,IncPy項目正按部就班的進行著,我也正準備著在九月的截止日期前投一篇論文。但我知道對於博士論文來說,IncPy還不算是一個實質性的工作。所以除了繼續努力準備投稿而外,我也花了一下時間想了想下一個項目要做什麼。

我希望可以把我的頭腦風暴說成是受純粹的學術精神所激勵,但事實上我是受日益增長的恐懼所驅使:我擔心我不能在一個能接受的時間內順利畢業,所以我逼迫我自己去想一些有希望發表出來的新想法。我也知道一篇論文被接收發表可能會花兩到三年,因此我要是想在第六年結束時畢業,就得在今年提交幾篇論文,並祈禱其中有至少兩篇會被接收。我感到了一種緊迫感,因為我的外源獎學金只持續到這一年結束。等我的經費一到期,要麼我就得去從教授那裏獲取經費支持並面對一些強制性的限制(例如再一次在Klee上投入工作),要麼我就得長期地擔任助教,而這會更加推遲我的畢業期限。時間真的不多了。

2010年7月29日,幾乎就是我想出IncPy的初始創意正好一年之後,我又有了一個相關的想法,同樣也是受計算型研究者分析數據時碰到的實際問題的啟發。我註意到由於研究者們經常以一種自發的“馬虎”的方式寫程序,他們的程序就會常常由於一些低級錯誤而崩潰,沒能產生任何一點分析結果,讓人垂頭喪氣。我的想法就是通過改變Python 語言的運行時環境(\emph{解釋器}),我可以消除掉這種崩潰,讓這些馬虎的程序能產生出一部分的結果,而不是什麼結果都沒有。我把這個Python解釋器的修改版命名為“SlopPy”,代表\emph{\textbf{Slop}py \textbf{Py}thon}\footnote{``sloppy''一詞意為馬虎、不謹慎的,所以作者的項目全稱是“馬虎的Python”,而縮寫也是“馬虎的”。——譯者註}。

雖然SlopPy和IncPy是非常不同的兩個想法,但我都是通過改變Python解釋器的行為來實現它們的。過去一年為了做IncPy,我花了近一千個小時來研究(修改)Python 的解釋器,這讓我有信心能夠輕易地實現出SlopPy。只用了兩個月,我就做出了一個能工作的基本版原型工具,跑了一些初始的實驗,並提交了一篇論文到一個二級會議。我以那個會議為目標,既是因為它的截止時間比較合適,也是因為我覺得SlopPy 這個想法還不足夠“大”來讓它被頂級會議接收。

\breakline

到了2010年10月的時候,我有兩篇論文都處於評審中。這時,我已經放棄了找一份大學教授工作的想法,畢竟我連一篇能用到博士論文裏的論文都還沒發出來;有競爭力的計算機教職候選人在博士生涯的這個時候都已經發了很多篇飽受贊譽的第一作者論文了。所以除非奇跡發生,要不然我是找不到一個一流大學的研究工作了,因此我把完成工作夠得上畢業定為了我的目標,都不再去擔心我的簡歷字面上好不好看了。

我收到的結果是奇跡的反面:我IncPy和SlopPy的論文都被拒收了。我感到有點失落但也並不震驚,因為在這之前我已經習慣了論文被拒。對我論文的批評意見都是合情合理的,所以解決這些問題可以強化我之後的投稿。

最值得註意的是,我介紹IncPy時采用了一種不太明智地包裝策略,這使得我的論文被安排給一些來自對我的研究理念不太“友好”的子領域的學者來審閱。理論上來說,專業論文應該只基於它的內容來評判,但實際上,審稿人有著自己不同的主觀品味和理念差異。所以我徹底地重寫了開頭用來自我推銷的部分,希望獲得更多友善的審稿人的贊同,我還將它重新投稿到了一個二級會議,以此來進一步增加它被接收的機率。我的計劃奏效了,2011年年初,我\href{http://www.pgbovine.net/projects/pubs/guo_issta11_camera_ready.pdf}{IncPy的那篇會議論文}的第二次投稿被接收了,盡管評審意見有點冷淡。

後來我又修改並重新提交了SlopPy的那篇論文到一個研討會,這個研討會正好和我要作報告介紹IncPy 的那個會議在一起舉行。這個策略很有效,因為想讓一篇論文被一個研討會接收比被一個會議接收實在是容易太多了。而且,Dawson也不用花額外的錢讓我去參加這個研討會,因為我本來也是要去同期舉行的那個會議上做IncPy 的報告的。\href{http://www.pgbovine.net/projects/pubs/guo_woda11_camera_ready.pdf}{SlopPy 的論文}和我料想中的一樣被接收了,盡管它不“作數”,只能算作我博士論文外的額外成果,但至少也比沒有論文發表好;我希望能把這篇論文吸收到我的博士論文中去作一個小章節,以此作為那些來自會議論文的更重要章節的補充。

\breakline

2010年10月,我剛把IncPy和SlopPy兩篇論文投出去的時候,我問了Dawson我要畢業的話需要達到什麼條件。不出所料,他回答我說我需要發表一些論文來證明我工作的合理性。不過他還有一個更為具體的建議:另一個能到達博士論文水準的項目,就是能夠把我對Python的興趣以及他所喜愛的類似Klee的想法結合起來,從而做出一個能夠自動查找Python程序中漏洞的工具。由於我對回到任何形式的Klee項目都不太感興趣,所以我沒采納他的建議,而是繼續思考IncPy和SlopPy還有沒有什麼可拓展的東西,可以形成一篇後續論文提交出去。

這時,我博士論文主題的初期構想已經開始在我腦中萌芽:IncPy和SlopPy都是\emph{提升計算型研究者生產力的軟件工具}。因此,思考接下來一個項目要做什麼的時候,我仍接著去尋找計算型研究者在工作中會碰到什麼樣的問題,然後設計新的工具來解決這些問題。

具體來說,我註意到研究者們每天運行計算實驗時,都要編輯並執行他們的Python程序幾十上百次;在有一個重大發現之前,他們往往要重復這個過程幾周或者幾個月。我覺得把每一次程序執行之間的變化記錄下來並進行比較,會對調試程序和獲得思路有幫助。為了便於進行這種比較,我打算拓展IncPy,使之能夠記錄下當一個Python程序執行時,哪部分代碼和數據被存取了的細節,從而維護一個計算型實驗的詳盡歷史記錄。我也覺得如果研究者之間能夠共享這種\emph{實驗歷史記錄}的話那就太棒了,這樣他們就能夠從中知道哪些實驗嘗試是奏效的,哪些是不行的。

我的直覺告訴我按這個思路發展出來的想法肯定是新穎的,而且能夠發表出來,但我還沒有形成一個清晰的研究亮點;我的思路還都是模糊的一團。我覺得思維有點被卡住了,於是我又尋求和Fernando見一次面,就是我博士第四年在伯克利一個實驗室的組會上作報告介紹IncPy時見過一次面的那位研究科學家。Fernando把與我的會面排入了日程,我們這個小時的交談為我的下一個項目播撒下了種子。

\breakline

我一把我拓展IncPy來記錄基於Python的實驗歷史的想法告訴Fernando,他立馬饒有興致地給我介紹起了一個我從沒聽過但卻很吸引我的話題:\emph{可重復性研究}。

實驗科學的根基之一就是任何一個人的研究發現應該能夠讓同行驗證、比較,並且以此為基礎繼續發展。在過去的十年中,來自不同領域越來越多的科學家都在通過編寫計算機程序來分析數據和獲得科學發現。每年都有幾千篇充斥著以數據、圖像、表格作支撐的定性結果發表出來。然而,現代科學中大家都心知肚明的一個令人遺憾的事實就是,想重復或者驗證他人的發現幾乎是不可能的,因為他們的原始代碼和用來產生結果的數據集幾乎很難得到。這造成的結果就是,大量論文中包含的觸目驚心的錯誤——有的的確是無心之失,但有的卻是公然造假——變得無法被檢驗,有的甚至變成了導致人員喪命的科學論斷。最近幾年,諸如Fernando這樣的改革派科學家們正竭盡全力喚起大家對計算科學中可重復性研究重要性的重視。

為什麼可重復性在實際中如此難以實現呢?一部分有很強競爭力的科學家故意不公開他們的代碼和數據,以此來避免潛在的競爭,但大多數人還是願意在別人有需要時分享給他們代碼和數據的。然而,主要的技術壁壘還是僅僅獲得了別人的代碼和數據,還是不足以重新運行和產生他們的實驗。每個人的代碼都需要一個高度特殊化的\emph{環境}來運行,任意兩臺計算機上的環境——即使它們有相同的操作系統——還是會有一些細微且互不兼容的差別。所以,如果你把你的代碼和數據發給你的同行們,他們還是有可能沒法重新運行你的實驗。

Fernando很喜歡我IncPy的記錄實驗歷史的想法,因為這樣可以記錄下一個實驗最初出現時,有關其軟件環境的信息。之後,使用Python的研究者就可以把他們的代碼、數據和環境發給想要重復他實驗的同行。會談出來以後我感到非常振奮,我找到了我想法的一個具體應用。可重復性研究的這個動機似乎也足夠吸引人,可以組成我第二篇基於IncPy 論文的故事情節,還能成為我博士論文的一部分。

正當我把更多的細節草記下來的時候,一個極為清晰的念頭瞬間襲來:\emph{為什麼只把實驗記錄限制在Python程序上呢?}有了我頭腦中的一些想法,我可以做出一個工具,能夠輕松重復任何語言寫的計算型實驗。頭腦還十分混亂的我草擬了一個計劃,我準備設計一個新工具,名字就叫“CDE”,也就是\emph{\textbf{C}ode, \textbf{D}ata, and \textbf{E}nvironment}\footnote{意為“代碼,數據,環境”。——譯者註}的意思。

\breakline

當我把我的想法告訴Dawson時,他非常支持,還鼓勵我再想得大膽一點:\emph{為什麼只把CDE瞄準在科學家的代碼上?為何不做一個針對所有種類軟件程序的多用途工具包?}這的確是字字珠璣。一大批軟件的創建人和貢獻者——不只是科學家——都遇到過讓別人運行自己的程序時,由於同樣的環境不兼容問題帶來的煩惱,這一現象被親切地稱為“依賴地獄”。依賴地獄問題在基於Linux的操作系統中尤為廣泛,因為用戶使用的不同變種的Linux之間存在大範圍的不完全兼容現象。在一個用戶的Linux 計算機上能運行的程序不一定也能在另一個用戶的稍有不同的Linux計算機上運行。只要把我的原始想法稍加改進,CDE就能幫用戶打包他們的程序,從而讓其他用戶能夠成功運行,而不再用擔心環境不一致的問題。我感到十分激動,因為CDE 有可能緩解Linux 裏存在了十余年的依賴地獄問題。

按慣例要進行實際調研,於是我在網上搜索了相關工作,看看有沒有什麼研究型原型工具和成熟產品級別的工具有著類似的功能。讓我欣慰的是,已有的工作並不算多,而且CDE在這個寬松的競爭環境中有兩方面顯得尤為突出:首先,我將CDE設計得比類似的工具更容易上手使用。作為用戶而言,你要新建一個自洽的代碼、數據和運行環境的包,就只需要運行你希望打包的程序就可以了。因此,如果你在Linux計算機上運行了一套程序,那麼CDE就能讓其他人在他們各自的計算機上都能返回到這一相同的程序,而不需要其他的安裝或者配置。第二,CDE所采取的的技術機制——一種叫\emph{系統調用重定向}的技術—— 使它在各種各樣復雜的實際情況中,比其他相關的工具更為可靠。

這時,CDE還只是以一堆筆記和設計提綱的形式存在,但當我意識到它比所有現存的工具概念上都更簡單,更易於使用,且更為可靠時,我還是感受到了它的巨大潛力。我身體裏的一部分感到震驚和激動:\emph{居然之前沒有人實現過這個東西?!回想起來這個想法這麼顯而易見!}一個令我懼怕的之前沒人做類似CDE工具的原因可能就是,要搞清楚所有的細節並讓其在實際中高效工作幾乎是不可能的。可能CDE也是一種論文上看起來不錯但卻實際中不可行的創意。我於是覺得除了自己親自實現一遍CDE這條路而外,沒有其他更好的辦法把這件事弄清楚了。

在橫跨2010年10月到11月那緊張的三個星期中,我都在為創建出CDE的第一個版本而高度研磨。正如我所懷疑的那樣,雖然CDE背後的研究創意是很直截了當的,但要讓CDE 在真正的Linux程序上工作,還有很多編程相關的臟活累活要幹。那幾個星期我與CDE 生活甚至呼吸都是在一起,生活中的其他事都被我拋之腦後。我夜以繼日地編寫程序,還常常在夢裏思考我需要解決的代碼中錯綜復雜的細節問題。每天早上,我剛醒來就跳起來編程,生怕今天我會遇到一個難以逾越的障礙,最終證明讓CDE在實際中工作是不可能的。不過這一天遲遲沒有到來 ,我也離我的第一個裏程碑越來越近:證明CDE是如何能讓我在兩臺Linux計算機之間毫不麻煩地傳遞一個復雜的科學程序,並重復一個實驗的。

在經過三周靠咖啡驅動的高強度研磨之後,我感到狂喜,我終於讓CDE在我的一個科學程序演示樣本上正常工作了。這時,我知道如果我繼續測試並不斷改進代碼,那麼CDE 有在多種實際的Linux程序上起作用的潛力。我制作了一個十分鐘的介紹CDE的演示視頻,搭建了一個包含視頻和CDE下載拷貝的\href{http://www.pgbovine.net/cde.html}{項目網站},並把網址發給了我的幾個朋友。我不知道的是,有一個朋友還在著名的計算機極客論壇Slashdot上幫我發了這麼一個廣告:
\begin{quote}
  斯坦福的研究者Philip Guo開發了一個名叫CDE的工具,可以自動將Linux程序及其所有依賴環境(包括系統級的庫,字體等等)打包,並能直接在另一臺Linux計算機上運行,而不需要做設置庫和程序版本,或者解決版本依賴地獄問題這些工作。他已經上傳了二進制文件、源代碼和視頻截圖。看起來應該對集群/雲部署和程序共享有很大幫助。
\end{quote}
還不到24小時,Slashdot論壇上的那個帖子就有幾百個回復了,我也開始收到幾十封郵件,都是來自世界各地下載並試用了CDE的Linux熱衷者的。其中有一些珍貴的來信,比如:“我就是想說你太棒了!這個創意能夠實際起作用實在是令我折服。我會在我們墨西哥提華納哲理\footnote{郵件原文為``hear'',是``here''(這裏)的常見混淆詞,兩者發音相同,故翻譯采用了“這裏”類似發音的“哲理”。——譯者註}的Linux社區裏推薦使用這個工具。”這些來自實際用戶的未經修飾的、隨感而發的贊譽,比起那些來自研究同行們對我之前創意或者論文的表揚,對我而言意味著更多。

\breakline

從科研的角度來看,我的使命現在已經完成了:我成功地創建了CDE的一個初始原型工具,並證明它在一個實際用例上是可行的。大多數應用工程領域裏的通常觀念認為,像CDE這樣的研究型原型工具就只是用來證明新穎想法的可行性的。研究者的工作就是創建出原型工具,實驗性地評估它們的效率,寫論文,然後開始又一個項目。作為研究者,期待人們像使用真實產品一樣使用你的原型工具其實是很天真的事情;如果你的創意很好,那門專業的工程師自然會適當地修改它,形成他們公司未來的產品。最好的情況是,其他一些研究組可能會把你的原型工具作為基礎,搭建出他們自己的原型工具然後撰寫論文並引用你的論文(例如,有十幾個其他大學的研究組拓展了Klee工具,並將他們改進的地方寫成了論文)。但卻幾乎很少聽到非科研工作者在他們的日常工作中會使用原型工具。總之,學術研究的目的是做出被驗證了的想法,而不是精美的產品。

所以,當時正確的行動路線應該是提交一篇關於CDE的論文,然後開始想一個新想法,實現出一個新原型工具,提交一篇新的論文,再如此反復直到我有足夠的內容來組成我的博士論文。我也確實提交並發表了兩篇關於CDE的會議論文(\href{http://www.pgbovine.net/projects/pubs/guo_usenix11_camera_ready.pdf}{一篇介紹性的短論文}和\href{http://www.pgbovine.net/projects/pubs/cde_LISA.pdf}{一篇後續的長論文})。但我之後並沒有像一個深謀遠慮的研究者那樣開始一個新項目,而是將我博士第五年的大部分時間都投入到將CDE轉變成一個有產品質量級別的軟件這一工作上去了。

我有很強烈的意願讓CDE使盡可能多的人受益。我並不願意讓CDE就此沒落下去,淪為又一個剛夠得上發論文的劣質原型工具。我知道改善CDE所付出的汗水將不會在科研界得到任何好處,甚至還有可能令我延期畢業,畢竟我本可以用這些時間來開發一個博士論文裏的新想法。但我不在意,因為這一年剩下的時間我仍然有外源獎學金支持,擁有絕對的自由去花時間做一個公益性質的軟件維護者,而不是一個被研究基金所累的傳統型研究者。

回看我的第四年,我極度渴望地想讓人們使用IncPy,這也是為什麼只有可憐的三個用戶時我還是很激動。盡管最終幾乎沒人接著使用IncPy,我想把它做成一個實際世界中工具的不理智願望還是促使我接觸到了伯克利的Fernando,而且也正是Fernando啟發我做出了CDE。現如今,在2010年11月,我第五年開頭的時候——在我的演示視頻出現在Slashdot 網站上的幾天內——CDE就已經有了幾十個用戶,而且還可能擁有得更多。從之前的反饋郵件來看,我意識到我做出了一個人們在多種我未曾預料到的情況下都想使用的東西。一言以蔽之,CDE引起了各種被依賴地獄所折磨的Linux用戶的共鳴。

\breakline

我把第五年中的大部分時間都花在了修復CDE的漏洞上,使其能夠在紛繁復雜的Linux程序上順利工作;潤色說明文檔、用戶手冊,以及常見問題,使得它更易用;與世界各地的用戶們郵件交流甚至是電話溝通;還做了不少報告,並發了些“營銷”郵件來吸引新用戶。

這時(2012年夏),CDE已經被超過10000人下載和使用了。我收到了用戶們的上百封郵件,裏面包括反饋、新特性要求、漏洞報告,還有一些有趣的軼事。盡管對商業軟件產品來說,這並不是一個很大的數字,但對僅由一個研究生維護的免費開源軟件來說,這已經是一個很大的數字了。

給我發了郵件道謝,並述說他們是如何在日常工作中利用CDE消除Linux的依賴地獄問題的軼事的用戶有以下這麼幾類:
\begin{itemize}
  \item NASA\footnote{NASA,即美國國家航空航天局(National Aeronautics and Space Administration)的簡稱,是美國聯邦政府的一個行政機構,負責制定、實施美國的民用太空計劃與開展航空科學及太空科學的研究。——譯者註}研究員(Ames和JPL研究中心\footnote{NASA華盛頓指揮部為NASA的最高管理機構,此外NASA還下設多個研究中心,Ames和JPL分別為埃姆斯研究中心(NASA-ARC),和噴氣推進實驗室(NASA-JPL)。——譯者註})
  \item 來自諸如植物生物學和醫學信息學這些領域的科學家
  \item 將計算型實驗部署到歐洲網格分布式計算架構\footnote{歐洲網格架構(European Grid Infrastructure)這一計劃致力於利用網格計算技術在歐洲提供高通量計算資源。——譯者註}中的科學家
  \item 軟件公司進行實驗代碼原型設計的工程師
  \item 開源軟件創建人和貢獻者
  \item Linux計算機系統管理員
  \item 在各種基於Linux的非兼容操作系統變種上運行程序的Linux愛好者
  \item 大型反病毒軟件公司的計算機安全分析師
  \item 非營利性技術項目“每個孩子一臺筆記本電腦”中的程序員誌願者
  \item 需要發布他們的研究原型工具的計算機系的博士生
  \item 大學裏開發諸如醫學可視化和蛋白質結晶軟件的程序員
  \item 大學裏需要打包編程作業的助教
\end{itemize}
那幾個月是到那時為止我博士生涯最愉快的一段時光,即使我知道我維護軟件的這些工作對我的博士論文不會有一點貢獻。CDE取得了初步成功之後,我不再在意我是否會因為沒有更多的論文,而延期一年或者幾年才能畢業;得知了我發明的軟件提升了這麼多人的計算體驗,我感到非常滿意。

\breakline

CDE還讓我實現了我長期以來的一個書呆子夢想:在谷歌做一個由YouTube在線直播的科技報告\footnote{谷歌科技報告(Google Tech Talks)是谷歌在視頻網站YouTube上的一個草根節目,用來向技術社區分享有趣的信息。——譯者註}。因為從讀研究生一開始,我就喜歡在線看有關各種學科的Google科技報告。我夢想著有一天我也能做這麼一個報告,但我也沒抱太大希望,因為谷歌似乎只會邀請一些知名的教授或者工程師——而不是不知名的研究生——來做報告。

有一天我正在網上搜索和CDE相關的項目時,偶然發現我2007年暑假在谷歌實習時的實習經理最近在一個可重復性研究研討會上發了一篇論文。我給他發了封郵件,把CDE作為一個使可重復性研究變得方便的工具推薦給了他,並詢問他的同事們是否有興趣試用一下。讓我驚喜的是,他給了我一個以眨眼笑臉結尾的回復,並邀請我去做一個報告:“我看了看你的這個工具。看起來很有意思啊!你有興趣來谷歌做一個科技報告嗎?我會幫忙組織和宣傳。不過不知道會有多少人來聽,可能有100個,也可能一個都沒有;-)”

比起先前的報告,我花了更多的時間來準備在谷歌的科技報告,因為我知道這次報告會被錄像。\href{https://www.youtube.com/watch?feature=player_embedded&v=6XdwHo1BWwY}{我的報告}進展得很順利,而且講完以後谷歌的一個工程經理(我之前從未與他謀面過)還把我拉到一邊,問了一些更細節的後續問題。看得出他對緩解公司內部的Linux依賴地獄問題很感興趣,這也是為什麼他喜歡我的報告的原因。他還給我提供了一個實習生職位,我可以在即將來臨的這個暑假來谷歌開展有關CDE的工作。

我很高興他能給我提供這個實習職位,也花時間深思熟慮了一番。我們系的教授通常不鼓勵博士後半期的學生去實習,因為他們希望學生們能更專註於完成他們的畢業論文。並且,獲得這個實習職位時,我還沒有發表過哪怕一篇能夠用到博士論文中的第一作者論文(有幾篇論文正處於審稿階段),所以我有點擔心暑假離開斯坦福會給Dawson留下一個我並沒有認真對待發表論文和畢業的印象。然而,我內心的直覺又認為這是一個我不能拒絕的絕無僅有的合適時機:我可以獲得一份很不錯的薪水,在暑假裏繼續開展我自己的開源項目工作。相反,幾乎所有的實習生——包括我2007年實習時——都只能在經理分派的公司內部項目上開展工作。我和Dawson聊了聊我這一左右為難的感受,他很支持我去實習,於是我接受了這個實習邀請。

我在谷歌度過了很棒的2011年暑假,把我幾乎所有的工作日都投入到提升CDE、獲得新用戶、找到Google內的一些獨創性的用途上去了。那個暑假的部分時間,我還和一個谷歌工程師開展了緊密的合作,他覺得CDE對他的日常工作很有幫助,這也是我修復新漏洞、改善文檔的一大動力。這時,我也沒再繼續想和CDE相關的新研究想法了:我只是研究細節並繼續改善CDE。在暑期實習結束、我博士第六年開始時,我最終停止了在CDE上的全職工作。

在構成我博士論文的五個項目當中,CDE是我最喜愛的,因為這一簡單優雅的創意最終變成了一個擁有超過10000個用戶的實用工具。它是目前為止從科研角度來說最不復雜的一個研究想法,但卻是工作起來最令人滿意的一個,因為它和實際世界有著緊密聯系。

\breakline

回到我第五年開始時——那時距我IncPy、SlopPy和CDE的論文發表還有很長一段時間——我計劃了一個備份方案,以防我自己的項目失敗。我給我們系裏的一個和我研究興趣相仿的新任助理教授Jeff發了封冷郵件,詢問他是否願意和我合作一個能夠用到我博士論文裏去的項目。我提到的兩個關鍵“賣點”是我有自己的資金支持,而且也願意把投論文到他喜歡的頂級會議作為自己的目標。而作為交換,他需要擔任我的論文委員會的成員。

和料想中的一樣,他接受了我的提議。對他來說這是一個極佳的協議,畢竟我們的出發點十分契合:我是一個需要發表論文來畢業的高年級學生,而他是一個需要發表論文來獲得終身教職的助理教授。更好的是,他不需要使用他的研究經費來負擔我的開銷。而最好的一點則是我願意做任何他想要我做的項目,因為我原來的那些個人項目已經給了我一直渴望的研究獨立性。

我用這個計劃來規避賭博的風險:如果我IncPy、SlopPy和CDE的想法不能得以發表的話,那我至少還有論文委員會成員Jeff批準的一個“正當”項目。Jeff和我認為對我來說最好的策略就是,以他的另一位學生去年創建的交互式數據格式重建工具Wrangler為基礎,繼續開展工作。

在第五年的末尾,我停止了在CDE上的工作,並花了兩個半月來新做了一些Wrangler的拓展功能。我的這個增強版本叫做“ProWrangler”,代表\emph{\textbf{Pro}active \textbf{Wrangler}}。 在實現了ProWrangler,並請同學來做控制性的用戶測試從而評估了它的效率之後,我在Jeff和原始版Wrangler工具的幾個原創者的幫助下,寫了一篇論文並投稿到了一個人機交互的頂級會議。

在2011年暑假進行到一半的時候,我收到了我們\href{http://www.pgbovine.net/projects/pubs/prowrangler_uist11_camera_ready.pdf}{ProWrangler的論文}被接收且審稿意見很不錯的好消息。到這時為止,這個成果最大的貢獻者就是Jeff在論文中引言和評估結果解釋部分的撰寫工作。我們的用戶測試其實並沒有展示出我們原本所期待的生產力提升效果。但不可思議的是,Jeff的專業寫作和議論的謀篇布局技巧把這個幾乎失敗的工作變成了一個驚喜的勝利。審稿人們很欣賞我們誠實地承認了評估環節的缺陷,並從中提取出了有價值的東西的這一做法。毫無疑問,如果沒有Jeff在寫作修辭方面的老練,我們的論文絕不會被接收。他有很多實踐經驗,在還是一個博士生的時候,他就發表了19 篇基本全都是頂級會議的論文,這是一個普通博士生發表數的五到十倍。這也是獲得一個像斯坦福這樣一流大學的教職所需要的硬性條件。

\breakline

在我的整個第五年中,我一直得為四個項目——IncPy、SlopPy、CDE和ProWrangler——來仔細地分割時間來發展思路、實現原型工具、提交論文、修改論文、再次提交論文,這四個項目的相關會議的論文提交截止時間遍布於整個這一年中。雖然我花了不少時間來培育CDE項目,但當論文提交截止日期來臨的時候,我還是得轉而專註到其他項目的工作上去。2011年暑假時,這四個項目的論文都成功發表了,它們大都經過了好幾輪的論文修改。我也終於松了口氣,因為我復雜的計劃終於有了回報,而且一個完整的畢業論文也似乎觸手可及了。

%%% Year Six: Endgame
\mychapter{第六年:塵埃落定}

在我第五年快結束時——我去谷歌夏季實習之前——我和Dawson見了一面並問了問如果我想在接下來這一年畢業需要些什麼。那時,IncPy和CDE都作為二級會議論文發表了,SlopPy是一篇研討會論文,ProWrangler的論文還在審稿中。Dawson表達了擔憂,他認為我的論文發表記錄還不足以畢業,所以我需要一個或者多個更實質性的貢獻來形成我的博士論文。他的期待似乎也是合理的,所以我的計劃是在秋季回到斯坦福,花幾個月來做一些能夠完成我的博士論文的新研究。但我擔憂的是我已經在過去幾年的高度研磨中變得精疲力盡,再沒有新的創意萌芽了。所以我又回到了計劃模式,想想有什麼方法能完成最後這未知的、能使我畢業的工作。

作為我策略的一部分,我還需要找到第三個(也是最後一個)能夠為我的畢業做強有力擔保的的論文委員會成員。我們系大多數的博士生不用計劃這麼多,因為他們做的都是教授批準的項目。他們不用為其他兩個論文委員會成員是誰而感到壓力,因為他們的導師會為他們的畢業擔保,而其他委員會成員通常也是會同意的。然而,我的情況卻很特殊,因為我並沒有做Dawson有熱情的項目,所以我也不可能指望他能由衷地認可我的工作。有Jeff在我的委員會裏能起到一點幫助,畢竟他個人也參與到了我們ProWrangler的項目中,並且可以為它的合格性負責。但我仍然還需要一名同意我畢業的委員會成員。

我最先給我之前在MSR的經理Tom發了郵件,向他提出了我的想法,希望在2011年秋季到MSR實習幾個月,做一個能夠放到我的博士論文裏去的新項目。我希望他加入我的論文委員會,這樣我就可以把2009年暑假實習時和他發表的三篇論文也放到我的博士論文裏去。可惜他似乎對我的這個想法不太感興趣,我也就沒有再進一步勸他了。

之後我又想了想將SlopPy從一篇研討會論文拓展成一篇完整的會議論文的可能性,這樣它就“算得上”是我博士論文裏的一個實質性貢獻了。之前在我第五年的時候,我和MIT的教授Martin聊過一次,正是Martin有影響力的論文直接啟發了SlopPy這個創意,我和他談了談關於一起合作把SlopPy拓展成一篇會議論文的事。他對這個合作很感興趣,但是當時我們並沒有安排出時間表,因為我在這一年的末尾一直忙於CDE和ProWrangler的工作。不過現在,我發現如果我能利用第六年開始的幾個月時間和Martin合作,並讓他加入我的論文委員會的話,這也許可以成為我畢業的通行證。Dawson也喜歡這個計劃,因為(意料中的)他也曾經想過如何把Klee之類的想法和SlopPy結合起來。我準備在暑假中期給Martin發郵件提出這個合作建議,不過在此之前,另一個更好的機會出現了,於是我也沒再繼續這個計劃。

當我在谷歌開始我最後一次暑期實習時,我期望我可以在這無憂無慮的三個月裏完善CDE,但我還是感到了一絲擔憂,因為回到學校以後我的畢業問題還是沒有保證。我需要再有一個靈感的爆發,而這個靈感最後來源於一個意想不到的地方。

\breakline

2011年夏天,我最終決定了畢業後就從學術界“退休”:我並不知道我會選什麼來作為自己的職業,但我並不打算在接下來這一年申請終生制的大學教職工作。

我做出這個決定有兩個原因:首先,我感覺我當前的論文發表記錄還不夠令人印象深刻來讓我獲得一份不錯的終生制教職工作。幾個月之後當我悲傷地看到幾個發表記錄比我更勝一籌的同學竟然都沒有獲得任何工作時,我的這種感覺更為堅定了。當然,我還是可以試著做幾年博士後(博士之後的臨時研究員),發更多的文章,然後再次申請教職工作。

但第二個、也是更為重要的一個原因讓做博士後也對我毫無意義:我特別有熱情的那幾類研究主題並不那麼容易獲得研究基金,因為它們不太被現在的學術機構所認可。沒有經費,就不可能給學生付工資。沒有上進的學生來在艱苦的苦力勞動中研磨,就不可能獲得被認可的論文。而每年沒有足夠量的發表數量,就不可能獲得終生教職。就算是我獲得了終身教職,我還是需要新的經費來給新的學生發工資,來讓他們幫助我實現出我的想法;申請經費的這個循環永無止境。考慮到我的研究興趣,我內心並沒有準備好在這些硬戰中作鬥爭,來讓經費機構認真考慮我的提案。為了說服同行審稿人接收我的論文,我已經度過了足夠困難的一段時間;而經費審查人會更嚴苛,因為他們掌管著幾百萬美元,也更願意把錢撥給那些做計算機科學科研中主流研究的同行們。

幾年來我一直在思考離開學術界,而現在我覺得我有了合理的原因來這樣做:我領教了計算機科學中的“學術遊戲”是怎麼玩的,也知道我並不想再繼續玩下去了。在一封發給我的一位即將履新的助理教授朋友的郵件中,我總結了一下我的感受:“通過過去的五年,我發現我更願意做一個科研的旁觀者,作為一個新研究的持續不斷的生產者的負擔對我來說實在是太重了。”

因為我的母親是一位極其成功的教授,我的父親也非常認同學術界,所以告訴他們我的這個決定還是有些困難的。我覺得他們不能真正理解我的想法;我怕他們覺得我是在現實中放棄自己,低估自己;我不把當一個教授作為一個實際目標也有幾年了。經常說到的學術界一大好處就是創造性自由的誘惑,但我離開學術界的決定卻實際上也在博士最後一年和找工作時解放了我的思想,讓我能在追尋我真正的職業熱情時更富有創造性。

\breakline

決定離開學術界後,隨之而來的一個影響就是,在夏季我要出席做報告介紹IncPy、SlopPy和CDE的三個學術會議上,我不用再憂愁該如何去“交際”。學術會議上高年級博士生、博士後,還有即將獲得終身教職的教授都會爭相攀談,以此來給資歷深的同事們留下一個好印象。對這些資歷較淺的研究者來說,會議上的職業交際也是一個需要認真對待的全職工作,因為他們初期的職業生涯和學術聲譽都依賴於在這方面的良好表現。但由於我已經從這種學術遊戲中脫身,我也就不用再在意這些事,可以盡情享受自我,不用再緊張和圓滑。

在其中一個會議的中間休息時,我發現Margo獨自坐在那裏,在她的筆記本電腦上做些什麼。讀者可能還記得,我博士第四年的時候去聖何塞的研討會做報告,介紹我最初的IncPy 論文時,第一次遇見了Margo。我內心辯論了一下要不要過去和她打個招呼,並重新自我介紹一下。我內心有些擔心她已經不記得我是誰了,況且我也沒有什麼有意思的東西告訴她。但由於我馬上就要從學術界“退休”了,我也沒有什麼閑談的議程,所以就算這個對話最後搞砸了,我也沒有什麼損失。於是我走了過去打了個招呼。我提醒了她一下我們之前是如何見過面的,她也似乎還記得我。我簡要地告訴她我馬上就要去做一個報告介紹我新的CDE項目,然後就要趕一班飛機飛回加利福利亞。我們簡短地花五分鐘聊了聊CDE,然後我就跑去做我的報告去了。當天晚上到家之後,我給她發了封後續郵件,裏面附了一個\href{http://www.pgbovine.net/cde.html}{CDE項目網頁}的鏈接,因為想到可能她的學生會有興趣在科研中使用CDE。這只是我向同行們推薦CDE時常規的禮貌性郵件,因此我其實並沒期待她的後續關註。

兩周之後,我驚喜地收到了一封來自Margo的郵件,她說她和她的學生Elaine聊了一下我。他們兩個都希望我博士畢業以後到哈佛大學去,以博士後身份和他們一起工作幾年。圍繞IncPy和CDE項目的寬泛研究主題,正好與Margo創建幫助計算型研究者更高產的工具的興趣相一致。我很高興她給我提供這個職位,但這個機會對我來說沒有多大意義,因為我已經決定要離開學術界。再做一個博士後對我來說也沒什麼用,因為做博士後的主要目的一般都是提升個人簡歷,以此增加獲得大學教職工作的幾率的。

這時靈感又來了。由於我畢業前正迫切需要一個更實質性的項目和一個論文委員會成員,我向Margo提出了這個反轉提議:我詢問我是否可以在2011年秋季訪問哈佛四個月,與她一起在一個項目上開展工作,而不是博士畢業後去做博士後。我和她可以在2012年1月提交一篇論文到一個會議,並把這個項目作為我博士論文的最後一個組成部分。我還問了她是否可以擔任我論文委員會的最後一個成員。Margo很喜歡這個想法,但她已經沒有足夠的經費來支持我了,因為她還需要用經費支持她自己的學生。我和Dawson聊了下,他很大度地願意出經費支持我這幾個月,即使我沒有在Klee上開展工作(我的外源獎學金那時已經到期了)。Margo很高興地同意了這個安排,於是在2011年9月我結束暑期實習之後,我就去了馬塞諸塞州的波士頓,開始了我研究生的第六年、同時也是最後一年的生活。

如果我沒有積極地抓住幸運地降臨在我身上的這些機會的話,研究生最後的這段征途也不會變成可能。如果兩年前Robert沒有告訴我聖何塞的那個研討會,如果我沒有投稿並做報告介紹了IncPy的論文,如果Margo不喜歡我的論文並且也沒有把我介紹給Elaine,如果我沒有一直和Elaine保持聯系,如果我在上個夏季介紹CDE的會議上沒有主動和Margo再一次打招呼,如果她沒有給我發一封親切的後續郵件,而我也沒有冒險向她提出不尋常的反轉提議的話,那麼我現在還在斯坦福,掙紮著四處尋找最後一個項目和論文委員會的成員。

\breakline

作為一個訪問研究員,我在波士頓度過了非常快樂和高產的四個月。景色的改變很提神:我可以高強度地專註於科研,回到家後也沒有那些通常的生活雜事。Elaine幫我找到了一處絕佳的一居室公寓住宅,距離我的辦公室只有五分鐘的步行距離,我還可以很容易地在校園裏或者哈佛廣場買到食物。這個完美的起居安排讓我得以全神貫註於我的工作,不受任何的打擾。

我的第一個月基本都用來和老同學老朋友們社交了,因為我的母校MIT就在哈佛附近。我還和Margo會面了幾次,討論了可能的研究想法。她對我在她寬松的指導氛圍下做自己的項目持開放態度,所以我也得到了幾乎全部的思考自由。然而,我也在頭腦風暴中采取了一個務實的方式,因為我想讓她對我的項目感到興奮,並強有力地支持我把它放到我的博士論文中去。因此,我也讀了她最近的一些論文和資金申請書來了解她的科研理念,從而我可以讓我的想法去迎合她的品味。這時,我已經懂得了保持與老資歷的合作者(以及論文審稿人)的主觀偏好相一致的重要性,即使我是在一個所謂客觀的技術領域內做科研。

想了幾個創意之後,我提出了個Margo喜歡的東西:一個用來監視研究人員基於計算機活動並幫助他們組織實驗和記錄實驗筆記的工具。這是傳統型電子實驗室筆記本的一個創新性轉折。Margo開玩笑地建議到把“BurritoBook”作為臨時的研發代號來描述我們計劃的這個工具,因為它無縫地覆蓋到了對用戶常規工作流的活動監視中的很多層\footnote{``Burrito''意為墨西哥卷餅,是一種將肉、豆、生菜、碎奶酪等食品和各種醬料放在墨西哥薄餅上,並把餅卷起來完全包緊的食物。Margo取名的含義應該是比喻這個項目涵蓋了用戶工作流中很多不同的方面,正像卷餅中包含了各種食材一樣。——譯者註}。後來Elaine把它簡稱為“Burrito”,我也漸漸習慣這個簡稱,最終它也成為了官方的項目名稱。

當時,我還以為我Burrito的創意自發地來源於對自己的想法和Margo的偏好這二者的結合,但當我回顧了我之前的一些筆記後,我意識到類似的創意其實已經在我腦裏醞釀幾年了。早在研究生二年級,當我想監控人們是如何編程時,我就最初想到了類似Burrito的創意,後來在第五年一開始我想拓展IncPy來記錄基於Python的編程歷史時,我又有了更為具體的想法。我的研究生從始至終,我都一直以多種專門的形式記著研究實驗的筆記,來記錄搭建原型工具和運行試驗的過程,因此我個人也體會到了和筆記記錄相關的低效問題帶來的痛苦。最後,雖然我並不是一個HCI(Human-Computer Interaction,人機交互)研究者,但我第二年與Scott和Joel、第五年與Jeff工作時的人機交互訓練給了我對於用戶需求的敏銳敏感性,這對我對Burrito的設計也影響深遠。

我花了幾個星期擬出了Burrito的高層面計劃,並與Margo討論了下初期的設計細節。很多對初始想法的改進來自於我對計算型研究者工作時的觀察,和對他們管理各種各樣的實驗筆記、代碼,以及數據文件時所面臨挑戰的采訪;我大多數的觀察樣本都是Elaine在MIT和哈佛的多個科學實驗室的朋友。在Rob領導的實驗室組會上做了一個關於Burrito 方案的報告之後,我還獲得了一些有用的早期反饋,這位Rob就是在我博士二年級剛開始時鼓勵我去和Scott與Joel工作,追尋我的人機交互興趣的那個MIT教授。

\breakline

社交時間結束之後,就是研磨的時間了。在2011年11月初,我在研究生階段最後一次轉變成編程的野獸,來把我Burrito的想法實現成一個能工作的原型工具。我連續72天進行了編程工作,在這兩個半月中只間歇性地休息了5天。這是迄今為止我所承受的最長的、幾乎是痛苦級別的、不停歇的、高強度工作。我第五年時初始時CDE的爆發式工作也才是21天的研磨而已,而這次卻幾乎是那次的三倍。我工作一直貫穿了感恩節、聖誕節、新年夜,不顧一切地專註於讓Burrito工作得足夠好的目標,從而能在2012年1月之前提交一篇會議論文。

這幾個月裏,我變成了一個反社會、壞脾氣的人,躲避一切紛擾,深深地沈浸在自己的工作中。我所思考的一切就只有計算機代碼;除了每周與Margo會面報告進展時,我幾乎不說連貫的英文句子。雖然我的外表和動作都像一個準人類(即,不修邊幅、衣冠不整),我的內心狀態卻是欣喜若狂的。我每天都要編程和調試十余個小時,但我的腦子還是很放松的,因為我的技術技藝應付所面對的挑戰已經是遊刃有余了。這時我已經積累了足夠多的經驗來設計、實現和“營銷”研究型原型工具了,我也有信心我的能力能讓這個項目成功工作。一路上我都獲得了Margo的極好反饋和支持,因此我覺得她會強烈支持我把Burrito放到我的博士論文中去。經歷了研究生早期在不確定並且失敗的項目上數年的研磨之後,我現在感到精力充沛,並朝一個我知道能夠實現的目標緊張地工作著。

2012年1月中旬前,Burrito原型工具已經具有了個良好的狀態,於是我進行了一個非正式的評估,撰寫了一篇論文,並按計劃將其投稿到了一個會議。我休息了幾天,重回正常人的狀態,與波士頓的朋友們告別,然後飛回到加利福利亞,迎接我博士生涯的尾聲。

\breakline

關於一篇博士論文是如何形成的普遍觀念是,一個學生靈光一現,受到啟發,提出了某個有洞察力的聰明的想法,然後就花幾年時間,邊啜飲上百杯的拿鐵和卡布奇諾,邊撰寫鴻篇巨著。在眾多科學和工程領域,這種看法是完全不準確的:“撰寫”工作僅僅是把一個人已發表的論文合輯成一個單獨的文檔,並填充一些引導性和總結性的章節內容。在一個學生坐下來“寫”畢業論文之前,所有的揮灑汗水的辛苦勞動其實都已經做完了。

在我們系,一個博士生生涯最重要的裏程碑就是他們的導師向他們\emph{豎起拇指}、讓他們開始撰寫論文的那一刻。這個手勢代表這個學生已經做了足夠的工作——通常是發表了主題一致的兩到四篇會議論文——並且有資格在幾個月內畢業了。

當我2012年1月回到斯坦福時,我的目標就是盡快獲得Dawson那至關重要的\emph{豎起的拇指}。我寫了一個短文檔,列舉了一些證據表明為什麼我覺得我做的工作夠得上畢業了。我的論據很簡單:我創建了五個新穎的軟件工具來改善計算型研究程序員的工作流——IncPy、SlopPy、 CDE、 ProWrangler和Burrito——並從我的身體力行的工作中發表了1 篇頂級會議論文,3 篇二級會議論文,以及3篇研討會論文(Burrito的那篇會議論文投稿最終被拒收了,所以我們重新投稿並發表在了\href{http://www.pgbovine.net/projects/pubs/guo_burrito_tapp_2012.pdf}{一個研討會上})。還有一個額外的收獲就是,我的另外兩個論文委員會成員,Jeff和Margo,都會認可我的畢業,因為我和他們一起成功地完成了項目(分別為ProWrangler和Burrito)。我把這個文檔通過郵件發給了Dawson,然後焦急地等待這他的回復。我覺得我的理由還算挺充分的,但我還是不確定他會不會希望我再多做一些工作才同意我畢業。讓我大松一口氣的是,他很快就向我\emph{豎起拇指},也是這時我知道我已經實質上完成研究生的學業了。

我花了接下來兩個月的時間把我的論文合並成一個230頁的題為\emph{促進研究型編程的軟件工具}的文檔。以下是我博士論文第一頁的摘要(總結):
\begin{quote}
  研究型編程是一種編寫程序,以期對數據進行深入理解為目標的編程活動。上百萬名科學、工程、商業、金融、公共政策以及新聞業的從業人員,還有眾多的學生和計算機愛好者,都在日常的工作中進行著研究型編程。

  我的論文即是,理解了研究型編程中所面臨的獨特挑戰後,就有可能應用動態程序分析,混合主導推薦系統,和操作系統級別的追蹤技術來使研究型編程更高產。

  本論文刻畫了研究性編程的過程,描述了研究型編程所面臨的典型挑戰,並提出了五個我所開發來解決一些關鍵挑戰的軟件工具。1)ProWrangler是一個交互式的圖形工具,可以幫助研究型程序員在分析前對數據進行格式重置和清洗。2)IncPy是一個Python解釋器,可用來加速數據分析腳本循環,並幫助程序員管理代碼和數據的依賴。3)SlopPy是一個Python腳本,可以自動地使已有腳本容錯,從而也使數據分析腳本循環提速。4)Burrito是一個基於Linux的系統,可以幫助程序員組織、標註和回顧之前實驗中的洞察力。5)CDE是一個軟件打包工具,它使得部署、存檔和代碼共享更加容易。綜上所述,這五個工具使得研究型程序員能夠通過減輕計算機數據管理和數據起源方面的負擔,來更快地叠代和發現可能的洞察力。
\end{quote}
我花了很多精力來增添一些導入性和總結性的章節,讓我的論文不僅僅是五個我前幾年搭建的單獨工具的說明。在我的撰寫過程中,Jeff和Margo都給了我很好的反饋,告訴我應該如何以一個更實質性的知識角度來搭建敘述我研究貢獻的框架。盡管我知道最終不會有什麼人來讀我的博士論文——構成博士論文的那些單獨的論文共容易獲得——我還是很滿意能將我所有的創意、洞察力、工具說明和評估結果聚集在一起,形成\href{http://www.pgbovine.net/projects/pubs/guo_phd_dissertation.pdf}{一篇緊密結合在一起的文檔}。

\breakline

我把我的口頭答辯安排在了2012年4月23日。最大的挑戰是找到一個五位教授(我的三個論文委員會成員和另外兩個答辯委員會成員)都有時間的兩小時。Margo那一周正好來加利福利亞參加一個會議,所以我圍繞她的日程來安排答辯。在我們系,口頭答辯的形式是學生先做一個小時的公開報告,總結他的畢業論文研究,接著的一個小時是私下會議環節,由委員會詢問測試性的問題。之後,委員會投票決定是否通過這個學生的畢業論文。實際上,幾乎沒有人的答辯會不通過,除非他的表現實在是太差了:委員會通常會先通讀一遍學生的畢業論文,同意之後才讓會學生參加答辯,所以這應該不會有什麼意外。

在口頭答辯上我沒有足夠的時間介紹全部五個項目,因此我選擇介紹三個項目,其中每一個都是和論文委員會中的一個成員合作的:與Dawson合作的IncPy,與Jeff合作的ProWrangler,還有與Margo合作的Burrito。大多數博士生都只和他們的導師合作發表文章,所以能夠展示和所有三位委員會成員合作的研究是一份難得的榮幸。我也很高興很多朋友和之前的同學——包括Scott、Joel、Peter、Robert、Greg和Fernando——都來出席了我的答辯。

盡管我研究生期間已經做過了十幾個學術報告,我答辯時還是比以往更緊張,因為我認識聽眾裏的每一個人。很奇怪,當一個房間裏都是陌生人而不是熟悉的面孔時,我反而感覺更輕松一些。私下會議環節沒有我預想中的那麼累人,但我的委員會確實提出了一些最終幫助我改善了畢業論文的問題和建議。

我通過了答辯以後,我的委員會成員和朋友們都很有禮節地以一個親切也是預想中的方式向我表示了祝賀。我最為珍視道賀來自於一個我只和他說過一次話的資深教授。他出現在我的答辯會上時我還有一絲驚訝,因為我覺得他不會對這個主題感興趣。答辯之後,他發郵件贊揚了我的報告:“我想說我非常喜歡你的答辯,一部分是因為這個工作的創造性,一部分是因為這是一個準備充分的報告,還有一部分是因為我前一年都在做研究型編程工作。”

\breakline

從科研的角度來看,在我那一年的26個斯坦福計算機系的博士畢業生中,我覺得我是極其平庸的一個,因為我大多數的論文都只是二級會議,也沒有很好地被學術圈所認可。我的博士論文也是尷尬地橫跨了幾個計算機子領域——編程語言、人機交互和操作系統——因此我的工作也沒有被其中任一領域的頂級學者認真對待。

盡管沒有得到主流的認可,我還是覺得我的博士生涯畫上了圓滿的句號,因為我成功地讓我的創意結出了果實,並且以一個我十分引以為豪的博士論文成功畢業。我的博士采取的是一個高度創業型的路線——伺機地尋找項目和合作者,在非傳統和遵守慣例二者中間又辟出了一條路使得我的論文得以發表。我感到非常幸運能夠以一種富有創造性的方式來安排我的博士生涯;如果沒有在斯坦福六年中前五年的外源獎學金支持,我不會有如此多的自由。

最終,我的博士論文也像大多數博士論文那樣,將人類知識的邊界向外拓寬了非常非常小的一點。我做的這五個原型工具包含了一些有意思的想法,可以被將來的研究者們加以改進。其實,如果將來的研究人員能夠引用我的論文作為粗糙的初始研究的例子,並以此說明為什麼他們的技術更高端,那麼我將非常榮幸。這也是科學研究是如何一步一步向前推動的:後一代總是在前一代想法的基礎上建造自己的工作。

然而對我來說,我論文最重要的貢獻不是那五個特殊的原型工具,而是我發現了一個普遍的問題——迎合大量且日益增長的計算型研究程序員的需求的軟件工具非常缺乏——並提出了一些人們可以用來提升生產力的早期原型解決方案,據我所知,我是第一批這樣做的計算機科學專業的博士生之一。我相信這些思路在接下來的十年會變得尤為重要,但由於我要從學術界退休了,我也就不再致力於直接推銷它們了。

因為我的博士論文主題離成為主流還有很長一段距離,所以想以它裏面的想法為基礎,繼續發展自己學術生涯的初級教授或者科學家需要奮力拼搏,來獲得資金會機構的認可,因為這些人是新項目立項的把關人,還需要獲得他們老資歷同行的認可,因為這些人是論文發表和終身制審核的把關人。如果有任何人想投入到這個榮耀的戰役中去,我很願意幫助他,但我卻不夠勇敢把我自己的生涯放到這上面去。相反,我現在計劃去追求一個完全不同的職業熱情,可能某一天它會成為我另一本書的主題:-)

\breakline

在準備撰寫本書的過程中,我翻閱了很多原來的研究筆記。有一天,我發現了如下這個關於一個我有興趣深入研究的話題的小片段:
\begin{quote}
  研究一下開發一些工具來提供給非軟件工程師,或者說是工作中需要編程的科學家、工程師和研究人員——他們不會在意參數、模型檢驗等等——他們只需要實用、輕量級、並且概念上簡單的工具,能夠隨時就能拿來用。
\end{quote}
讓我吃驚的是這個筆記是我六年前寫的,也就是2006年夏天我即將要開始我在斯坦福的博士生項目的時候。這是一段漫長、迂回、而且不可預料的旅程,但我還是由衷地高興我能夠把這一寬泛的話題——這幾年來吸引我的若幹話題中的一個——變成了我的博士畢業論文。如果沒有罕見的各種好運組合在一起,沒有我個人的主動性,沒有來自慷慨的朋友們那些富有洞察力的推動,以及接近一萬個小時的研磨,這個成就不可能變為現實。

%%% Epilogue
\mychapter{結語}
\emph{如果你不打算做一個教授,那為什麼還要受累讀一個博士呢?}這個常見問題非常重要,因為大多數博士生都不能獲得一份像他們大學裏的指導者以及行為榜樣——終身制教授—— 一樣的工作。空缺的教職職位並不多,因此大多數博士生是在為一份他們永遠不會得到的工作而訓練。(想象一下,如果醫學院或者法學院的畢業生不能分別得到醫生和律師的工作,那該多令人惶恐。)

那麼為什麼會有人在不打算做教授的情況下,還花六年甚至更多的時間來攻讀一個博士學位呢?每個人的動機不盡相同,但一個可能的答案是,博士生項目給某種類型的人提供了一個安全的環境,驅使他們超越他們精神上的極限,最終變得更強。比如,我六年的博士訓練讓我比起還是一個研究生新生時,變得更加智慧、有悟性、堅韌、剛毅、專註、有創造力、有口才、感覺敏銳,並且專業地高效。(兩個明顯的警示:不是每一個博士生都能如此受益——很多人也因掙紮而變得厭倦、疲憊不堪。當然,也有很多人在沒有讀博士的情況下,同樣培養出了這些積極的品質。)

有個不太恰當的類比:\emph{為什麼有人會花數年的時間訓練,來在類似鐵人三項這樣的比賽中超過他人——鐵人三項是一項極度折磨人的比賽,包括2.4英裏的遊泳,112英裏的自行車騎行,以及26.2英裏的長跑——而卻不打算成為一個職業的運動員?}簡而言之,這種經驗驅使人們突破他們身體上的極限,最終變得更強。從某種意義上來說,讀博士就是智力上的高強度體育訓練。

\breakline

下面是我博士生涯中學到了最值得記住的二十條經驗。我分享出來的原因不是為了主動給學生們提供一些建議,畢竟每個博士生的經歷都有很大的不同;也不是因為要鼓勵大家去讀博士,畢竟這些經驗可以來自多種地方。這個章節其實是作為我讀博士過程中收獲的總結。
\begin{enumerate}
  \item \textbf{結果勝過目的:}如果一個人做出了很好的結果,那麼沒有人會去問他一開始的目的是什麼。我研究生階段從來沒有過所謂的純粹知識上的動機:我開始讀博士是因為我對工業界工作的不滿意,給自己施壓來想出項目是出於對不能按時畢業的憂懼,幫助Scott、Joel以及Jeff做人機交互的項目是為了對沖我賭博的風險。但我最終成功了,因為我做出了結果:五個原型工具和若幹篇論文。在這個過程中,我培養起了我對項目的熱情,並為我的項目感到自豪。相反,我知道一些有著最理想主義目的的學生—— 夢想著並熱切地希望徹底變革他們的領域——但他們沒做出什麼結果,最終他們的幻想也破滅了。
  \item \textbf{輸出勝過輸入:}獲得博士學位的唯一途徑就是成功地產生科研輸出(例如,發表的論文),而不僅僅是靠上課或者讀他人的論文這些吸收來輸入。當然,一個人能產出成果之前的吸收也是有必要的,但也很容易過分吸收。我第一年時就曾掉入過這種陷阱,我漫無目的地讀了幾百篇論文——一種狂歡式的吸收——卻未從這一沒有方向的閱讀中綜合出任何有用的東西。相反,對畢業論文項目的相關工作文獻的搜索則有效得多了,因為我的閱讀是緊緊朝著一個清晰的目標的:找出競爭者,以及可以修改加到我自己的項目中的好創意。
  \item \textbf{找到相關的信息:}我的博士訓練教給了我如何在每一個時刻,高效地找到我所需要的最相關的信息。與傳統的課堂學習不同,當我做科研的時候,我並沒有教材,沒有課堂筆記,沒有授課教師提供給我的標準答案。有時候我工作所需要的東西會在一篇研究論文裏,有時候會在一段陳舊的計算機代碼裏,有時候會在一個隱蔽的網站上,有時候又會在某個需要我找到並尋求幫助的人的大腦裏。
  \item \textbf{創造幸運的機會:}整個研究生階段,我獲得了好幾次不可思議的好運,最後一年我得以到哈佛和Margo一起工作時,是我好運的頂點。但如果我沒有反復地把我自己和我的工作展示給他人的話——作報告,和同行們交流,尋求並提供幫助,表達謝意,這些幸運的機會也並不會出現。其實我的大多數努力最終都沒有變成意外的收獲,但如果我沒有持續嘗試,那麼我可能也不會這麼幸運。
  \item \textbf{博弈遊戲:}作為一個博士生,我處於社會等級的底層,沒有能力去改變這個“學術遊戲”。具體而言,盡管我懼怕我的論文一次又一次地被拒,我也沒有其他辦法,只能繼續學著怎麼盡自己所能去玩發表論文的遊戲。然而,我很高興我在我研究生的後半期,用自己獨特且有創造力的方式來玩這個遊戲,從事著很多非傳統的項目,與此同時也很好地遵守著發表和畢業的“規則”。
  \item \textbf{從底層做起:}在了解了年長博士生、教授以及其他老資歷同事們的動機和個性之後,我也能夠把我自己的研究動機提上他們的日程了。例如,當我通過閱讀Margo的論文和資金申請書,知道了她的研究品味之後,我想出了一個我和她都很有熱情的項目創意(Burrito)。如果我對她的興趣一無所知,那麼想出一個她也喜歡的創意就會困難得多了。
  \item \textbf{教授也是人:}盡管這一點聽起來十分顯然,但大家卻很容易忘記教授並不是無情的科研生產機器這一點。他們也是正常人類,有著自己的品味、偏好、興趣、動機、不足和恐懼。即使是備受尊敬的科學智者也有他們主觀的、不理性的怪癖。從一個學生的角度來說,由於教授是學生發表論文、畢業和未來工作的把關人,因此對教授既是專家也是人類這一點心領神會很重要。
  \item \textbf{爭取受人歡迎:}當我和喜歡我的人一起工作時,我更高興也更高產。當然,讓所有同事都喜歡自己是不可能的,因為大家的個性不可避免的有著差異。總的來說,我努力地找出天性就和我相處融洽的那些人,然後再花時間來培養這份關系。
  \item \textbf{盡一些義務:}作為一個資歷較淺的實驗室成員,盡一些義務、做一個“好士兵”是很有必要的,而不是從第一天就開始提一些蠻橫的要求。當還是MIT的本科生和碩士生的時候,我就履行了我的義務,我在一個教授同意的、有資金支持的項目上工作了兩年半,而不是嘗試創建自己的項目;我得到了滿意的回報,被排名最高的幾個博士生項目錄取,還獲得了兩個外源獎學金,可以用來支付我五年的研究生費用。後來,當我剛來到斯坦福時,我也履行了義務,在Klee 項目上工作了相當長的一段時間,直到最後退出。我也花了幾年時間來弄清楚什麼時候要聽從權威人士安排,什麼時候可以稍稍自私地推進自己的議程。
  \item \textbf{拒絕不好的默認安排:}默認安排通常沒有為底層人員(例如博士學生)的最大利益考慮,所以知道什麼時候拒絕,並請求換一個不同的安排是很重要的。當然,並不會有針對學生的邪惡陰謀;默認的安排只是自然地設定下來使當權者收益。舉個例子,像Dawson這樣的知名終身教授,很輕松就可以獲得為期數年的資金來支持學生,讓他們在類似Klee的“默認”項目上開展工作。只要不時的有論文發表出來,那麼這個教授和他的項目都會被認為是成功的,而不會有人在意有多少學生在路途中摔倒和失敗。學生們必須自我判斷他們被默認安排的項目是否有前景,如果沒有的話,就需要想辦法以一種合適的方式退出這個默認項目。
  \item \textbf{知道什麼時候該退出:}在第三年結束時退出Klee項目是我研究生階段最關鍵的一個決定。如果我沒有退出Klee,那麼就不會有IncPy,不會有SlopPy,不會有CDE,不會有ProWrangler,也不會有Burrito;有的只會是三年或者更長時間的痛苦的增量式進展,以及隨之而來的一個“勉強的畢業”。
  \item \textbf{從失敗中站起來:}失敗在研究生階段是不可避免的。我前三年所做的東西都沒有被放到我的畢業論文中去,並且我後三年嘗試的很多方向也都是死路一條。研究生院是一個鍛煉從失敗中重新站起的安全環境,因為比起實際工作中的失敗,研究生階段失敗帶來的風險是比較低的。在博士階段的前幾年,我經常因科研失敗而憂慮、心煩意亂、失去勇氣。但當我變得成熟後,我學會了化悲憤為力量,也就是我所說的\emph{想要做出結果的狂怒}。每一次被拒、懷疑和批評都鞭策我更加努力地工作來證明那些否定我的人是錯誤的。研究生頭幾年的失敗是我後幾年的成功之母。例如,第二年時我沒能在一旁觀察專業程序員的經歷教會我應該向誰、且如何尋求這方面的幫助,所以我後來就成功地觀察了計算型研究者的工作,這也對我的博士論文工作有所啟發;還有,我沒能獲得大批用戶來使用IncPy的失敗經歷也教會我應該如何更好地設計和推薦自己的軟件,因此我才獲得了10000個CDE的用戶。
  \item \textbf{與圈內人結盟:}當我與專業的圈內人合作時,我發表論文變得更為容易了,比如我第二年時與Scott和Joel的合作,在MSR實習時與Tom的合作,還有第五年時與Jeff的合作。他們知道在各自的子領域發表論文所要求的各種交易技巧;我和這幾位圈內人共同撰寫的五篇論文全都是第一次投稿就被接收了。然而,作為一個圈外人奮力拼搏——第二年與Dawson的經驗性軟件衡量研究,和之後我獨自的畢業論文項目——也是很有收獲的,盡管也更令人沮喪,因為論文總是一次又一次地被拒收。
  \item \textbf{多做報告:}在整個博士階段中,我做了二十余個科研報告,從在大學實驗室組會上的非正式報告,到大型酒店舞會廳的會議報告。例如我在IncPy 的項目初期的非正式報告,就有助於我獲得設計靈感和意見反饋;我在提交論文前做的報告,有助於我發現需要在論文中修改的公認的缺陷。此外,每次報告也是個絕佳鍛煉機會,有助於我提升我的公共演講技巧,和應對偶爾帶有敵意的提問的技巧。最後一點,作報告有時候還能點燃一些後續討論的火花,進一步獲得一些意外的收獲:比如,聽了我關於IncPy的第一個報告之後,一個研究生同學通過郵件發給我了一個Fernando一篇有關科學中的Python的博文鏈接;那封郵件鼓勵我去接觸一下Fernando,也正是Fernando後來啟發了我改進IncPy,然後又創造出了CDE。過了一年之後,我介紹CDE的谷歌科技報告直接給我帶來了非常棒的2011年暑期實習機會。
  \item \textbf{推銷,推銷,再推銷:}我把大部分研究生生活的時間都花在了埋頭苦幹、辛苦研磨、實現出研究創意上,但我意識到令人信服地將自己的工作推銷出去才是發表、獲得任何、順利畢業的關鍵。因為論文發表這一遊戲中有著白熱化競爭的天性,將接收和拒收區分開來的因素往往就是一篇論文的“營銷技巧”是如何吸引審稿人的品味的。因此,如果沒能將自己研究重要性的藍圖適當地推薦給目標聽眾——資歷深的學術界同行,那麼上千個小時的研磨就可能會付之東流。更為普遍的是,一個領域中有很多人都有好多想法,而更好的推銷者會更有可能讓他們的創意被現有圈子所接受。作為一個地位較低的學生,將我的想法“推銷”出去的一種最行之有效的方法,就是讓有影響力的人(比如像Margo 這樣的知名教授)對我的項目足夠有熱情,讓後讓他代表我去宣傳提升它們。
  \item \textbf{慷慨地提供幫助:}博士生經歷中我最喜歡的一個特征就是我並沒有一直處於和同學們的競爭中;並不是他們做得越好,我就會做得越差,反之亦然。因此,我們中的大多數都慷慨地互相幫助,尤其是在遭受外部審稿人嚴厲的批評之前,就互相給予關於想法和論文草稿的反饋。
  \item \textbf{尋求幫助:}在過去的六年裏,我變得善於決定何時、找誰、以及如何來尋求幫助。具體地說,每當感覺受困時,我就會去尋找能夠幫我脫離困境的專家。尋求幫助可能只需要簡單地找一個系裏的朋友,也可能需要有人介紹,或者甚至需要發冷郵件給陌生人。
  \item \textbf{表達誠摯的感謝:}這些年來,我學會了向給予過我幫助的人表達感謝。盡管獲得博士學位幾乎是一個人的戰鬥,但沒有其他十多個同學的慷慨幫助,我也不能完成這一任務。人們發現他們的建議或者反饋帶來了具體的益處之後會感到很高興,因此只要可能,我都會努力答謝每一個人具體的貢獻。甚至一個簡短的感謝郵件也會很有幫助。
  \item \textbf{想法喚起想法:}正如我第一年結束時發現的那樣,憑空想出一個實質性的想法幾乎是不可能的。想法總是構建在其他想法之上,所以找到一個堅實的起點很重要。比如,IncPy和SlopPy的動機都來自於我2009年在MSR實習時所碰到的惱人的和編程相關的低效問題。一年之後,我的一部分拓展IncPy的想法,再加上Fernando對可重復性研究的洞察力,還有Dawson對Linux依賴地獄的點撥,一起促成了CDE的創建。當然,創意有時候也需要數年時間才會開花,通常是在經過了幾次錯誤的開端之後:我在第二年期間和第四年末尾時都考慮過類似Burrito的想法,但直到第六年我才得以把這些模糊的想法固化起來,形成一個真正的項目。
  \item \textbf{辛苦且聰明地研磨:}這本書的書名叫\emph{\bookname},因為沒有哪個博士生未經過上萬小時單調乏味、講究實際的研磨。這段旅程讓我明白,如果沒有極端的努力來將創意變成現實,那麼富有創造力的想法也毫無意義:出現在辦公室,一屁股坐在位置上,開始辛苦地研磨,保持不大但是持續的進展,休息一下來反思和恢復精神,接著又一天天地循環往復,連續超過兩千天。然而,聰明地研磨和辛苦地研磨同等重要。看見有的學生盲目地拼命完成不會得到有利結果的任務真令人悲傷:從一個錯誤的角度切入一個科研問題,使用錯誤的工具,或者做著無用的差事。聰明地研磨需要有感知力、直覺和一種樂於尋求幫助的意願。
\end{enumerate}

\breakline

我回答一個含F開頭單詞的問題來結束這一章:\emph{讀博士有趣(fun)嗎?}

博士生經歷的某些方面非常有趣:提出新的想法很有趣;在白板上寫下軟件設計的草稿很有趣;和同事一起喝咖啡、探討想法很有趣;會議期間和有意思的人一起聊天很有趣;作報告並激起熱烈的討論很有趣;收到來自世界各地CDE用戶的熱情洋溢的郵件很有趣。但我可能六年中只花了幾百個小時在這些活動上,還占不到我全部工作時間的百分之五。

相比之下,我花了近萬個小時獨自在計算機前研磨——編程,調試,運行實驗,琢磨軟件工具,查找相關信息,以及撰寫、修改和重寫研究論文。任何經歷過開創性工作的人都知道日復一日的研磨基本沒什麼樂趣:它需要高度集中的註意力,嚴格的紀律性,對細節的敏銳註意力,高度的耐受性,還有對做出偉大工作的執著追求。

那麼,\emph{這有趣嗎?}

我用另一個含F開頭的單詞來回答:\emph{它有時候是有趣的,但更重要的是,它是充實的(fulfilling)。}有趣通常是瑣碎、短暫,並且易得的,但是充實卻只有在克服了巨大且有意義的挑戰後才會獲得。攻讀博士學位是我生命中最充實的體驗之一,我也感到很幸運有機會能在這期間變得富有創造力。

%%% Afterword
\mychapter{後記}
自發布的第一個月(2012年7月)開始,有上萬個人閱讀了\emph{\bookname},還有幾百個讀者給我發了有關本書的個人郵件。最吸引人的是,讀者們根據他們自己的生活體驗,對這本書有著不同的解讀。正如一句老話所說,\emph{仁者見仁,智者見智}。例如:
\begin{itemize}
  \item \textbf{本科生}知道了博士生活會包含什麼,這幫助他們調整了內心對博士生活的期待。一個學生通過郵件寫信給我:“我必須得說,盡管我是一個考慮明年申請博士的本科生,但你還是讓我明白了我接下來幾年的生活中會發生什麼。我待在實驗室實習做項目(其實我做的工作就是讀了你的書),以及在有關博士的一些漫想中變得迷茫,但這些時間都是值得的。”另一個讀者提到\emph{\bookname}是“一個讓普通人也可以深入了解攻讀博士這件事的優質資源$\cdots\cdots$這本書中導師的選擇、資金等等博弈,帶有些‘政治’意味遊戲的視角是很難得的。”
  \item \textbf{已經工作而沒有攻讀博士的人們}感謝我讓他們更加堅信,他們當初沒有選擇讀研究生是一個無比正確的決定。一位剛畢業、馬上就要成為工程師的讀者寫道:“當選擇我這一生要做什麼時,我只有兩個選項:成為一名教授,或者是成為一名專業工作者。考慮到一些綜合因素,我最終選擇了後者,但最重要的原因是,我覺得我不能犧牲掉我生活中日常交際和非學術的那一部分,來投入到科研所必須的奉獻上。我一直在想我的選擇是否正確。你的書讓我知道了要是讀博士我的生活會是怎樣一番景象,我想我當時做出了正確的決定。”
  \item \textbf{博士早期的學生}大都涉及我研究生頭幾年不得不經受的那些無聊且不需動腦的研磨工作。有些學生很振奮,因為他們發現如果在接下來幾年辛苦且聰明地研磨的話,他們的未來看起來還是很光明的。一個一年級的博士生告訴我:“你的意誌力真強。我剛剛經歷了一年的博士研磨,就覺得永無出頭之日了。你的工作真的讓我感到振奮,我現在感到非常有激情和決心。(至少在我讀了你書的那一天是這樣!)”一個二年級博士生給我說:“我在今天這一博士的關鍵階段發現了你的書。我的博士生資格考試\footnote{博士生資格考試:準備攻讀博士學位的學生通常需要在研究生的第一年或者第二年參加的資格考試,通過了這個考試後,才會被允許繼續攻讀博士學位。——譯者註}就要來了,但我覺得我的工作缺乏實質性,所以當我開始讀了你的書後,我準備扔掉我先前的工作。我不知道博士生活會是什麼樣,但我很喜歡它。這正是我需要的那種驅動力,哥們。”
  \item \textbf{博士中期的學生}對中間章節描述的那種“煉獄”般的精神狀態很有共鳴。一個讀者給我說:“我看見過許多其他領域所共有的事情。看到你描述的第三年時對項目的冷漠我不由失笑$\cdots\cdots$我花了更多時間才到達這個階段。我有資金支持,但我的導師並不怎麼管我,所以我有時候也和你一樣,四處尋找靈感,因為小幅的改進並不能解決根本問題。”
  \item \textbf{成功退出博士的學生}通過我研究生的整個經歷也間接地體驗了一番研究生生活,並從中獲得驗證,他們提前退出博士生涯的選擇是正確的。例如,一位第一年就離開博士學習的斯坦福計算機科學的博士生寫道:“感謝你讓我確信做博士時選擇退出是我做過的最好的決定。”
  \item \textbf{博士後期的學生}對我因學術界的論文發表遊戲而受挫很有共鳴,因為他們也一次又一次地面對過論文被拒,而被拒的原因似乎都是他們不可控制的。有個讀者說:“持續回歸到發表論文這一主題——或者更確切地說,迎合子領域中一小部分專家的口味來發表文章——真正切中了要害。”其他博士後期的學生希望想辦法為下一代的學生減輕一點這個過程中的痛苦:“有時候,我為你所經歷的一些事感到生氣,或者說沮喪,因為其中大多數都是不必要的,然而我也同樣經歷過這些事。我們可以探討一下,如何才能讓博士頭幾年的經歷變得不那麼令人沮喪,然後告訴將來的研究生,我覺得這會很有意思。”
  \item \textbf{教授們}很欣賞我故事中的努力拼搏和最終勝利的喜悅心情,因為他們在研究生階段都獲得了成功。斯坦福一個資歷較深的教授在Twitter上說:“我們班的大多數人(包括我)都想在第二年退出博士學習!讀博士很苦,但也很值得!”另一個年輕的教授寫到:“最終,博士學習對我來說是一個收獲頗豐的旅程,他幫助我成長了許多,我工作更加努力,更執著、強大和自信。這個轉變才是博士生項目的真正產出,而不是那一堆發表的論文。”最後,我還想分享選擇性閱讀本書的讀者裏最為直接的一個代表,一個教授高度贊揚了我的整本書,說他如何如何喜歡我的這本書,並且非常高興我最終還是決定去做一個大學教授!不過至少可以這樣說,她對最後幾個章節的理解是極其有創造性的。
  \item \textbf{普通讀者們}也很樂於看一看他們讀博士的朋友或家庭成員的生活可能是什麼樣的。一位母親寫道:“我不是一個博士生,但我卻對書中的很多事情產生了共鳴,因為我的女兒也是一個博士生,她現在已經是一個終身制系列的教授了。要是你早點寫出這本書來就好了!”還有一位同學告訴我:“我把這本書發給了我的奶奶,她很喜歡。我是我們家第一個獲得博士學位的人,所以她對這個過程有著很多問題,而你的回憶錄可以回答她的疑問,這樣我就不用再回答她了:)”
\end{itemize}

\breakline

我也收到了一些讀者的常問到的問題,我在這一小節裏做一個回答。

\emph{問:為什麼你一直執著於發表論文?我Z時期在Y大學讀了X專業,並且在沒有發表任何論文的情況下成功畢業了。}

我專註於發表論文這個話題,是因為努力發表論文幾乎占據了我還有我們系的同學們博士生活的大部分。2012年前後,斯坦福計算機系的學生需要在畢業前發表二到四篇論文。甚至我的導師也給我說了很多次,我至少得發表兩篇被認可的論文,他才會讓我畢業。讀者朋友們也告訴我,在其他領域,發表論文對博士生而言沒這麼重要;其實在過去,就算是計算機科學的博士生也沒有面臨這麼大的發表壓力。然而,對我和我的同學們而言,我們的感覺就是“要麼發論文,要麼走向滅亡”。這並不代表我贊同這種強烈地由論文驅動的研究風格(它有正面也有負面的影響),可它就是我獲得博士學位必須要玩的遊戲。

\emph{問:我是一個計算機科學教授,我想指出,你關於獲得教職工作和資金項目所需條件的觀點是有一些誤導性的。為什麼不提供更精確的信息呢?}

我完全承認我的觀點是短淺和片面的,因為我從來沒有在教師招聘委員會和資金審查委員會裏工作過。然而,我基於我六年博士生階段所看到的,把我認為這些過程是如何進行的最誠實的解釋寫了出來。

\emph{問:我是一名學生,需要一些建議。我應該攻讀博士學位嗎?我該退出我的博士項目嗎?我要不要換導師?我該不該試著當一個教授?}

\emph{\bookname}是一本描述我自己經歷的回憶錄,而不是給廣大學生的一般性建議。由於你所處的環境可能和我的有很大不同,我覺得我沒有資格給你提供建議。歡迎你以任何合理的方式來理解我的話,然後用你最好的判斷力來決定。祝你好運!

\emph{問:你本書的目的是對學術界的批判,或者是對學術界改革的呼喚嗎?}

絕對不是。除了盡可能誠實地講述我自己的故事而外,我別無其他用意。

\emph{你為什麼要離開學術界?你的科研視野和發表記錄也很優秀,所以你應該能獲得一份體面的教職工作,而且如果你足夠努力工作的話,還能獲得終身職位。}

這本回憶錄的目的是講述我博士階段的故事,而不是解釋我為什麼決定離開學術界。我不追求學術工作的完整原因已經超出了這本書的範圍。但是,如果非要我給出一個一句話的答案的話,那麼可以這樣說:\emph{我就是非常不想做教職工作。}

\emph{問:那麼你討厭學術界嗎?你覺得學術研究是不是一個無用的遊戲?}

不,恰恰相反,我很敬重學術界,而且深刻地理解它在推進創新的過程中所扮演的角色。盡管同行評議的發表規則是一個不甚完美的遊戲,我也沒有一個能夠控制質量的更好的解決方案。正如我在研究生最後一年發給一位教授的郵件中總結的那樣:“通過過去的五年,我發現我更願意做一個科研的旁觀者,作為一個新研究的持續不斷的生產者的負擔對我來說實在是太重了。”

\emph{問:我認為博士生要上課,教課,經歷磨人的資格考試,還要做很多和科研無關的事。為什麼你沒有描述其中任何一個呢?}

因為這些是都和獲得博士學位無關的活動。我選擇排他性地把我的回憶錄聚焦在做科研的過程上,因為這才是博士經歷的核心。

\emph{問:你在研究生六年期間交了朋友嗎,或者參加社交活動了嗎?}

答案是肯定的,不過你可能對我個人生活的細節不太感興趣。再次說明,我排他性地把我的回憶錄聚焦在做科研的過程上。

\emph{問:這本書裏的其他人物都是單一維度的。為什麼你不把他們的背景介紹得更細一點呢?}

因為我覺得代表別人發表觀點不太好。我可以自信地談論我自己的感受。我的同事可以寫他們自己對\emph{\bookname}的理解和響應,我也很樂意到時候給出他們文章的鏈接。

\emph{問:你怎麼知道你博士階段研磨超過了一萬個小時?}

我采取了一個非常保守的估算方式,每天研磨5個小時,乘以每年工作的335天(有30 天是“假期”),再乘以6年,大約就是10000個小時。不過大多數博士生可能工作得比這個數目還要長。

感謝閱讀,歡迎給我發郵件告訴我更多的反饋和問題!

\clearpage
\clearpage

\end{document}
